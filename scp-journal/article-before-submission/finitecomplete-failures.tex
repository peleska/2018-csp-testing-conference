% ==========================================================================
\section{Finite Complete Test Suites for CSP Failures Refinement}
\label{sec:finitecompletefails}
% ==========================================================================

Here, we define our notion of tests for failures refinement, and then prove
completeness of our suite. Finally, we study to complexity of our approach by
identifying a bound on the number of tests we need in a complete suite.

% ==========================================================================
\subsection{Test Cases for Verifying CSP Failures Refinement}

% -------------------------------------------------------------------------
\paragraph{Test Definition and Basic Properties}
In the domain of process algebras, test cases are typically represented by
processes interacting concurrently with the
SUT~\cite{Hennessy:1988:ATP:50497}. Considering an (unknown) process that
represents the behaviour of the SUT, we say that tests synchronise with the
process for the SUT over its visible events and use some additional events
outside the SUT process's alphabet to express whether the test execution
passed or failed.
%%%, or if no verdict could be obtained.

For a given reference process $P$, its normalised transition graph
$$
G(P) = ( N, \ii n, \Sigma, t : N\times\Sigma \pfun N, r : N \fun \mathbb{P}\mathbb{P}(\Sigma)),
$$
and each integer $j\ge 0$, we define a test for failures refinement as shown
below.
%
\begin{eqnarray}
\label{eq:UFP}
U_F(j) & = & U_F(j,0,\ii n)
\\
U_F(j,k,n) & = & \big(e:(\Sigma - [n]^0)  \then \efail\then \Stop \big)
\label{eq:ufa}
\\ & & \extchoice \nonumber
\\ & & (\minhits(n) =   \varnothing  )    \&   \big( \epass \then \Stop \big)
\label{eq:ufb}
\\ & & \extchoice \nonumber
\\ & & (k < j) \& \big(e:[n]^0   \then U_F(j,k+1,t(n,e) \big)
\label{eq:ufc}
\\ & & \extchoice \nonumber
\\ & & (k = j \wedge \minhits(n)\neq\varnothing) \& \taba \big( \Intchoice_{H\in\minhits(n)} (e:H   \then \epass \then\Stop   )  \big)
\label{eq:ufd}
\end{eqnarray}

% --------------------------------------------------------------------------
\paragraph{Explanation of the Test Definition} A test \pagebreak is performed by running
$U_F(j)$ concurrently with any SUT process $Q$, synchronising over $\Sigma$.
So, a \emph{test execution} is a trace of the concurrent process
%\[
$Q\parallel[\Sigma] U_F(j)$.
%\]

It is assumed that the events $\efail$ and $\epass$, indicating verdicts FAIL
and PASS for the test execution, are not included in $\Sigma$. Since we
assume that $Q$ is free of livelocks, it is guaranteed that events $\efail$
or $\epass$ always become visible, if they are the only events $U_F(j)/s$ is
ready to engage in: if $U_F(j)/s$ can only produce $\epass$ or $\efail$, the
occurrence of these events can never be blocked due to a livelock in $Q$
occurring in the same step of the execution.


The test is \emph{passed} by the SUT (written $Q\ \pass\ U_F(j)$) if, and
only if, {\it every} execution of $Q\parallel[\Sigma] U_F(j)$ terminates with
the event $\epass$. This can also be  expressed by means of a failures
refinement as defined below.
%
\begin{equation}
\label{eq:passF}
Q\ \pass\ U_F(j) \defs (\epass\then\Stop) \lessdet_F (Q\parallel[\Sigma] U_F(j)) \hide \Sigma
\end{equation}
%
This type of pass relation is often called \emph{must test}, because every
test execution must end with the $\epass$
event~\cite{Hennessy:1988:ATP:50497}.

It is necessary to use failures refinement in the definition above, and not
just traces refinement:~$(Q\parallel[\Sigma] U_F(j)) \hide \Sigma$ may have
the same visible traces $\varepsilon$ and $\epass$ as the ``Test Passed
Process'' $(\epass\then\Stop)$. However, the former may nondeterministically
refuse $\epass$, due to a deadlock occurring when a faulty SUT process
executes concurrently with $U_F(j,k,n)$ executing branch~(\ref{eq:ufd}), when
the guard condition $(k = j\wedge \minhits(n)\neq\varnothing)$ evaluates to
$\ist$. This is explained further in the next paragraphs. Alternatively, a
faulty SUT $Q$ might internally deadlock after a trace $s$ where $\#s < j$
and $\minhits(G(P)/s) \neq \varnothing$, so that the process
$(Q\parallel[\Sigma] U_F(j))/s$ deadlocks as well.

Intuitively, $U_F(j)$ is able to perform any trace $s$ of $P$, up to a length
$j$. If, after having already run through $s$ with $\#s \le j$, the SUT
accepts an event outside the initials of $P/s$
 (recall from Lemma~\ref{lemma:ufproperties} that $[n]^0 = [P/s]^0$ for $U_F(j)/s$),
the test immediately terminates with FAIL-event $\efail$. This is handled by
the branch (\ref{eq:ufa}) of the external choice. %in the process
%$U_F(j,\#s,n)$ defined above.

If $P/s$ is the $\Stop$ process or has $\Stop$ as an internal choice, this is
revealed by $\minhits(G(P)/s) = \varnothing$ (recall
(\ref{eq:minhitminaccempty}) and Lemma~\ref{lemma:ufproperties}). In this
case, the test may terminate successfully (branch (\ref{eq:ufb}) of the
choice in $U_F(j,\#s,G(P)/s)$). If $P/s$ may also nondeterministically engage
into events, branch (\ref{eq:ufc}) is simultaneously enabled. If $Q/s$ is
able to engage into an event in $\Sigma - [P/s]^0$, a test execution exists
where $U_F(j,\#s,G(P)/s)$ branches into (\ref{eq:ufa}) and produces the
$\efail$ event.

If the length of $s$ is still less than $j$, the test accepts any event $e$
from the initials $[P/s]^0 = [G(P)/s]^0$ and continues recursively as
$U_F(j,\#s+1,G(P)/s.e)$ in branch~(\ref{eq:ufc}); this follows from
Lemma~\ref{lemma:ufproperties}~(note that $G(P)/s.e = t(G(P)/s,e)$). A test
of this type is called \emph{adaptive}, because it accepts any legal
behaviour of the SUT, here any event from $[P/s]^0$, and adapts its
consecutive behaviour to the event selected by the SUT, here
$U_F(j,\#s+1,G(P)/s.e)$.

Now suppose that a test execution has run through a trace $s\in\trc(P)$ of
length $j$, so that $U_F(j)/s = U_F(j,j,n)$ with $n = G(P)/s$. If
$\minhits(n)\neq \varnothing$, the test changes its behaviour:~instead of
offering {\it all} legal events from $[n]^0$ to the SUT, it
nondeterministically chooses a minimal hitting set $H\in \minhits(n)$ and
only offers the events contained in $H$. If the SUT refuses to engage into
some event of $H$, this reveals a violation of failures refinement:~according
to Lemma~\ref{lemma:hseta}, a conforming SUT should accept at least one event
of each minimal hitting set in $\minhits(n)$. Therefore, the test execution
terminates with  $\epass$, only if such an event is accepted. Otherwise, it
deadlocks, and the test fails.

The specification of $U_F(j,k,n)$ implies that the test always stops after
having engaged into a trace $s\in\trc(Q)$ of maximal length $j$ or $j+1$. If
branch (\ref{eq:ufa}) is the last to be entered, the maximal length of $s$ is
$j+1$, and the test execution stops with $\efail$. If branch (\ref{eq:ufb})
is the last to be entered, the maximal length of $s$ is $j$, and the
execution stops with $\epass$. If branch (\ref{eq:ufd}) is the last to be
entered, then there are two possibilities. The first is that the process
accepts another event $e$ of some minimal hitting set $H\in\minhits(n)$ with
$n = G(P)/s$ according to Lemma~\ref{lemma:ufproperties}. In this case, the
final length of $s$ is $j+1$, and the execution terminates with $\epass$.
Alternatively, the test execution $(Q\parallel[\Sigma] U_F(j))/s$ deadlocks,
the final length of $s$ is $j$, and the execution stops without a PASS or
FAIL event. Such an execution is also interpreted as FAIL, because it reveals
that $(\epass\then\Stop) \not\lessdet_F (Q\parallel[\Sigma] U_F(j)) \hide
\Sigma$.

We observe that the number of possible executions of $Q\parallel[\Sigma]
U_F(j)$ is finite, because the number of traces $s$ with maximal length
$(j+1)$ is finite and the sets $[n]^0$, $(\Sigma - [n]^0)$, and $\minhits(n)$
are finite. Moreover, we further recall that $\minhits(n)$ may be empty, in
which case the indexed internal choice in (\ref{eq:ufd}) would be undefined.
The guard in that branch, however, requires $\minhits(n)\neq\varnothing$, and
branches (\ref{eq:ufa}) or (\ref{eq:ufb}) can be taken in this situation.

The following lemma establishes relationships between $U_F(j)$ and the
reference process $P$ from which it is derived.

\begin{lemma}\label{lemma:ufproperties}
If $s\in\trc(P)$ satisfies $\#s\le j$, then
$s, s.e\in\trc(U_F(j))$ for all $e\in\Sigma$, and the following properties hold.
\begin{eqnarray}
\label{eq:ifpa}
  &  & U_F(j)/s = U_F(j,\#s,G(P)/s)
\\
\label{eq:ifpb}
e\not\in [P/s]^0 & \implies & U_F(j)/s.e = (\efail\then\Stop)
\\
\label{eq:ifpc}
U_F(j)/s = U_F(j,\# s,n)  & \implies & [n]^0 = [P/s]^0
\\
\label{eq:ifpd}
U_F(j)/s = U_F(j,\# s,n)  & \implies & \minhits(n) = \minhits(P/s)
\end{eqnarray}
\end{lemma}
\begin{proof}
We prove (\ref{eq:ifpa}) by induction over the length of $s$. For $\#s = 0$,
the statement holds because $U_F(j)$ starts with the initial node $\ii n$ of
$G(P)$. Suppose that the statement holds for all traces $s$ with length $\# s
\le k < j$, so that $U_F(j)/s = U_F(j,\#s,G(P)/s)$. Now let $s.e$ be a trace
of $P$, so that $e\in [P/s]^0$. Since $[G(P)/s]^0 = [P/s]^0$ for all traces
$s$ of $P$, we conclude that $e\in  [G(P)/s]^0$, so $U_F(j,\#s,G(P)/s)$ can
engage into $e$ by executing branch (\ref{eq:ufc}). Since $t$ is the
transition function of $G(P)$ and $e\in [G(P)/s]^0$, $t(G(P)/s,e)$ is
defined, and $t(G(P)/s,e) = G(P)/s.e$. So, the new recursion in branch
(\ref{eq:ufc}) is so that $U_F(j)/s.e = U_F(j,\#s,G(P)/s)/e =
U_F(j,\#s+1,G(P)/s.e)$ as required.

To prove (\ref{eq:ifpb}), we apply (\ref{eq:ifpa}) to conclude that $U_F(j)/s
= U_F(j,\#s,G(P)/s)$, because $s$ is a trace of $P$. Noting again that
$[G(P)/s]^0 = [P/s]^0$, this implies that $e\not\in [G(P)/s]^0$, so
$U_F(j,\#s,G(P)/s)$ can engage in $e$ by entering branch (\ref{eq:ufa}). The
specification of this branch implies that $U_F(j)/s.e = U_F(j,\#s,G(P)/s)/e =
(\efail\then\Stop)$.

Statement (\ref{eq:ifpc}) follows trivially from (\ref{eq:ifpa}), because
$[G(P)/s]^0 = [P/s]^0$ for all traces $s$ of $P$. Finally, statement
(\ref{eq:ifpd}) follows trivially from (\ref{eq:ifpa}), because, according to
(\ref{eq:minhitsGP}), $\minhits(G(P)/s) = \minhits(P/s)$ for all traces of
$P$. \xbox
\end{proof}
%
Note that it is not guaranteed for $U_F(j)$ to run through the traces $s,
s.e$ in Lemma~\ref{lemma:ufproperties}, if $\minhits(P/u) = \varnothing$ for
some prefix $u$ of $s$: in such a case, $U_F(j)$ may stop with a $\epass$
event by entering branch (\ref{eq:ufb}). Therefore,
Lemma~\ref{lemma:ufproperties} just states the existence of
$U_F(j)$-executions $s, s.e$ satisfying the properties stated there.

% -------------------------------------------------------------------------
\paragraph{Complete Testing Assumption} As explained above, passing a test case
$U_F(j)$ requires that none of the possible executions $(Q\parallel[\Sigma]
U_F(j))$ stops after $\efail$ or stops without having produced the event
$\epass$. Therefore, it is necessary to determine whether all possible
executions have been covered in the repeated runs of $(Q\parallel[\Sigma]
U_F(j))$. The theoretical completeness results are, therefore, based on a
\emph{complete testing
assumption}~\cite{hierons_testing_2004,DBLP:conf/icfem/CavalcantiG07}, which
means that every possible behaviour of the SUT is performed after a finite
number of test executions. In practice, this is realised by executing each
test several times, recording the traces that have been performed, and using
hardware or software coverage analysers to determine whether all possible
behaviours of the SUT have been observed. Therefore, testing nondeterministic
SUTs comes at the price of having to apply some grey-box testing techniques.
%enabling us to decide whether all SUT behaviours have been observed.

% =========================================================================
\subsection{A Finite Complete Test Suite for Failures Refinement}
% =========================================================================

A CSP \emph{fault model} ${\cal F} = (P,\sqsubseteq,{\cal D})$ consists of a
reference process $P$, a conformance relation $\sqsubseteq \in \{\lessdet_T,
\lessdet_F\}$, and a fault domain ${\cal D}$, which is a set of CSP processes
over $P$'s alphabet that may or may not conform to $P$. A test suite $\TS$ is
called \emph{complete} with respect to fault model ${\cal F}$, if, and only
if, the following conditions are fulfilled.
\begin{description}
\item[1.~Soundness] If $P \sqsubseteq Q$, then $Q$ passes all tests in $\TS$.
\item[2.~Exhaustiveness] If $P \not\sqsubseteq Q$ and $Q\in{\cal D}$,
then $Q$ fails at least one test in $\TS$.
\end{description}
%
The following main theorem establishes the completeness of our test suite.

% -------------------------------------------------------------------------
\begin{theorem}\label{th:failurestest}
Let $P$ be a non-terminating, divergence-free CSP process over alphabet $\Sigma$ whose
normalised transition graph $G(P)$ has $p$ states. Define fault domain ${\cal
D}$ as the set of all divergence-free CSP processes over alphabet $\Sigma$,
whose transition graph has at most $q$ states with $q \ge p$. Then the test
suite
\[
\TS_F = \{ U_F(j)~|~0 \le j < pq  \}\quad\text{with $U_F(j)$ as specified in (\ref{eq:UFP})}
\]
is complete with respect to ${\cal F} = (P,\lessdet_F,{\cal D})$.
\end{theorem}
% ------------------------------------------------------------------------
%
The proof of the theorem follows directly from the two lemmas below. The
first lemma establishes that test suite $\TS_F$ is sound, and the second
establishes that the suite is also exhaustive.
%
\begin{lemma}\label{lemma:mainfsound}
A test suite $\TS_F$ generated from a CSP process $P$, as specified in
Theorem~\ref{th:failurestest}, is passed by every CSP process $Q$ satisfying
$P\lessdet_F Q$.
\end{lemma}
\begin{proof}{~}We make two points in separate steps below. The
first is that the test execution cannot reach branch~(\ref{eq:ufa}) and raise
a $fail$ event.  The second is that it cannot deadlock without raising a
$pass$ event. This case would also be interpreted as FAIL, since then
$\epass\then\Stop$ is not failures refined by $(Q\parallel[\Sigma]
U_F(j))\hide \Sigma$.

\paragraph{Step~1} Suppose that $P\lessdet_F Q$, so $P\lessdet_T Q$ and $Q\
conf\ P$ according to (\ref{eq:failconf}). Since   $\trc(Q)\subseteq
\trc(P)$, any adaptive test $U_F(j)$ running in parallel with $Q$ will always
enter the branches (\ref{eq:ufb}), (\ref{eq:ufc}), or (\ref{eq:ufd}) of the
external choice construction for $U_F(j,k,n)$. To see this, consider
 $U_F(j,k,n) = U_F(j)/s$
with $s\in\trc(Q)$. Lemma~\ref{lemma:ufproperties} implies $U_F(j,k,n) =
U_F(j,k,G(P)/s)$, so $[n]^0 = [G(P)/s]^0 = [P/s]^0$. As a consequence,
$[Q/s]^0\subseteq [P/s]^0 = [n]^0$, so branch~(\ref{eq:ufa}) can never be
entered in the parallel execution of $Q$ and $U_F(j)$, and the $fail$ event
cannot occur.

\paragraph{Step~2} For proving that a test execution can never deadlock without a
$\epass$ event, it has to be shown that a test execution can neither block at
branch (\ref{eq:ufc}) nor at branch (\ref{eq:ufd}). These cases are
considered separately below.

\paragraph{Step~2.1} Suppose that the test execution blocks at branch (\ref{eq:ufc})
after having run through a trace $s$ with $\#s < j$. Since  $P\lessdet_T Q$
by assumption, $s$ is a trace of $P$, thus $U_F(j)/s = U_F(j,\#s,G(P)/s)$
according to Lemma~\ref{lemma:ufproperties}. Therefore, $U_F(j)/s$  can enter
branch (\ref{eq:ufc}) with any event from $[G(P)/s]^0$. Since we assume that
$(Q\parallel[\Sigma] U_F(j))/s$ deadlocks, this means that $[G(P)/s]^0$ is
{\it not} a hitting set of $\minaccs(Q/s)$, because otherwise at least one
$e\in [G(P)/s]^0$ would be accepted by $Q/s$ and the test execution would not
deadlock. Now suppose that $\minhits(G(P)/s) = \varnothing$. Then   branch
(\ref{eq:ufb}) can be entered, and the test stops after $\epass$. Otherwise,
if $\minhits(G(P)/s) \neq \varnothing$, let $H\in\minhits(G(P)/s)$. Since $H$
contains only elements that are contained in some minimal acceptance of
$P/s$, and all these minimal acceptances are subsets of $[G(P)/s]^0$, $H$ is
a subset of $[G(P)/s]^0$ as well. Since $[G(P)/s]^0$, however, is not a
hitting set of $\minaccs(Q/s)$, also $H$ is not a hitting set of
$\minaccs(Q/s)$. Now this is a contradiction to Lemma~\ref{lemma:hseta},
since $Q\ conf\ P$ by assumption, so $H$ should also be a (not necessarily
minimal) hitting set in $\minaccs(Q/s)$. This proves that the test execution
cannot block  at branch (\ref{eq:ufc}) without being able to pass the test by
entering branch (\ref{eq:ufb}).

\paragraph{Step~2.2} Suppose that the execution blocks at
branch (\ref{eq:ufd}) after having run through some $s\in\trc(Q)\subseteq
\trc(P)$ with $\#s = j$. From Lemma~\ref{lemma:ufproperties} we know that
$U_F(j)/s = U_F(j,k,n) = U_F(j,\#s,G(P)/s)$, so $\minhits(n) =
\minhits(P/s)$. Branch (\ref{eq:ufb}) of $U_F(j,k,n)$ leads always to a PASS
verdict and is taken if $\minhits(n) = \varnothing$. If $\minhits(n) \neq
\varnothing$,  the assumption that $(Q\parallel[\Sigma] U_F(j))/s$ blocks at
branch (\ref{eq:ufd}) implies that there exists some $H\in\minhits(n)$ that
is not a hitting set of $\minaccs(Q/s)$. Again, by Lemma~\ref{lemma:hseta},
this contradicts the assumption that $Q\ conf\ P$. As a
consequence, the test execution can never deadlock at branch (\ref{eq:ufd})
without entering branch (\ref{eq:ufb}) and passing the test.

\bigskip \noindent%
Note that the line of reasoning in this proof requires that $Q$ is free of
livelocks, because otherwise a $\epass$ event might not become visible, due
to unbounded sequences of hidden events performed by $Q$. \xbox
\end{proof}
%
\begin{lemma}\label{lemma:mainfexhaustive}
A test suite $\TS_F$ specified as in Theorem~\ref{th:failurestest} is
exhaustive for the fault model specified there.
\end{lemma}
\begin{proof}
Consider a process $Q\in{\cal D}$ with $P\not\lessdet_F Q$. According to
(\ref{eq:failconf}), this non-conformance can be caused in two possible ways
corresponding to the cases $P\not\lessdet_T Q$ and $\neg(Q\ conf\ P)$. These
cases can be characterised as follows:
\begin{description}
\item[Case~1] $\trc(Q)\not\subseteq \trc(P)$
\item[Case~2] There exists a joint trace $s\in\trc(Q)\cap\trc(P)$ and a
    minimal acceptance $A_Q$ of $\minaccs(Q/s)$, such that (see
    Lemma~\ref{lemma:tgtrcref}, (\ref{eq:failrefb}))
\begin{equation}
\label{eq:accsnotcontained}
\forall A_P\in\minaccs(P/s): A_P\not\subseteq A_Q,
\end{equation}
\end{description}
It has to be shown for each of these cases that at least one test execution
of some $(Q\parallel[\Sigma] U_F(j))$ with $j < pq$ ends with the $\efail$
event or deadlocks. We do this by analysing the product graph of the
reference process $P$ and the SUT process $Q$: any trace
$s\in\trc(Q)\cap\trc(P)$ gives rise to a path labelled by the events of $s$
through this product graph. Any error can be detected after running through
such a trace and then either observing an event outside $[P/s]^0$ (the
violation described by Case~1) or identifying an illegal acceptance $A_Q$ (as
in Case~2). It is not guaranteed, however, that $s$ is short enough to be
executed by one of the test cases $U_F(j)$ with $0\le j < pq$. So, it has to
be shown that for any $s$ leading to an error situation, there exists a trace
$u$ of maximal length $pq-1$ leading to the same error.

\medskip
\noindent {\bf Case~1.} Consider a  trace $s.e\in\trc(Q)$ with $s\in\trc(P)$,
but $s.e\not\in\trc(P)$. Such a trace always exists because $\varepsilon$ is
a trace of every process. In this case, $s$ is also a trace of the product
graph $G = G(P)\times G(Q)$ defined in Section~\ref{sec:ntg}, and $G/s =
(G(P)/s,G(Q)/s)$ holds. The length of $s$ is not known, but from the
construction of $G$,  we know that $G$ has at most $pq$ reachable states,
because $G(P)$ has $p$ states, and $G(Q)$ has at most $q$ states. By
Lemma~\ref{lemma:reachproduc}, $(G(P)/s,G(Q)/s)$ can be reached by a trace
$u\in\trc(G)$ of length $\#u < pq$. Now the construction of the transition
function of $G$ implies that $u$ is also a trace of $P$ and $Q$, which means
that $(G(P)/s,G(Q)/s) = (G(P)/u,G(Q)/u)$. Since test $U_F(pq-1)$ accepts all
traces of $P$ up to length $pq-1$, $u$ is also a trace of this test, and, by
construction and by Lemma~\ref{lemma:ufproperties}, $U_F(pq-1)/u =
U_F(pq-1,\#u,G(P)/u)$. Since $s.e\not\in\trc(P)$, $e$ is an element of
$\Sigma-[P/u]^0 = \Sigma - [G(P)/s]^0$. Hence, in at least one execution,
$U_F(pq-1,\#u,G(P)/u)$ executes its first branch (\ref{eq:ufa}) with this
event $e$, so that the test fails. Again, the assumption of non-divergence of
$Q$ is needed for this conclusion.

\medskip
\noindent {\bf Case~2.} We note again that $s$ is a trace of the product
graph $G$, but we do not know its length. Again, by
Lemma~\ref{lemma:reachproduc}, the state $G/s$ can be reached by a trace
$u\in\trc(Q)\cap\trc(P)$ of maximal length $\#u < pq$. We consider the test
$U_F(\# u)$, for which $U_F(\# u)/u = U_F(\#u,\#u,G(P)/u)$, because of
Lemma~\ref{lemma:ufproperties}. $U_F(\#u)$ can always perform branch
(\ref{eq:ufc}) until the trace $u$ has been completely processed.
 $U_F(\#u,\#u,G(P)/u)$ may execute branches (\ref{eq:ufa}) or (\ref{eq:ufd})
only:~(\ref{eq:accsnotcontained}) implies that $P/s$ has at least one
non-empty minimal acceptance. By (\ref{eq:minhitminaccempty}) this is
equivalent to $\minhits(P/s) = \minhits(G(P)/s)\neq\varnothing$, and we
observe that $G(P)/s = G(P)/u$, so $\minhits(G(P)/u)\neq\varnothing$. As a
consequence, branch (\ref{eq:ufb}) cannot be taken because its guard
condition evaluates to $\isf$  for $U_F(\#u,\#u,G(P)/u)$. The guard condition
$(k < j)$ for branch (\ref{eq:ufc}) evaluates to $\isf$ for
$U_F(\#u,\#u,G(P)/u)$, too. If branch (\ref{eq:ufa}) is executed, the test
always fails. If branch (\ref{eq:ufd}) is executed, the test   deadlocks and
therefore fails for the execution where $Q/u$ selects the minimal acceptance
$A_Q$ as specified in (\ref{eq:accsnotcontained}) and $U_F(\#u,\#u,G(P)/u)$
selects a minimal hitting set $H\in\minhits(P/u)$ that has an empty
intersection with $A_Q$. The existence of such an $H$ is guaranteed because
of Lemma~\ref{lemma:hseta}. As a consequence, $(Q\parallel[\Sigma]U_F(\#
u))/u$ cannot produce the $\epass$ event in this execution; this means that
the test fails. The complete testing assumption guarantees that this
execution really occurs if $(Q\parallel[\Sigma]U_F(\# u))$ is executed
sufficiently often. This concludes the proof. \xbox
\end{proof}
%
Our notion of test can be specialised to deal with traces refinement~(see
Section~\ref{sec:finitecomplete}). We next present an example.

% ==========================================================================
