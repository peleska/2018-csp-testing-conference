% =========================================================================

\section{Introduction}
\label{sec:intro}

% =================================================================================

\paragraph{Motivation}
Model-based testing (MBT) is an active research field; results are currently
being evaluated and integrated into industrial verification processes by many
companies worldwide. This holds particularly for the embedded and
cyber-physical systems domains, where critical systems require rigorous
testing~\cite{jp2018ets,DBLP:conf/isola/0001BH18}.

In the safety-critical domain, test suites with guaranteed fault coverage are
of particular interest. For black-box testing, guarantees can be given only
if certain hypotheses are satisfied. These hypotheses are usually specified
by a \emph{fault domain}:~a set of models that may or may not conform to a
given reference model. \emph{Complete} test strategies guarantee to accept
every  system under test (SUT) conforming to the reference model, and uncover
every conformance violation, provided that the SUT behaviour is captured by a
member of the fault domain.

Generation techniques for complete test suites have been developed for
various formalisms; we mention here representative work for finite state
machines~\cite{hierons_testing_2004,simao_reducing_2012}, timed
automata~\cite{Springintveld2001}, {\sf\em
Circus}~\cite{DBLP:journals/acta/CavalcantiG11}, Timed
CSP~\cite{Schneider:1995:OST:203471.203475}, general labelled transition
systems~\cite{DBLP:journals/cn/Tretmans96}, symbolic state
machines~\cite{DBLP:conf/icst/Petrenko16}, and Kripke
structures~\cite{Huang2017}. In this article, we tackle \emph{Communicating
Sequential Processes~(CSP)}~\cite{Hoare:1985:CSP:3921,Roscoe2010}. This is a
mature process algebra that has been shown to be well-suited for the
description of reactive control systems in many publications over almost five
decades. Many of these applications are described in~\cite{Roscoe2010} and in
the references there. Industrial success has also been reported; see, for
example, \cite{976937,DBLP:conf/prdc/ShiPK99,DBLP:conf/amast/ButhKPS97}.

% ==================================================================================

\paragraph{Main Contributions}

This article presents   complete black-box test suites for software and
systems modelled using CSP. They can be generated for non-terminating,
divergence-free, finite-state CSP processes with finite alphabets,
interpreted both in the trace and the failures semantics. Divergence freedom
is usually assumed in black-box testing, since it cannot distinguish between
divergence and deadlock using testing.

\newpage
\noindent %
The main novel contributions in this article may be summarised as
follows.
\begin{enumerate}
\item It is shown that trace or failures conformance can be established
    with finitely many test cases, provided suitable fault domains are
    chosen, so that the true behaviour of the SUT is reflected by members
    of these domains.

\item The definition of these fault domains is based on the well-known normalised transition graphs~\cite{Roscoe:1994:chapter} representing the  trace and failures semantics of finite-state CSP processes. A fault domain contains all CSP processes over a given alphabet, whose normalised transition graphs have at most $q$ nodes for some $q\in\mathbb{N}$.

\item Worst-case complexity bounds for the number of test executions to be performed are given.

\item It is shown that the maximal length of the test traces involved cannot be further reduced without losing the test suite's completeness property.

\item Likewise, is shown that the non-deterministic behaviour of the SUT cannot be
checked for admissibility with smaller or fewer sets of events.
\end{enumerate}

% ==================================================================================

\paragraph{Related Work}
Our results complement and extend work previously published
in~\cite{Hennessy:1988:ATP:50497,DBLP:conf/fm/PeleskaS96,peleska1997a,DBLP:conf/icfem/CavalcantiG07,DBLP:conf/pts/CavalcantiS17}.
None of these provide sufficient conditions for constructing finite complete
test suites. So, they also do not provide complexity bounds on the number of
test executions needed to establish conformance between an SUT and a
reference process. In~\cite{DBLP:conf/pts/CavalcantiS17}, fault domains are
used, but these contain all processes refining a ``top'' fault domain
process. This concept is orthogonal to the one investigated here:~members of
our fault domain need not be in refinement relation to any other process in
the domain. They just adhere to the same upper bound $q$ of nodes in their
normalised transition graphs.

The minimal sets of events used for checking nondeterministic behaviour of the SUT used in our article were already suggested in~\cite{DBLP:conf/fm/PeleskaS96,peleska1997a,DBLP:conf/icfem/CavalcantiG07}; in the present article, however, they have been identified for the first time as {\it minimal hitting sets}~\cite{5533149} of minimal acceptances in a given process state, and we establish an upper bound stating how many of these sets need to be checked for the ``most extreme form of nondeterminism'' that may be exhibited by the SUT.

In~\cite{Hennessy:1988:ATP:50497,DBLP:conf/icfem/CavalcantiG07,DBLP:conf/pts/CavalcantiS17},
the authors devised linear test cases:~after running through a preset trace
$s$, test cases for traces refinement check for illegal acceptance of a
specific event $e$, and test cases verifying  nondeterministic behaviour
check for  acceptance of events from some set $A$. In the present article, we
follow the alternative approach proposed
in~\cite{DBLP:conf/fm/PeleskaS96,peleska1997a} and use {\it adaptive} test
cases. This means that each test case adapts its trace execution to the
nondeterministic behaviour of the SUT, checks for trace violations at any
point during the test execution, and checks for the acceptance of a given
minimal hitting set of events after any legal trace of a test case-specific
length.

The adaptive test cases have the advantage that test executions only lead to
an inconclusive result if the reference process allows for a nondeterministic
choice between deadlock and trace continuation in a certain state. In
contrast to this, the linear test cases may lead to many more futile
executions with inconclusive results, if the SUT refuses to engage into the
next event $e$ from the preset trace $s$, due to legal nondeterministic
choices leading to a refusal of $e$. Moreover, the preset traces $s$ need to
be executed twice according to the strategies devised in
\cite{Hennessy:1988:ATP:50497,DBLP:conf/icfem/CavalcantiG07,DBLP:conf/pts/CavalcantiS17},
because traces refinement and correctness of nondeterministic behaviour are
checked by two different sets of test cases.

The approach  to specifying fault domains by means of normalised transition
graphs has been inspired by the typical method used in the construction of
complete test suites for finite state machines (FSMs). There, fault domains
typically contain all FSMs over a given alphabet whose number of states does
not exceed a given value
$q$~\cite{chow:wmethod,vasilevskii1973,luo_test_1994}.

%There,
%fault domains are specified as collections of processes refining a ``most
%general'' fault domain member. With that concept, complete test suites may be
%finite or infinite. This result gives important insight into the theory of
%fault-domain testing for CSP, but we are particularly interested in {\it
%finite} suites when it comes to practical application.
%While~\cite{DBLP:conf/pts/CavalcantiS17} may require additional criteria to
%select tests from still infinite test suites, here, we further restrict fault
%domains using a graph representation of processes (originally elaborated for
%model checking) to obtain finite test suites.
%
%Our approach to the definition of CSP fault domains is presented in this
%paper. We take advantage of results on model checking of CSP
%processes, where the failures semantics of a finite-state CSP process is
%represented as a finite normalised transition graph, whose edges are labelled
%by the events the process engages in, and whose nodes are labelled by minimal
%acceptances or, alternatively, maximal
%refusals~\cite{Roscoe:1994:chapter}. The maximal refusals express the
%degree of nondeterminism present in a process state that is in
%one-one-correspondence to a node of the normalised transition graph. Inspired
%by the way fault-domains are specified for finite state machines (FSMs),
%we define them here as the set of CSP processes whose normalised transition
%graphs do not exceed the order (that is, the number of nodes) of the
%reference model's graph by more than a given constant.
%
%The main contribution of this paper is the construction of two test suites to
%verify the conformance relations \emph{trace refinement} and \emph{failures
%refinement} for any given non-terminating, non-divergent, finite-state,
%finite alphabet CSP process. We prove their completeness with respect to
%fault domains of the described type.  The existence of -- possibly infinite
%-- complete test suites has been established for process algebras, and for
%CSP in particular, by several
%authors~\cite{Hennessy:1988:ATP:50497,Schneider:1995:OST:203471.203475,DBLP:conf/fm/PeleskaS96,peleska1997a,DBLP:conf/icfem/CavalcantiG07,DBLP:conf/pts/CavalcantiS17}.
%To the best of our knowledge, this article is the first to present {\it
%finite}, complete test suites associated with this class of fault domains and
%conformance relations.
%
%Our result is not a simple transcription of existing knowledge about finite,
%complete test suites for FSMs. The capabilities of CSP to express
%nondeterminism in a more fine-grained way than it is possible for FSMs
%requires a more complex approach to testing for conformity in the failures
%model than required for model-based testing against nondeterministic FSMs, as
%published, for example,
%in~\cite{hierons_testing_2004,DBLP:conf/hase/PetrenkoY14}. CSP distinguishes
%between external choice, where the environment can control the behaviour of a
%process, and internal choice where the behaviour is decided internally and
%cannot be influenced by the environment. In contrast, FSMs specify
%nondeterministic behaviour by offering more than one possible output for a
%given input, and this output can be controlled by an FSM representing the
%environment. Loosely speaking, this corresponds to external choice in CSP,
%while there is no equivalent to internal choice in nondeterministic FSMs.
%
%Finally, we prove two optimality results here. (1)~We show that our approach
%to testing the admissibility of an SUT's nondeterministic behaviour is
%optimal in the sense that it cannot be established with fewer ``probes''
%investigating the SUT's degree of nondeterminism. These probes are minimal
%sets of events offered by the tests to the SUT, such that a refusal of the
%SUT to engage into at least one of these events reveals a violation of
%failures refinement. (2)~Furthermore, it is shown that the maximal length of
%traces to be investigated in a complete test suite cannot be further reduced
%without loosing the suite's capability to uncover any violation of trace or
%failures refinement.

% ==================================================================================

\paragraph{Overview} In Section~\ref{section:preliminaries}, we present the
background relevant to our work. In Section~\ref{sec:finitecompletefails},
finite complete test suites for verifying failures refinement are presented.
A sample test suite is presented in Section~\ref{sec:case}. Test suites
checking traces refinement are a simplified version of the former class; they
are presented in Section~\ref{sec:finitecomplete}. The optimality results are
presented in Section~\ref{sec:complexity}, together with further complexity
considerations. Our results are discussed in Section~\ref{sec:conc}, where we
also conclude. References to further related work are given throughout the
paper where appropriate.

% ==================================================================================
