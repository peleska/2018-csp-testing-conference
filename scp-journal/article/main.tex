%%%%%%%%%%%%%%%%%%%%%%%%%%%%%%%%%%%%%%%%%%%%%%%%%%%%%%%%%%%%%%%%%%%
%%%%%%%%%%%%%%%%%%%%%%%%%%%%%%%%%%%%%%%%%%%%%%%%%%%%%%%%%%%%%%%%%%%

\documentclass[3p,times]{elsarticle}

%%%%%%%%%%%%%%%%%%%%%%%%%%%%%%%%%%%%%%%%%%%%%%%%%%%%%%%%%%%%%%%%%%%
%%%%%%%%%%%%%%%%%%%%%%%%%%%%%%%%%%%%%%%%%%%%%%%%%%%%%%%%%%%%%%%%%%%

%%%%%%%%%%%%%%%%%%%%%%%%%%%%%%%%%%%%%%%%%%%%%%%%%%%%%%%%%%%%%%%%%%%

\journal{Science of Computer Programming}

%%%%%%%%%%%%%%%%%%%%%%%%%%%%%%%%%%%%%%%%%%%%%%%%%%%%%%%%%%%%%%%%%%%

%%%%%%%%%%%%%%%%%%%%%%%
%% Elsevier bibliography styles
%%%%%%%%%%%%%%%%%%%%%%%
%% To change the style, put a % in front of the second line of the current style and
%% remove the % from the second line of the style you would like to use.
%%%%%%%%%%%%%%%%%%%%%%%

%% Numbered
%\bibliographystyle{model1-num-names}

%% Numbered without titles
%\bibliographystyle{model1a-num-names}

%% Harvard
%\bibliographystyle{model2-names.bst}\biboptions{authoryear}

%% Vancouver numbered
%\usepackage{numcompress}\bibliographystyle{model3-num-names}

%% Vancouver name/year
%\usepackage{numcompress}\bibliographystyle{model4-names}\biboptions{authoryear}

%% APA style
%\bibliographystyle{model5-names}\biboptions{authoryear}

%% AMA style
%\usepackage{numcompress}\bibliographystyle{model6-num-names}

%% `Elsevier LaTeX' style
\bibliographystyle{elsarticle-num}
%%%%%%%%%%%%%%%%%%%%%%%

\usepackage[cmex10]{amsmath}
\usepackage{amssymb}
\setcounter{tocdepth}{3}
\usepackage{graphicx}
%
\usepackage{url}
\urldef{\mailsa}\path|{peleska,huang}@uni-bremen.de|
\urldef{\mailsb}\path|ana.cavalcanti@york.ac.uk|
\urldef{\mailsc}\path|adenilso@icmc.usp.br|

\newcommand{\keywords}[1]{\par\addvspace\baselineskip
\noindent\keywordname\enspace\ignorespaces#1}

\usepackage{hyperref}

%\usepackage{float}
%\usepackage{lscape}
%\usepackage{lscape}
%
%\usepackage[linesnumbered,lined,boxed]{algorithm2e}
\usepackage[draft]{fixme}
%
\usepackage{zed-csp}
%
%\DeclareMathSymbol{\B}{\mathalpha}{AMSb}{"42}
%\DeclareMathSymbol{\I}{\mathalpha}{AMSb}{"49}
%\DeclareMathSymbol{\N}{\mathalpha}{AMSb}{"4E}
%\DeclareMathSymbol{\Pwr}{\mathalpha}{AMSb}{"50}
%\DeclareMathSymbol{\Q}{\mathalpha}{AMSb}{"51}
%\DeclareMathSymbol{\R}{\mathalpha}{AMSb}{"52}
%\DeclareMathSymbol{\Z}{\mathalpha}{AMSb}{"5A}
%\DeclareMathSymbol{\Sol}{\mathalpha}{AMSb}{"53}
%
%\newcommand{\obsv}[1]{{\cal O}(#1)}
%\newcommand{\ctrl}[1]{{\cal C}(#1)}
%\newcommand{\vdot}[1]{\stackrel{.}{#1}}
%
%\newcommand{\HS}{{\mathcal H}}
%
%\newcommand{\Interval}{\I}
%
\newcommand{\ist}{\mbox{{\tt true}}}
\newcommand{\isf}{\mbox{{\tt false}}}
%\newcommand{\emptytrace}{\langle~\rangle}
%\newcommand{\abegin}{\mathbf{begin}}
%\newcommand{\aend}{\mathbf{end}}
%\newcommand{\alet}{\mathbf{let}}
%\newcommand{\aendlet}{\mathbf{endlet}}
%\newcommand{\ain}{\mathbf{in}}
%\newcommand{\afor}{\mathbf{for}}
%\newcommand{\adownto}{\mathbf{downto}}
%\newcommand{\aforall}{\mathbf{foreach}}
%\newcommand{\awhile}{\mathbf{while}}
%\newcommand{\ado}{\mathbf{do}}
%\newcommand{\aod}{\mathbf{od}}
%\newcommand{\aenddo}{\mathbf{enddo}}
%\newcommand{\aif}{\mathbf{if}}
%\newcommand{\afi}{\mathbf{fi}}
%\newcommand{\athen}{\mathbf{then}}
%\newcommand{\aelse}{\mathbf{else}}
%\newcommand{\aelseif}{\mathbf{elseif}}
%\newcommand{\aendif}{\mathbf{endif}}
%\newcommand{\ainout}{\mathbf{inout}}
%\newcommand{\awhere}{\mathbf{where}}
%\newcommand{\areturn}{\mathbf{return}}
%\newcommand{\aout}{\mathbf{out}}
%\newcommand{\aprocedure}{\mathbf{procedure}}
%\newcommand{\afunction}{\mathbf{function}}
%\newcommand{\abreak}{\mathbf{break}}
%\newcommand{\Sup}[1]{\overline{#1}}
%\newcommand{\Inf}[1]{\underline{#1}}
%
\newcommand{\taba}{\hspace*{3mm}}
%\newcommand{\tabb}{\hspace*{6mm}}
\newcommand{\tabc}{\hspace*{9mm}}
\newcommand{\tabd}{\hspace*{12mm}}
%\newcommand{\tabe}{\hspace*{15mm}}
%\newcommand{\tabf}{\hspace*{18mm}}
%
%\newcommand{\gca}[1]{{{#1}^{\vartriangleright}}}
%\newcommand{\gcb}[1]{{{#1}^{\vartriangleleft}}}
%\newcommand{\gclr}{{{\vartriangleleft\atop\longleftarrow}\atop
%                   {\longrightarrow\atop\vartriangleright}}}
%
%\newcommand{\Nat}{{\mathbb N}}
%\newcommand{\Real}{{\mathbb R}}
%
%\newcommand{\trans}{\longrightarrow}
%\newcommand{\transp}{\longrightarrow_{\power}}
%\newcommand{\transl}{\longrightarrow_{L}}
%\newcommand{\transg}{\longrightarrow_{G}}
%\newcommand{\transcfg}[1]{\stackrel{#1}{\longrightarrow}_{CFG}}
%\newcommand{\isdefd}{=_{\mbox{\footnotesize def}}}
%\newcommand{\equivdef}{\equiv_{\mbox{\footnotesize def}}}
%\newcommand{\mitem}{\mbox{\em M-Item}}
%\newcommand{\currt}{\hat{t}}
%\newcommand{\sigmaa}{\sigma_A}
%\newcommand{\strictimplies}{\stackrel{\bullet}{\Rightarrow}}
%\newcommand{\trl}{/\!/}
%\newcommand{\q}{\textbf{q}}
%\newcommand{\eqc}[2]{[#1;#2]}
%\newcommand{\vest}{V_{\text{\sl est}}}
%\newcommand{\vmax}{V_{\text{\sl MRSP}}}
%\newcommand{\wout}{\mathsf{W}}
%\newcommand{\eout}{\mathsf{EB}}
%\newcommand{\areb}{\mathsf{allowRevokeEB}}
%\newcommand{\sbia}{\mathsf{SBAvailable}}
%\newcommand{\sbz}{\mathsf{sb}_0}
%\newcommand{\ticmd}{\mathsf{TICmd}}
%\newcommand{\dmicmd}{\mathsf{DMICmd}}
%
%\newcommand{\sob}{\mathsf{speedOnBoard}}
%\newcommand{\std}{\mathsf{speedToDriver}}
%\newcommand{\pstd}{\mathsf{permittedSpeedToDriver}}
%\newcommand{\csmsw}{\mathsf{csmSwitch}}
%\newcommand{\sbicmd}{\mathsf{sbiCmd}}
%\newcommand{\sbidisplay}{\mathsf{DMIdisplaySBI}}
%
%
%\newcommand{\dvw}{\mathsf{dV}_{\mathsf{warning}}}
%\newcommand{\dvs}{\mathsf{dV}_{\mathsf{sbi}}}
%\newcommand{\dve}{\mathsf{dV}_{\mathsf{ebi}}}
%
%
%\newcommand{\calcw}{\mathsf{dV\_warning(float)}}
%\newcommand{\calcs}{\mathsf{dV\_sbi(float)}}
%\newcommand{\calce}{\mathsf{dV\_ebi(float)}}
%\newcommand{\calcstd}{\mathsf{calc\_speed\_to\_driver()}}
%\newcommand{\calcpstd}{\mathsf{calc\_permitted\_speed\_to\_driver()}}
%\newcommand{\calcsob}{\mathsf{calc\_speed\_onboard()}}
%
\newcommand{\trc}{\text{\it traces}}
\newcommand{\fails}{\text{\it failures}}
%\newcommand{\qtrc}{\text{qtraces}}
%\newcommand{\iotrc}{\text{L}}
%
%
%%\newtheorem{definition}{Definition}
%%\newtheorem{property}{Property}
%%\newtheorem{lemma}{Lemma}
%%\newtheorem{theorem}{Theorem}
%%\newtheorem{corollary}{Corollary}
%%\newtheorem{property}{Property}
%
%%\newtheorem{note}{Note}[section]
%
\newcommand{\xbox}{\unskip\nobreak\hfil\penalty50
      \hskip2em\hbox{}\nobreak\hfil$\Box$%
      \parfillskip=0pt\finalhyphendemerits=0 \par%

      \medskip
      \parfillskip=0pt plus 1fil
}
%
%\newcommand{\dontshow}[1]{}
%\newcommand{\annot}[1]{\textbf{\textcolor{red}{#1}}}
%\newcommand{\modelsf}{\models_{\text{F}}}
%
%\newcommand{\ta}{\mathbf{A}}
%\newcommand{\te}{\mathbf{E}}
%\newcommand{\tx}{\mathbf{X}}
%\newcommand{\tf}{\mathbf{F}}
%\newcommand{\tg}{\mathbf{G}}
%\newcommand{\tu}{\mathbf{U}}
%\newcommand{\tr}{\mathbf{R}}
%\newcommand{\tw}{\mathbf{W}}
%
%\newcommand{\ttu}[1]{\mathbf{U}^{#1}}
%\newcommand{\ttg}[1]{\mathbf{G}^{#1}}
%\newcommand{\ttf}[1]{\mathbf{F}^{#1}}
%
\newcommand{\epass}{{\text{\it pass}}}
\newcommand{\efail}{{\text{\it fail}}}
%
\newcommand{\pass}{\underline{\text{pass}}}
% \newcommand{\tfail}{\underline{\text{fail}}}
%
%\newcommand{\xpass}[1]{{#1}_\text{pass}}
%\newcommand{\xfail}[1]{{#1}_\text{fail}}
%
%\newcommand{\tc}{{\text{TC}}}
\newcommand{\TS}{{\text{TS}}}
\newcommand{\ii}[1]{\underline{#1}}
%\newcommand{\inc}{\,{\rm I}\,}
%
%\newcommand{\scov}{\text{SCOV}}
%\newcommand{\after}{\text{-after-}}
%\newcommand{\reduction}{\preceq}
%
%\newcommand{\ps}{\le_s}
%\newcommand{\pss}[1]{\stackrel{#1}{\ps}}
%
%\newcommand{\pe}{\sim_s}
%\newcommand{\pes}[1]{\stackrel{#1}{\pe}}
\newcommand{\ol}{\overline}
%\newcommand{\diag}{\mathbf{diag}}
%\newcommand{\primem}{\mathbf{prime}}
%
%\newcommand{\ltl}{\text{\tt LTL}}
%
\newcommand{\accs}{\text{\it Acc}}
\newcommand{\minaccs}{\text{\it minAcc}}
%
\newcommand{\refs}{\text{\it Ref}}
\newcommand{\maxrefs}{\text{\it maxRef}}
\newcommand{\minhits}{\text{\it minHit}}
\newcommand{\minsub}{{\it min}_\subseteq}

%\newcommand{\failure}{{\text {\rm failures}}}

\newcommand{\prefs}{\text{\bf pref}}

\newcommand{\card}[1]{{~|\! #1\!|~}}

\newcommand{\pmax}{P_\text{max}}

\newtheorem{definition}{Definition}
\newtheorem{property}{Property}
\newtheorem{lemma}{Lemma}
\newtheorem{theorem}{Theorem}
\newtheorem{corollary}{Corollary}

\newcounter{examplectr}
\newenvironment{example}[1][]
{\refstepcounter{examplectr}

\medskip
\noindent
{\bf Example~\theexamplectr.}}     %%%\label{#1}
{
%\unskip\nobreak\hfil\penalty50
%      \hskip2em\hbox{}\nobreak\hfil$\Box$%
%      \parfillskip=0pt \finalhyphendemerits=0 \par
}


\newenvironment{proof}[1][]
{

\medskip
\noindent
{\bf Proof.\ }}
{
%\unskip\nobreak\hfil\penalty50
%      \hskip2em\hbox{}\nobreak\hfil$\Box$%
%      \parfillskip=0pt \finalhyphendemerits=0 \par
}

%%%%%%%%%%%%%%%%%%%%%%%%%%%%%%%%%%%%%%%%%%%%%%%%%%%%%%%%%%%%%%%%%%%%%%%%%%%%%%%%
%%%%%%%%%%%%%%%%%%%%%%%%%%%%%%%%%%%%%%%%%%%%%%%%%%%%%%%%%%%%%%%%%%%%%%%%%%%%%%%%

% To avoid formulas in $ $ to be broken across lines
\relpenalty=10000
\binoppenalty=10000

% ===========================================================================

\begin{document}

%%%%%%%%%%%%%%%%%%%%%%%%%%%%%%%%%%%%%%%%%%%%%%%%%%%%%%%%%%%%%%%%%%%

\begin{frontmatter}

\title{Finite Complete Suites for CSP Refinement Testing}
%\tnotetext[mytitlenote]{Fully documented templates are available in the elsarticle package on \href{http://www.ctan.org/tex-archive/macros/latex/contrib/elsarticle}{CTAN}.}

%% Group authors per affiliation:
\author{Jan Peleska, Wen-ling Huang \fnref{jpwlh}}
\address{Bremen, Germany}
\fntext[jpwlh]{University of Bremen, \mailsa}

\author{Ana Cavalcanti \fnref{ana}}
\address{York, United Kingdom}
\fntext[ana]{University of York, \mailsb}

%% or include affiliations in footnotes:
%\author[mymainaddress,mysecondaryaddress]{Elsevier Inc}
%\ead[url]{www.elsevier.com}
%
%\author[mysecondaryaddress]{Global Customer Service\corref{mycorrespondingauthor}}
%\cortext[mycorrespondingauthor]{Corresponding author}
%\ead{support@elsevier.com}
%
%\address[mymainaddress]{1600 John F Kennedy Boulevard, Philadelphia}
%\address[mysecondaryaddress]{360 Park Avenue South, New York}

%%%%%%%%%%%%%%%%%%%%%%%%%%%%%%%%%%%%%%%%%%%%%%%%%%%%%%%%%%%%%%%%%%%

\begin{abstract}
In this paper, new contributions for model-based testing using Communicating
Sequential Processes (CSP) are presented. For a finite non-terminating CSP
process representing the reference model, finite test suites for checking the
conformance relations traces and failures refinement are presented, and their
completeness (that is, capability to uncover conformity violations) is
proven. The fault domains for which complete failure detection can be
guaranteed are specified by means of normalised transition graphs
representing the failures semantics of finite-state CSP processes. While
complete test suites for CSP processes have been previously investigated by
several authors, a sufficient condition for their finiteness is presented
here for the first time. Moreover, it is shown that the test suites are
optimal in two aspects:~(a)~the maximal length of test traces cannot be
further reduced, and (b)~the nondeterministic behaviour cannot be tested with
smaller or fewer sets of events, without losing the test suite's completeness
property.
\end{abstract}

%%%%%%%%%%%%%%%%%%%%%%%%%%%%%%%%%%%%%%%%%%%%%%%%%%%%%%%%%%%%%%%%%%%

\begin{keyword}
Model-based testing, CSP\sep Traces Refinement\sep Failures Refinement\sep
Complete Test Suites
\end{keyword}

%%%%%%%%%%%%%%%%%%%%%%%%%%%%%%%%%%%%%%%%%%%%%%%%%%%%%%%%%%%%%%%%%%%

\end{frontmatter}

%%%%%%%%%%%%%%%%%%%%%%%%%%%%%%%%%%%%%%%%%%%%%%%%%%%%%%%%%%%%%%%%%%%
%%%%%%%%%%%%%%%%%%%%%%%%%%%%%%%%%%%%%%%%%%%%%%%%%%%%%%%%%%%%%%%%%%%

% =========================================================================

\section{Introduction}
\label{sec:intro}

% =================================================================================

\paragraph{Motivation}
Model-based testing (MBT) is an active research field; results are currently
being evaluated and integrated into industrial verification processes by many
companies worldwide. This holds particularly for the embedded and
cyber-physical systems domains, where critical systems require rigorous
testing~\cite{jp2018ets,DBLP:conf/isola/0001BH18}.

In the safety-critical domain, test suites with guaranteed fault coverage are
of particular interest. For black-box testing, guarantees can be given only
if certain hypotheses are satisfied. These hypotheses are usually specified
by a \emph{fault domain}:~a set of models that may or may not conform to a
given reference model. \emph{Complete} test strategies guarantee to accept
every  system under test (SUT) conforming to the reference model, and uncover
every conformance violation, provided that the SUT behaviour is captured by a
member of the fault domain.

Generation techniques for complete test suites have been developed for
various formalisms; we mention here representative work for finite state
machines~\cite{hierons_testing_2004,simao_reducing_2012}, timed
automata~\cite{Springintveld2001}, {\sf\em
Circus}~\cite{DBLP:journals/acta/CavalcantiG11}, Timed
CSP~\cite{Schneider:1995:OST:203471.203475}, general labelled transition
systems~\cite{DBLP:journals/cn/Tretmans96}, symbolic state
machines~\cite{DBLP:conf/icst/Petrenko16}, and Kripke
structures~\cite{Huang2017}. In this article, we tackle \emph{Communicating
Sequential Processes~(CSP)}~\cite{Hoare:1985:CSP:3921,Roscoe2010}. This is a
mature process algebra that has been shown to be well-suited for the
description of reactive control systems in many publications over almost five
decades. Many of these applications are described in~\cite{Roscoe2010} and in
the references there. Industrial success has also been reported; see, for
example, \cite{976937,DBLP:conf/prdc/ShiPK99,DBLP:conf/amast/ButhKPS97}.

% ==================================================================================

\paragraph{Main Contributions}

This article presents   complete black-box test suites for software and
systems modelled using CSP. They can be generated for non-terminating,
divergence-free, finite-state CSP processes with finite alphabets,
interpreted both in the trace and the failures semantics. Divergence freedom
is usually assumed in black-box testing, since it cannot distinguish between
divergence and deadlock using testing.

\newpage
\noindent %
The main novel contributions in this article may be summarised as
follows.
\begin{enumerate}
\item It is shown that trace or failures conformance can be established
    with finitely many test cases, provided suitable fault domains are
    chosen, so that the true behaviour of the SUT is reflected by members
    of these domains.

\item The definition of these fault domains is based on the well-known normalised transition graphs~\cite{Roscoe:1994:chapter} representing the  trace and failures semantics of finite-state CSP processes. A fault domain contains all CSP processes over a given alphabet, whose normalised transition graphs have at most $q$ nodes for some $q\in\mathbb{N}$.

\item Worst-case complexity bounds for the number of test executions to be performed are given.

\item It is shown that the maximal length of the test traces involved cannot be further reduced without losing the test suite's completeness property.

\item Likewise, is shown that the non-deterministic behaviour of the SUT cannot be
checked for admissibility with smaller or fewer sets of events.
\end{enumerate}

% ==================================================================================

\paragraph{Related Work}
Our results complement and extend work previously published
in~\cite{Hennessy:1988:ATP:50497,DBLP:conf/fm/PeleskaS96,peleska1997a,DBLP:conf/icfem/CavalcantiG07,DBLP:conf/pts/CavalcantiS17}.
None of these provide sufficient conditions for constructing finite complete
test suites. So, they also do not provide complexity bounds on the number of
test executions needed to establish conformance between an SUT and a
reference process. In~\cite{DBLP:conf/pts/CavalcantiS17}, fault domains are
used, but these contain all processes refining a ``top'' fault domain
process. This concept is orthogonal to the one investigated here:~members of
our fault domain need not be in refinement relation to any other process in
the domain. They just adhere to the same upper bound $q$ of nodes in their
normalised transition graphs.

The minimal sets of events used for checking nondeterministic behaviour of the SUT used in our article were already suggested in~\cite{DBLP:conf/fm/PeleskaS96,peleska1997a,DBLP:conf/icfem/CavalcantiG07}; in the present article, however, they have been identified for the first time as {\it minimal hitting sets}~\cite{5533149} of minimal acceptances in a given process state, and we establish an upper bound stating how many of these sets need to be checked for the ``most extreme form of nondeterminism'' that may be exhibited by the SUT.

In~\cite{Hennessy:1988:ATP:50497,DBLP:conf/icfem/CavalcantiG07,DBLP:conf/pts/CavalcantiS17},
the authors devised linear test cases:~after running through a preset trace
$s$, test cases for traces refinement check for illegal acceptance of a
specific event $e$, and test cases verifying  nondeterministic behaviour
check for  acceptance of events from some set $A$. In the present article, we
follow the alternative approach proposed
in~\cite{DBLP:conf/fm/PeleskaS96,peleska1997a} and use {\it adaptive} test
cases. This means that each test case adapts its trace execution to the
nondeterministic behaviour of the SUT, checks for trace violations at any
point during the test execution, and checks for the acceptance of a given
minimal hitting set of events after any legal trace of a test case-specific
length.

The adaptive test cases have the advantage that test executions only lead to
an inconclusive result if the reference process allows for a nondeterministic
choice between deadlock and trace continuation in a certain state. In
contrast to this, the linear test cases may lead to many more futile
executions with inconclusive results, if the SUT refuses to engage into the
next event $e$ from the preset trace $s$, due to legal nondeterministic
choices leading to a refusal of $e$. Moreover, the preset traces $s$ need to
be executed twice according to the strategies devised in
\cite{Hennessy:1988:ATP:50497,DBLP:conf/icfem/CavalcantiG07,DBLP:conf/pts/CavalcantiS17},
because traces refinement and correctness of nondeterministic behaviour are
checked by two different sets of test cases.

The approach  to specifying fault domains by means of normalised transition
graphs has been inspired by the typical method used in the construction of
complete test suites for finite state machines (FSMs). There, fault domains
typically contain all FSMs over a given alphabet whose number of states does
not exceed a given value
$q$~\cite{chow:wmethod,vasilevskii1973,luo_test_1994}.

%There,
%fault domains are specified as collections of processes refining a ``most
%general'' fault domain member. With that concept, complete test suites may be
%finite or infinite. This result gives important insight into the theory of
%fault-domain testing for CSP, but we are particularly interested in {\it
%finite} suites when it comes to practical application.
%While~\cite{DBLP:conf/pts/CavalcantiS17} may require additional criteria to
%select tests from still infinite test suites, here, we further restrict fault
%domains using a graph representation of processes (originally elaborated for
%model checking) to obtain finite test suites.
%
%Our approach to the definition of CSP fault domains is presented in this
%paper. We take advantage of results on model checking of CSP
%processes, where the failures semantics of a finite-state CSP process is
%represented as a finite normalised transition graph, whose edges are labelled
%by the events the process engages in, and whose nodes are labelled by minimal
%acceptances or, alternatively, maximal
%refusals~\cite{Roscoe:1994:chapter}. The maximal refusals express the
%degree of nondeterminism present in a process state that is in
%one-one-correspondence to a node of the normalised transition graph. Inspired
%by the way fault-domains are specified for finite state machines (FSMs),
%we define them here as the set of CSP processes whose normalised transition
%graphs do not exceed the order (that is, the number of nodes) of the
%reference model's graph by more than a given constant.
%
%The main contribution of this paper is the construction of two test suites to
%verify the conformance relations \emph{trace refinement} and \emph{failures
%refinement} for any given non-terminating, non-divergent, finite-state,
%finite alphabet CSP process. We prove their completeness with respect to
%fault domains of the described type.  The existence of -- possibly infinite
%-- complete test suites has been established for process algebras, and for
%CSP in particular, by several
%authors~\cite{Hennessy:1988:ATP:50497,Schneider:1995:OST:203471.203475,DBLP:conf/fm/PeleskaS96,peleska1997a,DBLP:conf/icfem/CavalcantiG07,DBLP:conf/pts/CavalcantiS17}.
%To the best of our knowledge, this article is the first to present {\it
%finite}, complete test suites associated with this class of fault domains and
%conformance relations.
%
%Our result is not a simple transcription of existing knowledge about finite,
%complete test suites for FSMs. The capabilities of CSP to express
%nondeterminism in a more fine-grained way than it is possible for FSMs
%requires a more complex approach to testing for conformity in the failures
%model than required for model-based testing against nondeterministic FSMs, as
%published, for example,
%in~\cite{hierons_testing_2004,DBLP:conf/hase/PetrenkoY14}. CSP distinguishes
%between external choice, where the environment can control the behaviour of a
%process, and internal choice where the behaviour is decided internally and
%cannot be influenced by the environment. In contrast, FSMs specify
%nondeterministic behaviour by offering more than one possible output for a
%given input, and this output can be controlled by an FSM representing the
%environment. Loosely speaking, this corresponds to external choice in CSP,
%while there is no equivalent to internal choice in nondeterministic FSMs.
%
%Finally, we prove two optimality results here. (1)~We show that our approach
%to testing the admissibility of an SUT's nondeterministic behaviour is
%optimal in the sense that it cannot be established with fewer ``probes''
%investigating the SUT's degree of nondeterminism. These probes are minimal
%sets of events offered by the tests to the SUT, such that a refusal of the
%SUT to engage into at least one of these events reveals a violation of
%failures refinement. (2)~Furthermore, it is shown that the maximal length of
%traces to be investigated in a complete test suite cannot be further reduced
%without loosing the suite's capability to uncover any violation of trace or
%failures refinement.

% ==================================================================================

\paragraph{Overview} In Section~\ref{section:preliminaries}, we present the
background relevant to our work. In Section~\ref{sec:finitecompletefails},
finite complete test suites for verifying failures refinement are presented.
A sample test suite is presented in Section~\ref{sec:case}. Test suites
checking traces refinement are a simplified version of the former class; they
are presented in Section~\ref{sec:finitecomplete}. The optimality results are
presented in Section~\ref{sec:complexity}, together with further complexity
considerations. Our results are discussed in Section~\ref{sec:conc}, where we
also conclude. References to further related work are given throughout the
paper where appropriate.

% ==================================================================================

% =========================================================================
\section{Preliminaries}
\label{section:preliminaries}
% =========================================================================

We present CSP~(Section~\ref{section:csp}) and the concept of minimal hitting
sets~(Section~\ref{sec:hit}), which is central to our notion of test
for failures refinement. To study complexity, we also introduce the concept
of Sperner families~(Section~\ref{sec:sperner}).

% =========================================================================
\subsection{CSP, Refinement, and Normalised Transition Graphs}
\label{section:csp}
% =========================================================================

\paragraph{Communicating Sequential Processes (CSP)} This is a process algebra supporting
system development by refinement. Using CSP, we model both systems and their
components using processes. They are characterised by their patterns of
interactions, modelled by synchronous, instantaneous, and atomic events.

Throughout this paper, the alphabet of the processes, that is, the set of
events in scope, is denoted by $\Sigma$ and supposed to be finite.
The FDR tool~\cite{fdr} supports model checking and semantic analyses of
finite-state CSP processes.

A prefixing operator $e \then P$ defines a process that is ready to engage in
the event $e$, pending agreement of its environment to synchronise. After $e$
occurs, the process behaves as defined by $P$. The environment can be other
processes, in parallel, or the environment of a system as a whole.

Two forms of choice support branching behaviour. An external choice $P
\extchoice Q$ between processes $P$ and $Q$ offers to the environment the
initial events of $P$ and $Q$. Once a synchronisation takes place, the
process that has offered the event that has occurred is chosen and the other
is discarded. In an internal choice $P \intchoice Q$, the environment does
not have an opportunity to interfere:~the choice is made by the process.
%
\begin{example}\label{example:CSP}
  We consider the processes $P$, $Q$, and $R$ defined below. $P$ is initially
  ready to engage in the event $a$, and then makes an internal choice to
  behave like either $Q$ or $R$.
  \begin{eqnarray*}
  P & = & a \then (Q\intchoice R)
  \\
  Q & = & a \then P \extchoice c \then P
  \\
  R & = & b \then P \extchoice c \then R
  \end{eqnarray*}
  $Q$, for instance, offers to the environment the choice to engage in $a$
  again or $c$. In both cases, afterwards, we have a recursion back to $P$.
  In $R$, if $b$ is chosen, we also have a recursion back to $P$. If $c$ is
  chosen, the recursion is to $R$.
  \xbox
\end{example}
%
Iterated forms $\Extchoice i: I @ P(i)$ and $\Intchoice i: I @ P(i)$ of the
external and internal choice operators define a choice over a collection of
processes $P(i)$. If the index set $I$ is empty, the external choice is the
process $\Stop$, which deadlocks:~does not engage into any event or
terminate. For an external choice $\Extchoice e: A @ e \then P(e), A\subseteq
\Sigma$ over a set $A$ of events, we use the abbreviation $e:A \then P(e)$.
An iterated internal choice is not defined for an empty index set.

There are several parallelism operators. A widely used form of parallelism $P
\parallel[cs] Q$ defines a process in which the behaviour is characterised by
those of $P$ and $Q$ in parallel, synchronising on the events in the set
$cs$. Other forms of parallelism available in CSP can be defined using this
parallelism operator.

Interactions that are not supposed to be visible to the environment can be
hidden. The operator $P \hide H$ defines a process that behaves as $P$, with
the interactions modelled by events in the set $H$ hidden. Frequently, hiding
is used in conjunction with parallelism:~it is often desirable to make
actions of each process in a network of parallel processes, perhaps used for
coordination of the network, invisible, while events happening at their
interfaces remain observable.

A rich collection of process operators allows us to define networks of
parallel processes in a concise and elegant way, and reason about safety,
liveness, and divergences.  A comprehensive account of the notation is given
in~\cite{Roscoe2010}.

A distinctive feature of CSP is its treatment of refinement~(as opposed to
bisimulation), which is convenient for reasoning about program correctness,
due to its treatment of nondeterminism and divergence.  A variety of semantic
models capture different notions of refinement. The simplest model
characterises a process by its possible \emph{traces}; the set $\trc(P)$
denotes the sequences of (non-hidden) events in which $P$ can engage.  We say
that a process \emph{$P$ is trace-refined by another process $Q$}, written $P
\lessdet_T Q$, if $\trc(Q)\subseteq \trc(P)$.

In fact, in every semantic model, subset containment is used to define refinement. The
model we focus on first is the failures model, which captures both
sequences of interactions and deadlock behaviour. A \emph{failure} of a process $P$
is a pair $(s,X)$ containing a trace $s$ of $P$ and a \emph{refusal}:~a set $X$ of
events in which $P$ may refuse to engage, after having performed the events of
$s$. The failures model of a process $P$ records all its failures in a set
$\fails(P)$.

Semantic definitions specify, for each operator, how the traces or failures
of the resulting process can be calculated from those of each operand. For
example, for internal choice, $\fails(P\intchoice Q) =
\fails(P)\cup\fails(Q)$; see~\cite[p.~210]{Roscoe:1997:TPC:550448} for a
comprehensive list of these definitions covering $\trc(P)$ and $\fails(P)$.

Using the notation $P/s$ to denote the behaviour ot the process $P$ after
having engaged into the events in the trace $s$, the set $\refs(P/s) \defs
\{~X | (s,X) \in \fails(P)~\}$ contains the  refusals of $P$ after $s$.
Refusals are subset-closed~\cite{Hoare:1985:CSP:3921,Roscoe2010}: if $(s,X)$
is a failure of $P$ and $Y\subseteq X$, then $(s,Y)\in\fails(P)$ and
$Y\in\refs(P/s)$ follows.

For divergence-free processes, failures refinement, $P \lessdet_F Q$, is
defined by $\fails(Q)\subseteq\fails(P)$. Since refusals are subset-closed,
$P \lessdet_F Q$ implies $(s,\varnothing)\in\fails(P)$ for all traces
$s\in\trc(Q)$. So, for divergence-free processes, failures refinement implies
traces refinement. Therefore, using the conformance relation $conf$ below
%
\begin{equation}\label{eq:conf}
  Q\ conf\ P \defs \forall s\in \trc(P) \cap \trc(Q): \refs(Q/s)
  \subseteq \refs(P/s),
\end{equation}
%
failures refinement can be expressed by $\lessdet_T$ and $conf$ as proven
in~\cite{DBLP:conf/icfem/CavalcantiG07}.
%
\begin{equation}%\label{thm:fref-tref-conf}
\label{eq:failconf}
(P \lessdet_F Q) \iff (P \lessdet_T Q \land Q\ conf\ P)
\end{equation}
%
For finite processes, since refusals are subset-closed, $\refs(P/s)$ can be
constructed from the set of \emph{maximal refusals}.
%
\begin{equation}
\maxrefs(P/s) = \{ R \in\refs(P/s)~|~\forall R'\in \refs(P/s) - \{ R\}: R \not\subseteq R'\}
\end{equation}
%
Conversely, with the maximal refusals $\maxrefs(P/s)$ at hand, we can
reconstruct the refusals in the set $\refs(P/s)$. % as described by
%
\begin{equation}\label{eq:refMaxRef}
\refs(P/s) = \{ R'\in 2^\Sigma~|~\exists R\in \maxrefs(P/s): R'\subseteq R \}.
\end{equation}
%
Deterministic process states $P/s$ have exactly the one maximal refusal
defined by $\Sigma-[P/s]^0$, where $[P/s]^0$ denotes the \emph{initials} of
$P/s$, that is, the events that $P/s$ may engage into. The maximal refusals
in combination with the initials of a process express its degree of
nondeterminism as illustrated by the next example.
%
\begin{example}
\label{ex:nondetdegree} $P = (\Stop \intchoice Q)$ has maximal refusals
$\maxrefs(P) = \{ \Sigma \}$, because $\Stop$ refuses to engage in any event,
and this is carried over to $P$ by the internal choice. However, $P$ is
distinguished from $\Stop$ by its initials, which are defined by $[P]^0 = [
\Stop \intchoice Q]^0 = [Q]^0$. So $P$ may engage nondeterministically in any
initial event of $Q$, but also refuse everything, due to internal selection
of $\Stop$. Assuming an alphabet $\Sigma = \{a,b,c,d\}$, the process
%
$$Q = (e:\{a,b\} \then \Stop) \intchoice (e:\{c,d\} \then \Stop)$$
%
has maximal refusals $\maxrefs(Q) = \{ \{c,d\}, \{a,b\} \}$ and initials
$[Q]^0=\Sigma$. In contrast to $P$, nondeterminism is reflected here by
two maximal refusals. \xbox
\end{example}

% =========================================================================
\paragraph{Normalised Transition Graphs for CSP Processes} \label{sec:ntg}

As shown in~\cite{Roscoe:1994:chapter}, the failures semantics of any
finite-state CSP process $P$ can be represented by a \emph{normalised
transition graph} $G(P)$ defined by a tuple
$$
G(P) = ( N, \ii n, \Sigma, t : N\times\Sigma \pfun N, r : N \fun \mathbb{P}\mathbb{P}(\Sigma)),
$$
with nodes $N$, initial node $\ii n\in N$, and process alphabet $\Sigma$. The
partial \emph{transition function} $t$ maps a node $n$ and an event
$e\in\Sigma$ to its successor node $t(n,e)$. If $(n,e)$ is in the domain of
$t$, then there is a transition, that is, an outgoing edge, from $n$ with
label $e$, leading to node $t(n,e)$. Normalisation of $G(P)$ is reflected by
the fact that $t$ is a function.

The graph construction in~\cite{Roscoe:1994:chapter} implies that all nodes
$n$ in $N$ are reachable by sequences of edges labelled by $e_1\dots e_k$ and
connecting states $\ii n,n_1,\dots,n_{k-1},n$, such that
\[
n_1 = t(\ii n,e_1), \quad n_i = t(n_{i-1},e_i),\ i = 2,\dots,k-1,\quad
n= t(n_{k-1},e_k).
\]
%
By construction, $s\in\Sigma^*$ is a trace of $P$, if, and only if, there is
a path through $G(P)$ starting  at $\ii n$ whose edge labels coincide with
the events in $s$ in the order they appear. In analogy to $\trc(P)$, we use
the notation $\trc(G(P))$ for the set of finite, initialised paths through
$G(P)$, each path represented by its finite sequence of edge labels. We note
that $\trc(P) = \trc(G(P))$. Since $G(P)$ is normalised, there is a unique
node reached by following the events from $s$ one by one, starting in $\ii
n$. Therefore, $G(P)/s$  is also well defined.

By $[n]^0$ we denote the \emph{initials} of $n$:~the set of events occurring
as labels in any outgoing transitions.
$$
[n]^0 = \{ e\in\Sigma~|~(n,e)\in\dom~t \}
$$
The graph construction from~\cite{Roscoe:1994:chapter} guarantees that $[G(P)/s]^0 = [P/s]^0$ for all
traces $s$ of $P$.

The total function $r$ maps each node $n$ to a non-empty set of  (possibly
empty) subsets of $\Sigma$. The graph construction guarantees that
$r(G(P)/s)$ represents the maximal refusals of $P/s$ for all $s\in\trc(P)$.
As a consequence,
\begin{equation}\label{eq:refsAndR}
(s,X)\in\fails(P) \iff s\in\trc(G(P)) \wedge \exists R\in r(G(P)/s): X\subseteq R,
\end{equation}
so $G(P)$ allows us to re-construct the failures semantics of $P$.
%\fixme{alcc: Why not leave the commented out material in the report?}
% Commented out just to shorten the report.
%To see that this approach works only for finite CSP processes, we consider
%the example where $\Sigma$ is infinite. In this case,
%$\maxrefs(\Stop/\langle\rangle)$ is empty, and so we cannot use this set to
%calculate the refusals of $\Stop$, that is, $\refs(\Stop/\langle\rangle)$ as
%defined above. As with refusals, we also use the transition graph-oriented
%notation $\maxrefs(n) \subseteq r(n)$ to denote the maximal refusals
%associated with graph state $n$: if $n$ is the state reached in the
%transition graph by following the edge labels in trace $s$, then $\maxrefs(n)
%= \maxrefs(P/s)$.

% Commented out just to shorten the report.
%Well-formed normalised transition graphs must not refuse an event of the
%initials of a state in {\it every} refusal applicable in this state; more
%formally,
%%
%\begin{equation}
%\label{eq:wellformedg}
%\forall n\in N, e\in\Sigma: (n,e)\in\dom~t \Rightarrow
%\exists R\in r(n): e\not\in R
%\end{equation}
%%
%By construction, normalised transition graphs reflect the \emph{failures
%semantics} of finite-state CSP processes:~the traces $s$ of a process are
%defined by the sequences of transitions associated with paths through its
%graph, starting at $\ii n$. The pairs $(s,R)$ with $s\in\trc(P)$ and $R\in
%r(G(P)/s)$ represent the failures of $P$.

% =========================================================================
\paragraph{Acceptances} \label{sec:accs}

When investigating tests for failures refinement, the notion of
\emph{acceptances}, which is dual to refusals, is useful. While the
original introduction of acceptances presented
in~\cite[pp.~75]{Hennessy:1988:ATP:50497} was independent of refusals, we use
the definition from~\cite[pp.~278]{Roscoe:1997:TPC:550448}.   A \emph{minimal
acceptance} of a CSP process state $P/s$ is the complement of a maximal
refusal of the same state. The set of minimal acceptances of $P/s$ is denoted
by $\minaccs(P/s)$ and formally defined as
%
\begin{equation}\label{eq:accref}
\minaccs(P/s) = \{ \Sigma - R~|~R\in \maxrefs(P/s)  \}
\end{equation}
%
With this definition, a (not necessarily minimal) \emph{acceptance} of $P/s$ is a superset of some minimal acceptance and a subset of the initials $[P/s]^0$. Denoting the acceptances of $P/s$ by $\accs(P/s)$, this leads to the formal definition
\begin{equation}
\label{eq:accrefall}
\accs(P/s)  = \{ B\subseteq [P/s]^0~|~\exists A\in\minaccs(P/s):  A \subseteq B   \}
\end{equation}
%
Acceptances have the following intuitive interpretation.  If the behaviour of
$P/s$ is deterministic, its only acceptance equals $[P/s]^0$, because $P/s$
never refuses any of the events in this set. If $P/s$ is nondeterministic, it
internally chooses one of its \emph{minimal acceptance sets} $A$ and never
refuses any event in $A$, while {\it possibly} refusing the events from the
set $[P/s]^0 - A$ and {\it always} refusing those in the set $\Sigma -
[P/s]^0$.

Exploiting (\ref{eq:accref}), the nodes of a normalised transition graph can
alternatively be labelled with minimal acceptances; this captured the same
information conveyed by maximal refusals. Since process states $P/s$ are
equivalently expressed by states $G(P)/s$ of $P$'s normalised transition
graph, we also write $\minaccs(G(P)/s)$ and note that (\ref{eq:refsAndR}) and
(\ref{eq:accref}) imply
\begin{equation}\label{eq:maxrefsminaccs}
\minaccs(G(P)/s) = \{ \Sigma - R | R\in r(G(P)/s) \} = \minaccs(P/s).
\end{equation}
%
Given any non-diverging, non-terminating, finite-state process $P$, it can be
re-con\-struc\-ted from its graph $G(P)$ with initial state $\ii
n$ and transition function $t$, using $P$'s normalised syntactic
representation~\cite[pp.~277]{Roscoe:1997:TPC:550448} specified as follows.
\begin{eqnarray*}
\text{normalised}(P) & = & P_N(\ii n)
\\
P_N(n) & = & \Intchoice_{A\in\minaccs(n)\cup \{ [n]^0 \} } e:A\then P_N(t(n,e))
\end{eqnarray*}
With this definition, it is established that $P$ is semantically equivalent
to $\text{normalised}(P)$ in the failures semantics.

% .....................................................................................
 \begin{figure}[t]
   %%\hspace*{-40mm}
   \begin{center}
     \includegraphics[trim=0cm 1.4cm 0cm 1.4cm,clip,scale=.6,%width=\textwidth
                              ]{p.pdf}
   \end{center}
   %%\vspace*{-10mm}
   \caption{Normalised transition graph of CSP process $P$ from Example~\ref{ex:a}.}
   \label{fig:tga}
 \end{figure}
% .......................................................................................

\begin{example}\label{ex:a}
We consider the process $P$ in Example~\ref{example:CSP}; its transition
graph $G(P)$ is shown in Fig.~\ref{fig:tga}. The process state
$P/\varepsilon$ (where $\varepsilon$ denotes the empty trace) is represented
as node 0, with $\{ a\}$ as the only minimal acceptance, since $a$ is never
refused and no other events are accepted. Having engaged in $a$, the
transition from node 0 leads to node 1 representing the process state $P/a
= Q\intchoice R$. The internal choice induces several minimal acceptances
derived from $Q$ and $R$. Since these processes accept their initial events
in external choice, $Q\intchoice R$ induces minimal acceptance sets $\{a,c\}$
and $\{b,c\}$. We note that the event $c$ can never be refused, since it is
contained in each minimal acceptance set.

Having engaged in $c$, the next process state is represented by node 2. Due
to normalisation, there is only a single transition satisfying $t(1,c) = 2$.
This transition, however, can have been caused by either $Q$ or $R$ engaging
into $c$, so node 2 corresponds to process state $Q/c \intchoice R/c = P
\intchoice R$. This is reflected by the two minimal acceptance sets labelling
node 2. From node~2, event $c$ leads to node~3. Since $P$ does not engage
into $c$, the $R$-component of $P\intchoice R$ must have processed $c$, so
node~3 corresponds to $R/c = R$, and so it is labelled by $R$'s minimal
acceptance $\{b,c\}$. \xbox
\end{example}
%
\noindent%
Summarising, refinement between finite-state CSP processes $P, Q$ can be
expressed using their normalised graphs
\begin{eqnarray*}
G(P) & = & ( N_P, {\ii n}_P, \Sigma, t_P : N_P\times\Sigma \pfun N_P, r_P : N_P \fun \mathbb{P}\mathbb{P}(\Sigma))
\\
G(Q) & = & ( N_Q, {\ii n}_Q, \Sigma, t_Q : N_Q\times\Sigma \pfun N_Q, r_Q : N_Q \fun \mathbb{P}\mathbb{P}(\Sigma))
\end{eqnarray*}
%
as established by the results  in the following lemma. There, result
(\ref{eq:trcrefa}) reflects traces refinement in terms of graph traces;
result (\ref{eq:failconf}) expresses failures refinement in terms of traces
refinement and $conf$; result (\ref{eq:failrefa}) states how $conf$ can be
expressed by means of the maximal refusal functions of the graphs involved;
and result (\ref{eq:failrefb}) states the same in terms of the minimal
acceptances that can be derived from the maximal refusal functions by means
of (\ref{eq:maxrefsminaccs}).
%
\begin{lemma}
  \label{lemma:tgtrcref}
  \begin{eqnarray}
  P \lessdet_T Q & \Leftrightarrow & \trc(G(Q)) \subseteq\trc(G(P))
  \label{eq:trcrefa}
%  \\
%  \label{eq:failconf}
%  P \lessdet_F Q & \Leftrightarrow & P \lessdet_T Q \wedge Q\ conf\ P
  \\
  \label{eq:failrefa}
  Q\ conf\ P & \Leftrightarrow & \nonumber
  \forall s\in\trc(G(Q))\cap \trc(G(P)), R_Q\in r_Q(G(Q)/s):  \nonumber
  \\ & & \tabd
  \exists R_P\in r_P(G(P)/s): R_Q\subseteq R_P
  \\
  \label{eq:failrefb}
   & \Leftrightarrow &
    \forall s\in\trc(G(Q))\cap \trc(G(P)), A_Q\in\minaccs(G(Q)/s):  \nonumber
   \\ & & \tabd
  \exists A_P\in\minaccs(G(P)/s): A_P\subseteq A_Q
 \end{eqnarray}
  \xbox
\end{lemma}
%
\begin{proof}
To prove (\ref{eq:trcrefa}), we recall that $P\lessdet_T Q$ is defined as
$\trc(Q)\subseteq \trc(P)$ and, moreover, $\trc(P)=\trc(G(P))$ and
$\trc(Q)=\trc(G(Q))$. %Result (\ref{eq:failconf}) has been shown
%in~\cite{DBLP:conf/icfem/CavalcantiG07}.
%
%\medskip
%\noindent
%To prove (\ref{eq:failconf}), we derive
%\begin{argue}
%P \lessdet_F Q
%\\
%\Leftrightarrow \fails(Q)\subseteq \fails(Q) & definition of $\lessdet_F$
%\\
%\Leftrightarrow \forall (s,R) \in  \fails(Q): (s,R) \in\fails(P)
%& property of $\subseteq$
%\\
%\Leftrightarrow \forall (s,\varnothing) \in  \fails(Q): (s,\varnothing) \in\fails(P)
%\wedge {}
%\\\tabc \forall s\in\trc(Q), R\in \refs(Q/s): R\in\refs(P/s)
%\\
%& Definition of failures, refusals, subset closure
%\\
%\Leftrightarrow \trc(Q)\subseteq\trc(P)
%\wedge {}
%\\\tabc \forall s\in\trc(Q), R\in \refs(Q/s): R\in\refs(P/s)
%\\
%& Definition of failures and traces
%\\
%\Leftrightarrow \trc(Q)\subseteq\trc(P)
%\wedge {}
%\\\tabc \forall s\in\trc(Q)\cap\trc(P): \refs(Q/s) \subseteq\refs(P/s)
%\\
%& Property of $\cap$ and $\subseteq$
%\\
%\Leftrightarrow
%P\lessdet_T Q \wedge Q\ conf\ P
%& Definition of $\lessdet_T$ and $conf$
%\end{argue}
%
To prove (\ref{eq:failrefa}), we derive
\begin{argue}
  Q\ conf\ P
  \\
  \Leftrightarrow \forall s\in\trc(P)\cap \trc(Q): \refs(Q/s)\subseteq\refs(P/s)
  %\\
  & Definition of $conf$  (\ref{eq:conf})
  \\
  \Leftrightarrow \forall s\in\trc(G(P))\cap \trc(G(Q)): \refs(Q/s)\subseteq\refs(P/s)
  \\
  & $\trc(P) = \trc(G(P))$, $\trc(Q) = \trc(G(Q))$
  \\
  \Leftrightarrow \forall s\in\trc(G(P))\cap \trc(G(Q)), R_Q\in r_Q(G(Q)/s):
  \exists R_P\in r_P(G(P)/s): R_Q\subseteq R_P
  \\ %
  & Property of $r_P, r_Q$~(subset closure) and (\ref{eq:refMaxRef})
\end{argue}

\medskip
\noindent
Finally, (\ref{eq:failrefb}) follows from (\ref{eq:failrefa}) using (\ref{eq:accref})
and the fact that $R_Q \subseteq R_P$ is equivalent to
$\Sigma - R_P \subseteq \Sigma - R_Q$.
\xbox
\end{proof}

% =========================================================================
\paragraph{Reachability Under Sets of Traces} \label{sec:V} Given a finite-state CSP
process $P$ and its normalised transition graph $G(P)$,
%\[
%G(P) = ( N, \ii n, \Sigma, t : N\times\Sigma \pfun N, r : N \fun \mathbb{P}\mathbb{P}(\Sigma)),
%\]
we suppose that $V\subseteq\Sigma^*$ is a prefix-closed set  of sequences of
events. By $t(\ii n,V)$ we denote the set
\[
t(\ii n,V) = \{ n\in N~|~\exists s\in V: s\in\trc(P)\wedge G(P)/s = n \}
\]
of nodes in $N$ that are reachable in $G(P)$ by applying traces of $V$. The
lemma below specifies a construction method for such sets $V$ reaching {\it
every} node of $N$.

\begin{lemma}
\label{lemma:extendV} Let $P$ be a CSP process with normalised transition
graph $G(P)$. Let $V\subseteq\Sigma^*$ be a finite prefix-closed set of
sequences of events. Suppose that  $G(P)$ reaches $k < |N|$ nodes under $V$,
that is, $|t(\ii n,V)| = k$. Let $V.\Sigma$ denote the set of all sequences
from $V$, extended by any event of $\Sigma$.  Then $G(P)$ reaches at least
$(k+1)$ nodes under $V\cup V.\Sigma$.
\end{lemma}
\begin{proof}
Suppose that $n'\in (N - t(\ii n,V))$.  Since all nodes in $N$ are reachable,
there exists a trace $s$ such that $G(P)/s = n'$. Decompose $s = s_1.e.s_2$
with $s_i\in\Sigma^*, e\in\Sigma$, such that $G(P)/s_1 \in t(\ii n,V)$ and
$G(P)/s_1.e \not\in t(\ii n,V)$. Such a decomposition always exists, because
$V$ is prefix-closed and therefore contains the empty trace $\varepsilon$.
Note, however, that it is not necessarily the case that $s_1\in V$. Since
$G(P)$ reaches $G(P)/s_1$ under $V$, there exists a trace $u\in V$ such that
$G(P)/u = G(P)/s_1 = \ol n$. Since $s = s_1.e.s_2$ is a trace of $P$ and
$G(P)/s_1 = \ol n$, then $(\ol n,e)$ is in the domain of $t$. So, $ G(P)/u.e
= G(P)/s_1.e = n$ is a well-defined node of $N$ not contained in $t(\ii
n,V)$. Since $u.e\in V\cup V.\Sigma$, $G(P)$ reaches at least the additional
node $n$ under $V\cup V.\Sigma$. This completes the proof. \xbox
\end{proof}

% =========================================================================
\paragraph{Graph Products} \label{sec:GP} For proving our main theorems, it is
necessary to consider the \emph{product} of normalised transition graphs. We
need this only for the investigation of corresponding traces in reference
processes and processes for SUTs. So, the labelling of nodes with maximal
refusals or minimal acceptances are disregarded in the product construction.
We consider two normalised transition graphs
\[
G_i = ( N_i, \ii n_i, \Sigma, t_i : N_i\times\Sigma \pfun N_i, r_i : N_i \fun \mathbb{P}\mathbb{P}(\Sigma)),\qquad i = 1,2,
\]
over the same alphabet $\Sigma$. Their product is defined by
%
\begin{eqnarray}
G_1\times G_2 & = & (N_1\times N_2,(\ii n_1,\ii n_2), t:(N_1\times N_2)\times\Sigma\pfun (N_1\times N_2))
\\
\dom~t & = & \{ ((n_1,n_2),e)\in (N_1\times N_2)\times\Sigma | (n_1,e)\in\dom~t_1\wedge (n_2,e) \in\dom~t_2 \}
\\
t((n_1,n_2),e) & = & (t_1(n_1,e),t_2(n_2,e))\ \text{for $((n_1,n_2),e)\in\dom~t$}
\end{eqnarray}
%
The following lemma is used in the proof of our main theorem.
%
\begin{lemma}\label{lemma:reachproduc}
If $G_1$ has $p$ states and $G_2$ has $q$ states, then every reachable state
$(n_1,n_2)$ of the product graph $G_1\times G_2$ can be reached by a trace
%%%$s\in\trc(G_1)\cap \trc(G_2)$  we don't need this
of maximal length $(pq-1)$.
\end{lemma}
\begin{proof}
The product graph $G_1\times G_2$ has at most $pq$ states. The empty trace $\varepsilon$
reaches its initial state $(\ii n_1,\ii n_2)$. Applying Lemma~\ref{lemma:extendV}
$(pq-1)$ times with $V=\{\varepsilon \}$ implies that $G_1\times G_2$ reaches
all of its reachable states (there are at most $pq$ of them) under
$V' = V \cup V.\Sigma\cup\dots \cup V.\Sigma^{(pq-1)}$. The maximal length of traces in
$V'$ is $(pq-1)$.
\xbox
\end{proof}
%
This concludes our presentation of CSP and of results regarding its
semantics that are used in the next section.

% =========================================================================
\subsection{Minimal Hitting Sets}
\label{sec:hit}
% =========================================================================

\paragraph{Definition} The main idea of the underlying test strategy for failures
refinement is based on solving a \emph{hitting set problem}. Given a finite
collection of finite sets ${\cal C} = \{ A_1,\dots,A_n\}$, such that each
$A_i$ is a subset of a universe $\Sigma$, a \emph{hitting set}
$H\subseteq\Sigma$ is a set satisfying the following property.
%
\begin{equation}
  \label{eq:hit}
  \forall A\in {\cal C}: H\cap A \neq\varnothing.
\end{equation}
%
A \emph{minimal hitting set} is a hitting set that cannot be further reduced
without losing the characteristic property (\ref{eq:hit}). By $\minhits({\cal
C})$ we denote the collection of minimal hitting sets for a collection ${\cal
C}$. For the pathological case where $C$ contains an empty set,
$\minhits({\cal C})$ is also empty. The problem of determining minimal hitting sets is %%% \cite{Book1975-BOOKRM,5533149}
NP-hard~\cite{5533149}. We see below, however, that using minimal hitting
sets, we can reduce the effort of testing for failures refinement from a
factor of $2^{|\Sigma|}$ to a factor that equals the number of minimal
hitting sets.

% -----------------------------------------------------------------------------
\paragraph{Minimal Hitting Sets of Minimal Acceptances} In this article, we are
interested in the minimal hitting sets of minimal acceptances; for these, the
abbreviated notation
$%\[ \
\minhits(P/s) = \minhits(\minaccs(P/s))
$ %\]
is used. The minimal hitting sets
of minimal acceptances may be alternatively characterised by means of the
failures of a process as is done in~\cite{DBLP:conf/icfem/CavalcantiG07}. To
this end, in~\cite{DBLP:conf/icfem/CavalcantiG07}, the authors define, for
any collection ${\cal C}\subseteq 2^\Sigma$ of subsets from $\Sigma$
\begin{equation}\label{eq:minsub}
\minsub({\cal C}) = \{ A\in {\cal C}~|~\forall B\in {\cal C}:
B\subseteq A\Rightarrow B = A \}.
\end{equation}
The collection $\minsub({\cal C})$ contains all those sets of
${\cal C}$ that are not true subsets of other members of ${\cal C}$. With
this definition, the relation between failures and minimal hitting sets of
minimal acceptances is established in the following lemma.
%
\begin{lemma}
\label{lemma:failureshittingsets}
For any trace $s$ of CSP process $P$, define
$%\[
{\cal A}_s = \{ A\subseteq \Sigma~|~(s,A)\not\in\fails(P)  \}.
$ %\]
Then
$ %\[
\minhits(P/s) = \minsub {\cal A}_s
$ %\]
for all traces $s$ of $P$.
\end{lemma}
\begin{proof}
We derive
\begin{argue}
(s,A)\not\in \fails(P)
\\
\Leftrightarrow A\not\in Ref(P/s)
\\
\Leftrightarrow  \forall R\in maxRef(P/s): A\not\subseteq R
\\
\Leftrightarrow   \forall B\in minAcc(P/s): A\not\subseteq (\Sigma\setminus
B)& this follows from (\ref{eq:accref})
\\
\Leftrightarrow  \forall B\in minAcc(P/s): A\cap B\neq \varnothing
\\
\Leftrightarrow  A \text{ is a (not necessarily minimal)}
              \text{ hitting set of } minAcc(P/s)
\end{argue}
%
Specialising this derivation valid for arbitrary $A$ satisfying $(s,A)\not\in {\text{failures}}(P)$ to
minimal $A$ with this property proves the statement of the lemma.
\xbox
\end{proof}

% -------------------------------------------------------------------------
\paragraph{Minimal Hitting Sets of Normalised Transition Graphs} Since, as
previously explained, minimal acceptances can be used to label the nodes of a
normalised transition graph, and since $\minaccs(P/s) = \minaccs(G(P)/s)$ by
(\ref{eq:maxrefsminaccs}), the notation of minimal hitting sets also carries
over to graphs: we write $\minhits(n)$ for nodes $n$ of $G(P)$ and observe
that
\begin{equation}
\label{eq:minhitsGP}
\minhits(G(P)/s) = \minhits(P/s) \ \text{for all}\ s\in \trc(P).
\end{equation}

% ------------------------------------------------------------------------
\paragraph{Characterisation of $conf$ by Minimal Hitting Sets} The following lemma
establishes that the $conf$ relation specified in (\ref{eq:conf}) can be
characterised by means of minimal acceptances and their minimal hitting sets.
%
\begin{lemma}
\label{lemma:hseta}
Let $P, Q$ be two finite-state CSP processes.
%% satisfying $P\lessdet_T Q$.
For each $s\in\trc(P)$,
let $\minhits(P/s)$ denote the
collection of all minimal hitting sets of $\minaccs(P/s)$.
Then the following statements are equivalent.
\begin{enumerate}
\item $Q\ conf\ P$

\item For all $s\in\trc(P)\cap \trc(Q)$ and $H \in  \minhits(P/s)$, $H$ is
a (not necessariliy minimal) hitting set of $\minaccs(Q/s)$.
\end{enumerate}
\end{lemma}
\begin{proof}
We apply (\ref{eq:maxrefsminaccs}) and (\ref{eq:minhitsGP}), so that
$\minaccs(P/s)$ and $\minaccs(G(P)/s)$, as well as $\minhits(P/s)$ and
$\minhits(G(P)/s)$ are used interchangeably. For showing ``$(1) \Rightarrow
(2)$'', we assume $Q\ conf\ P$ and $s\in\trc(P)\cap \trc(Q)$.
Lemma~\ref{lemma:tgtrcref} (\ref{eq:failrefb}), states that
$%\[
\forall A_Q\in\minaccs(G(Q)/s):
\exists A_P\in\minaccs(G(P)/s): A_P\subseteq A_Q
$. %\]
Therefore, $H \in  \minhits(P/s)$ not only implies $H\cap A_P\neq\varnothing$
for all minimal acceptances $A_P$, but also $H\cap A_Q\neq\varnothing$ for
every minimal acceptance $A_Q$, because $A_P\subseteq A_Q$ for at least one
$A_P$. So, each $H \in \minhits(P/s)$ is also a hitting set for
$\minaccs(G(Q)/s)$ as required.

To prove ``$(2) \Rightarrow (1)$'', we assume that (2) holds, but that $P\
conf\ Q$ does {\it not} hold. According to Lemma~\ref{lemma:tgtrcref},
(\ref{eq:failrefb}), there exists $s\in\trc(P)\cap \trc(Q)$ such that
\[
\exists A_Q\in\minaccs(G(Q)/s): \forall A_P\in\minaccs(G(P)/s):
A_P\not\subseteq A_Q \qquad (*)
\]
Let $A$ be such an acceptance set $A_Q$ fulfilling (*).
Define
$%\[
\overline H = \bigcup\{ A_P \setminus A~|~A_P\in\minaccs(G(P)/s) \}.
$ %\]
Since $A_P \setminus A \neq\varnothing$ for all $A_P$ because of (*),
$\overline H$ is a hitting set of $\minaccs(G(P)/s)$ which has an  empty
intersection with $A$. Minimising $\overline H$ yields   a minimal hitting
set $H\in \minhits(P/s)$ which is {\it not} a hitting set of
$\minaccs(G(Q)/s)$, a contradiction to Assumption~2.
This completes the proof of the lemma.
\xbox
\end{proof}
%
We note that $\minaccs(P) = \{ \varnothing \}$ if $P = Q\intchoice \Stop$.
Since $\Stop$ accepts nothing, its minimal acceptance is $\varnothing$, and
this carries over to $Q\intchoice \Stop$.  From (\ref{eq:failrefb}) we
conclude that $\varnothing\in\minaccs(P)$ implies $\minaccs(P) = \{
\varnothing \}$. This clarifies that $\minhits(P/s)$ is empty if, and only
if, $\minaccs(P) = \{ \varnothing \}$. The proof of Lemma~\ref{lemma:hseta}
covers the situations where $\minaccs(P/s) = \{ \varnothing \}$ and so
$\minhits(P/s) = \varnothing$. Trivially,
\begin{equation}
\label{eq:minhitminaccempty}
\minaccs(P/s) = \{ \varnothing \} \iff \minhits(P/s) = \varnothing
\end{equation}
holds.

% =========================================================================
\subsection{Sperner Families}
\label{sec:sperner}
% =========================================================================

In preparation for complexity results presented in
Section~\ref{sec:complexity}, we consider how many minimal hitting sets can
maximally exist for a collection of minimal acceptances. To this end, the
following definitions and results are useful.

A \emph{Sperner Family} is a collection ${\cal S} \subseteq 2^\Sigma$ of sets
from a given finite universe $\Sigma$ that do not contain each other, that
is, $H_1 \not\subseteq H_2 \wedge H_2 \not\subseteq H_1$ holds for each pair
$H_1\neq H_2\in {\cal S}$. Specialising antichains known from partial orders
to finite sets partially ordered by $\subseteq$ results in Sperner families.
Given an {\it arbitrary} collection of subsets ${\cal C}\subset 2^\Sigma$,
the sub-collection $\minsub({\cal C})$ defined in (\ref{eq:minsub}) is a
Sperner Family contained in ${\cal C}$.

\newpage
\noindent
We further observe that
\begin{itemize}
\item the maximal refusals of a CSP process state,
\item the minimal acceptances of a CSP process state, and
\item the minimal hitting sets of a given collection of sets
\end{itemize}
are Sperner families. Moreover, given any finite alphabet $\Sigma$ with
$\card{\Sigma} = n$, every collection $S$ of subsets with identical
cardinality $k \le n$ is a Sperner family,  because $A_1, A_2\in S \wedge A_1
\subseteq A_2\wedge \card{A_1} = \card{A_2}$ implies $A_1 = A_2$. Given  any
Sperner Family $S$ of $\Sigma$, $S$ represents the minimal acceptances in the
initial state of the CSP process
%\[
$P = \Intchoice_{A\in S}\big( e:A \then P(e) \big)$.
%\]

The cardinality of Sperner Families is determined by the following theorem.
\begin{theorem}[Sperner's Theorem~\cite{sperner_satz_1928}]
\label{th:sperner} Given a  Sperner family ${\cal S}$ over an $n$-element
universe $\Sigma$, its cardinality $\card{{\cal S}}$ is bound by
\[
\card{{\cal S}} \le \binom{n}{\lfloor\frac{n}{2}\rfloor}.
\]
The upper bound is reached if, and only if, one of the following cases apply:
\begin{enumerate}
\item For even $n$, if ${\cal S}$ consists of all subsets of $\Sigma$ with
    cardinality $n/2$;
\item For odd $n$, if one of the following cases holds;
\begin{enumerate}
\item ${\cal S}$ consists of all subsets of $\Sigma$ with cardinality $(n+1)/2$;
    or
\item ${\cal S}$ consists of all subsets of $\Sigma$ with cardinality $(n-1)/2$.
\end{enumerate}
\end{enumerate}
\xbox
\end{theorem}
It is shown in Section~\ref{sec:complexity} that this upper bound can
actually be reached by the Sperner Family containing the hitting sets
associated with the minimal acceptances of a CSP process state.

% ==========================================================================
\section{Finite Complete Test Suites for CSP Failures Refinement}
\label{sec:finitecompletefails}
% ==========================================================================

Here, we define our notion of tests for failures refinement, and then prove
completeness of our suite. Finally, we study to complexity of our approach by
identifying a bound on the number of tests we need in a complete suite.

% ==========================================================================
\subsection{Test Cases for Verifying CSP Failures Refinement}

% -------------------------------------------------------------------------
\paragraph{Test Definition and Basic Properties}
In the domain of process algebras, test cases are typically represented by
processes interacting concurrently with the
SUT~\cite{Hennessy:1988:ATP:50497}. Considering an (unknown) process that
represents the behaviour of the SUT, we say that tests synchronise with the
process for the SUT over its visible events and use some additional events
outside the SUT process's alphabet to express whether the test execution
passed or failed.
%%%, or if no verdict could be obtained.

For a given reference process $P$, its normalised transition graph
$$
G(P) = ( N, \ii n, \Sigma, t : N\times\Sigma \pfun N, r : N \fun \mathbb{P}\mathbb{P}(\Sigma)),
$$
and each integer $j\ge 0$, we define a test for failures refinement as shown
below.
%
\begin{eqnarray}
\label{eq:UFP}
U_F(j) & = & U_F(j,0,\ii n)
\\
U_F(j,k,n) & = & \big(e:(\Sigma - [n]^0)  \then \efail\then \Stop \big)
\label{eq:ufa}
\\ & & \extchoice \nonumber
\\ & & (\minhits(n) =   \varnothing  )    \&   \big( \epass \then \Stop \big)
\label{eq:ufb}
\\ & & \extchoice \nonumber
\\ & & (k < j) \& \big(e:[n]^0   \then U_F(j,k+1,t(n,e) \big)
\label{eq:ufc}
\\ & & \extchoice \nonumber
\\ & & (k = j \wedge \minhits(n)\neq\varnothing) \& \taba \big( \Intchoice_{H\in\minhits(n)} (e:H   \then \epass \then\Stop   )  \big)
\label{eq:ufd}
\end{eqnarray}

% --------------------------------------------------------------------------
\paragraph{Explanation of the Test Definition} A test \pagebreak is performed by running
$U_F(j)$ concurrently with any SUT process $Q$, synchronising over $\Sigma$.
So, a \emph{test execution} is a trace of the concurrent process
%\[
$Q\parallel[\Sigma] U_F(j)$.
%\]

It is assumed that the events $\efail$ and $\epass$, indicating verdicts FAIL
and PASS for the test execution, are not included in $\Sigma$. Since we
assume that $Q$ is free of livelocks, it is guaranteed that events $\efail$
or $\epass$ always become visible, if they are the only events $U_F(j)/s$ is
ready to engage in: if $U_F(j)/s$ can only produce $\epass$ or $\efail$, the
occurrence of these events can never be blocked due to a livelock in $Q$
occurring in the same step of the execution.


The test is \emph{passed} by the SUT (written $Q\ \pass\ U_F(j)$) if, and
only if, {\it every} execution of $Q\parallel[\Sigma] U_F(j)$ terminates with
the event $\epass$. This can also be  expressed by means of a failures
refinement as defined below.
%
\begin{equation}
\label{eq:passF}
Q\ \pass\ U_F(j) \defs (\epass\then\Stop) \lessdet_F (Q\parallel[\Sigma] U_F(j)) \hide \Sigma
\end{equation}
%
This type of pass relation is often called \emph{must test}, because every
test execution must end with the $\epass$
event~\cite{Hennessy:1988:ATP:50497}.

It is necessary to use failures refinement in the definition above, and not
just traces refinement:~$(Q\parallel[\Sigma] U_F(j)) \hide \Sigma$ may have
the same visible traces $\varepsilon$ and $\epass$ as the ``Test Passed
Process'' $(\epass\then\Stop)$. However, the former may nondeterministically
refuse $\epass$, due to a deadlock occurring when a faulty SUT process
executes concurrently with $U_F(j,k,n)$ executing branch~(\ref{eq:ufd}), when
the guard condition $(k = j\wedge \minhits(n)\neq\varnothing)$ evaluates to
$\ist$. This is explained further in the next paragraphs. Alternatively, a
faulty SUT $Q$ might internally deadlock after a trace $s$ where $\#s < j$
and $\minhits(G(P)/s) \neq \varnothing$, so that the process
$(Q\parallel[\Sigma] U_F(j))/s$ deadlocks as well.

Intuitively, $U_F(j)$ is able to perform any trace $s$ of $P$, up to a length
$j$. If, after having already run through $s$ with $\#s \le j$, the SUT
accepts an event outside the initials of $P/s$
 (recall from Lemma~\ref{lemma:ufproperties} that $[n]^0 = [P/s]^0$ for $U_F(j)/s$),
the test immediately terminates with FAIL-event $\efail$. This is handled by
the branch (\ref{eq:ufa}) of the external choice. %in the process
%$U_F(j,\#s,n)$ defined above.

If $P/s$ is the $\Stop$ process or has $\Stop$ as an internal choice, this is
revealed by $\minhits(G(P)/s) = \varnothing$ (recall
(\ref{eq:minhitminaccempty}) and Lemma~\ref{lemma:ufproperties}). In this
case, the test may terminate successfully (branch (\ref{eq:ufb}) of the
choice in $U_F(j,\#s,G(P)/s)$). If $P/s$ may also nondeterministically engage
into events, branch (\ref{eq:ufc}) is simultaneously enabled. If $Q/s$ is
able to engage into an event in $\Sigma - [P/s]^0$, a test execution exists
where $U_F(j,\#s,G(P)/s)$ branches into (\ref{eq:ufa}) and produces the
$\efail$ event.

If the length of $s$ is still less than $j$, the test accepts any event $e$
from the initials $[P/s]^0 = [G(P)/s]^0$ and continues recursively as
$U_F(j,\#s+1,G(P)/s.e)$ in branch~(\ref{eq:ufc}); this follows from
Lemma~\ref{lemma:ufproperties}~(note that $G(P)/s.e = t(G(P)/s,e)$). A test
of this type is called \emph{adaptive}, because it accepts any legal
behaviour of the SUT, here any event from $[P/s]^0$, and adapts its
consecutive behaviour to the event selected by the SUT, here
$U_F(j,\#s+1,G(P)/s.e)$.

Now suppose that a test execution has run through a trace $s\in\trc(P)$ of
length $j$, so that $U_F(j)/s = U_F(j,j,n)$ with $n = G(P)/s$. If
$\minhits(n)\neq \varnothing$, the test changes its behaviour:~instead of
offering {\it all} legal events from $[n]^0$ to the SUT, it
nondeterministically chooses a minimal hitting set $H\in \minhits(n)$ and
only offers the events contained in $H$. If the SUT refuses to engage into
some event of $H$, this reveals a violation of failures refinement:~according
to Lemma~\ref{lemma:hseta}, a conforming SUT should accept at least one event
of each minimal hitting set in $\minhits(n)$. Therefore, the test execution
terminates with  $\epass$, only if such an event is accepted. Otherwise, it
deadlocks, and the test fails.

The specification of $U_F(j,k,n)$ implies that the test always stops after
having engaged into a trace $s\in\trc(Q)$ of maximal length $j$ or $j+1$. If
branch (\ref{eq:ufa}) is the last to be entered, the maximal length of $s$ is
$j+1$, and the test execution stops with $\efail$. If branch (\ref{eq:ufb})
is the last to be entered, the maximal length of $s$ is $j$, and the
execution stops with $\epass$. If branch (\ref{eq:ufd}) is the last to be
entered, then there are two possibilities. The first is that the process
accepts another event $e$ of some minimal hitting set $H\in\minhits(n)$ with
$n = G(P)/s$ according to Lemma~\ref{lemma:ufproperties}. In this case, the
final length of $s$ is $j+1$, and the execution terminates with $\epass$.
Alternatively, the test execution $(Q\parallel[\Sigma] U_F(j))/s$ deadlocks,
the final length of $s$ is $j$, and the execution stops without a PASS or
FAIL event. Such an execution is also interpreted as FAIL, because it reveals
that $(\epass\then\Stop) \not\lessdet_F (Q\parallel[\Sigma] U_F(j)) \hide
\Sigma$.

We observe that the number of possible executions of $Q\parallel[\Sigma]
U_F(j)$ is finite, because the number of traces $s$ with maximal length
$(j+1)$ is finite and the sets $[n]^0$, $(\Sigma - [n]^0)$, and $\minhits(n)$
are finite. Moreover, we further recall that $\minhits(n)$ may be empty, in
which case the indexed internal choice in (\ref{eq:ufd}) would be undefined.
The guard in that branch, however, requires $\minhits(n)\neq\varnothing$, and
branches (\ref{eq:ufa}) or (\ref{eq:ufb}) can be taken in this situation.

The following lemma establishes relationships between $U_F(j)$ and the
reference process $P$ from which it is derived.

\begin{lemma}\label{lemma:ufproperties}
If $s\in\trc(P)$ satisfies $\#s\le j$, then
$s, s.e\in\trc(U_F(j))$ for all $e\in\Sigma$, and the following properties hold.
\begin{eqnarray}
\label{eq:ifpa}
  &  & U_F(j)/s = U_F(j,\#s,G(P)/s)
\\
\label{eq:ifpb}
e\not\in [P/s]^0 & \implies & U_F(j)/s.e = (\efail\then\Stop)
\\
\label{eq:ifpc}
U_F(j)/s = U_F(j,\# s,n)  & \implies & [n]^0 = [P/s]^0
\\
\label{eq:ifpd}
U_F(j)/s = U_F(j,\# s,n)  & \implies & \minhits(n) = \minhits(P/s)
\end{eqnarray}
\end{lemma}
\begin{proof}
We prove (\ref{eq:ifpa}) by induction over the length of $s$. For $\#s = 0$,
the statement holds because $U_F(j)$ starts with the initial node $\ii n$ of
$G(P)$. Suppose that the statement holds for all traces $s$ with length $\# s
\le k < j$, so that $U_F(j)/s = U_F(j,\#s,G(P)/s)$. Now let $s.e$ be a trace
of $P$, so that $e\in [P/s]^0$. Since $[G(P)/s]^0 = [P/s]^0$ for all traces
$s$ of $P$, we conclude that $e\in  [G(P)/s]^0$, so $U_F(j,\#s,G(P)/s)$ can
engage into $e$ by executing branch (\ref{eq:ufc}). Since $t$ is the
transition function of $G(P)$ and $e\in [G(P)/s]^0$, $t(G(P)/s,e)$ is
defined, and $t(G(P)/s,e) = G(P)/s.e$. So, the new recursion in branch
(\ref{eq:ufc}) is so that $U_F(j)/s.e = U_F(j,\#s,G(P)/s)/e =
U_F(j,\#s+1,G(P)/s.e)$ as required.

To prove (\ref{eq:ifpb}), we apply (\ref{eq:ifpa}) to conclude that $U_F(j)/s
= U_F(j,\#s,G(P)/s)$, because $s$ is a trace of $P$. Noting again that
$[G(P)/s]^0 = [P/s]^0$, this implies that $e\not\in [G(P)/s]^0$, so
$U_F(j,\#s,G(P)/s)$ can engage in $e$ by entering branch (\ref{eq:ufa}). The
specification of this branch implies that $U_F(j)/s.e = U_F(j,\#s,G(P)/s)/e =
(\efail\then\Stop)$.

Statement (\ref{eq:ifpc}) follows trivially from (\ref{eq:ifpa}), because
$[G(P)/s]^0 = [P/s]^0$ for all traces $s$ of $P$. Finally, statement
(\ref{eq:ifpd}) follows trivially from (\ref{eq:ifpa}), because, according to
(\ref{eq:minhitsGP}), $\minhits(G(P)/s) = \minhits(P/s)$ for all traces of
$P$. \xbox
\end{proof}
%
Note that it is not guaranteed for $U_F(j)$ to run through the traces $s,
s.e$ in Lemma~\ref{lemma:ufproperties}, if $\minhits(P/u) = \varnothing$ for
some prefix $u$ of $s$: in such a case, $U_F(j)$ may stop with a $\epass$
event by entering branch (\ref{eq:ufb}). Therefore,
Lemma~\ref{lemma:ufproperties} just states the existence of
$U_F(j)$-executions $s, s.e$ satisfying the properties stated there.

% -------------------------------------------------------------------------
\paragraph{Complete Testing Assumption} As explained above, passing a test case
$U_F(j)$ requires that none of the possible executions $(Q\parallel[\Sigma]
U_F(j))$ stops after $\efail$ or stops without having produced the event
$\epass$. Therefore, it is necessary to determine whether all possible
executions have been covered in the repeated runs of $(Q\parallel[\Sigma]
U_F(j))$. The theoretical completeness results are, therefore, based on a
\emph{complete testing
assumption}~\cite{hierons_testing_2004,DBLP:conf/icfem/CavalcantiG07}, which
means that every possible behaviour of the SUT is performed after a finite
number of test executions. In practice, this is realised by executing each
test several times, recording the traces that have been performed, and using
hardware or software coverage analysers to determine whether all possible
behaviours of the SUT have been observed. Therefore, testing nondeterministic
SUTs comes at the price of having to apply some grey-box testing techniques.
%enabling us to decide whether all SUT behaviours have been observed.

% =========================================================================
\subsection{A Finite Complete Test Suite for Failures Refinement}
% =========================================================================

A CSP \emph{fault model} ${\cal F} = (P,\sqsubseteq,{\cal D})$ consists of a
reference process $P$, a conformance relation $\sqsubseteq \in \{\lessdet_T,
\lessdet_F\}$, and a fault domain ${\cal D}$, which is a set of CSP processes
over $P$'s alphabet that may or may not conform to $P$. A test suite $\TS$ is
called \emph{complete} with respect to fault model ${\cal F}$, if, and only
if, the following conditions are fulfilled.
\begin{description}
\item[1.~Soundness] If $P \sqsubseteq Q$, then $Q$ passes all tests in $\TS$.
\item[2.~Exhaustiveness] If $P \not\sqsubseteq Q$ and $Q\in{\cal D}$,
then $Q$ fails at least one test in $\TS$.
\end{description}
%
The following main theorem establishes the completeness of our test suite.

% -------------------------------------------------------------------------
\begin{theorem}\label{th:failurestest}
Let $P$ be a non-terminating, divergence-free CSP process over alphabet $\Sigma$ whose
normalised transition graph $G(P)$ has $p$ states. Define fault domain ${\cal
D}$ as the set of all divergence-free CSP processes over alphabet $\Sigma$,
whose transition graph has at most $q$ states with $q \ge p$. Then the test
suite
\[
\TS_F = \{ U_F(j)~|~0 \le j < pq  \}\quad\text{with $U_F(j)$ as specified in (\ref{eq:UFP})}
\]
is complete with respect to ${\cal F} = (P,\lessdet_F,{\cal D})$.
\end{theorem}
% ------------------------------------------------------------------------
%
The proof of the theorem follows directly from the two lemmas below. The
first lemma establishes that test suite $\TS_F$ is sound, and the second
establishes that the suite is also exhaustive.
%
\begin{lemma}\label{lemma:mainfsound}
A test suite $\TS_F$ generated from a CSP process $P$, as specified in
Theorem~\ref{th:failurestest}, is passed by every CSP process $Q$ satisfying
$P\lessdet_F Q$.
\end{lemma}
\begin{proof}{~}We make two points in separate steps below. The
first is that the test execution cannot reach branch~(\ref{eq:ufa}) and raise
a $fail$ event.  The second is that it cannot deadlock without raising a
$pass$ event. This case would also be interpreted as FAIL, since then
$\epass\then\Stop$ is not failures refined by $(Q\parallel[\Sigma]
U_F(j))\hide \Sigma$.

\paragraph{Step~1} Suppose that $P\lessdet_F Q$, so $P\lessdet_T Q$ and $Q\
conf\ P$ according to (\ref{eq:failconf}). Since   $\trc(Q)\subseteq
\trc(P)$, any adaptive test $U_F(j)$ running in parallel with $Q$ will always
enter the branches (\ref{eq:ufb}), (\ref{eq:ufc}), or (\ref{eq:ufd}) of the
external choice construction for $U_F(j,k,n)$. To see this, consider
 $U_F(j,k,n) = U_F(j)/s$
with $s\in\trc(Q)$. Lemma~\ref{lemma:ufproperties} implies $U_F(j,k,n) =
U_F(j,k,G(P)/s)$, so $[n]^0 = [G(P)/s]^0 = [P/s]^0$. As a consequence,
$[Q/s]^0\subseteq [P/s]^0 = [n]^0$, so branch~(\ref{eq:ufa}) can never be
entered in the parallel execution of $Q$ and $U_F(j)$, and the $fail$ event
cannot occur.

\paragraph{Step~2} For proving that a test execution can never deadlock without a
$\epass$ event, it has to be shown that a test execution can neither block at
branch (\ref{eq:ufc}) nor at branch (\ref{eq:ufd}). These cases are
considered separately below.

\paragraph{Step~2.1} Suppose that the test execution blocks at branch (\ref{eq:ufc})
after having run through a trace $s$ with $\#s < j$. Since  $P\lessdet_T Q$
by assumption, $s$ is a trace of $P$, thus $U_F(j)/s = U_F(j,\#s,G(P)/s)$
according to Lemma~\ref{lemma:ufproperties}. Therefore, $U_F(j)/s$  can enter
branch (\ref{eq:ufc}) with any event from $[G(P)/s]^0$. Since we assume that
$(Q\parallel[\Sigma] U_F(j))/s$ deadlocks, this means that $[G(P)/s]^0$ is
{\it not} a hitting set of $\minaccs(Q/s)$, because otherwise at least one
$e\in [G(P)/s]^0$ would be accepted by $Q/s$ and the test execution would not
deadlock. Now suppose that $\minhits(G(P)/s) = \varnothing$. Then   branch
(\ref{eq:ufb}) can be entered, and the test stops after $\epass$. Otherwise,
if $\minhits(G(P)/s) \neq \varnothing$, let $H\in\minhits(G(P)/s)$. Since $H$
contains only elements that are contained in some minimal acceptance of
$P/s$, and all these minimal acceptances are subsets of $[G(P)/s]^0$, $H$ is
a subset of $[G(P)/s]^0$ as well. Since $[G(P)/s]^0$, however, is not a
hitting set of $\minaccs(Q/s)$, also $H$ is not a hitting set of
$\minaccs(Q/s)$. Now this is a contradiction to Lemma~\ref{lemma:hseta},
since $Q\ conf\ P$ by assumption, so $H$ should also be a (not necessarily
minimal) hitting set in $\minaccs(Q/s)$. This proves that the test execution
cannot block  at branch (\ref{eq:ufc}) without being able to pass the test by
entering branch (\ref{eq:ufb}).

\paragraph{Step~2.2} Suppose that the execution blocks at
branch (\ref{eq:ufd}) after having run through some $s\in\trc(Q)\subseteq
\trc(P)$ with $\#s = j$. From Lemma~\ref{lemma:ufproperties} we know that
$U_F(j)/s = U_F(j,k,n) = U_F(j,\#s,G(P)/s)$, so $\minhits(n) =
\minhits(P/s)$. Branch (\ref{eq:ufb}) of $U_F(j,k,n)$ leads always to a PASS
verdict and is taken if $\minhits(n) = \varnothing$. If $\minhits(n) \neq
\varnothing$,  the assumption that $(Q\parallel[\Sigma] U_F(j))/s$ blocks at
branch (\ref{eq:ufd}) implies that there exists some $H\in\minhits(n)$ that
is not a hitting set of $\minaccs(Q/s)$. Again, by Lemma~\ref{lemma:hseta},
this contradicts the assumption that $Q\ conf\ P$. As a
consequence, the test execution can never deadlock at branch (\ref{eq:ufd})
without entering branch (\ref{eq:ufb}) and passing the test.

\bigskip \noindent%
Note that the line of reasoning in this proof requires that $Q$ is free of
livelocks, because otherwise a $\epass$ event might not become visible, due
to unbounded sequences of hidden events performed by $Q$. \xbox
\end{proof}
%
\begin{lemma}\label{lemma:mainfexhaustive}
A test suite $\TS_F$ specified as in Theorem~\ref{th:failurestest} is
exhaustive for the fault model specified there.
\end{lemma}
\begin{proof}
Consider a process $Q\in{\cal D}$ with $P\not\lessdet_F Q$. According to
(\ref{eq:failconf}), this non-conformance can be caused in two possible ways
corresponding to the cases $P\not\lessdet_T Q$ and $\neg(Q\ conf\ P)$. These
cases can be characterised as follows:
\begin{description}
\item[Case~1] $\trc(Q)\not\subseteq \trc(P)$
\item[Case~2] There exists a joint trace $s\in\trc(Q)\cap\trc(P)$ and a
    minimal acceptance $A_Q$ of $\minaccs(Q/s)$, such that (see
    Lemma~\ref{lemma:tgtrcref}, (\ref{eq:failrefb}))
\begin{equation}
\label{eq:accsnotcontained}
\forall A_P\in\minaccs(P/s): A_P\not\subseteq A_Q,
\end{equation}
\end{description}
It has to be shown for each of these cases that at least one test execution
of some $(Q\parallel[\Sigma] U_F(j))$ with $j < pq$ ends with the $\efail$
event or deadlocks. We do this by analysing the product graph of the
reference process $P$ and the SUT process $Q$: any trace
$s\in\trc(Q)\cap\trc(P)$ gives rise to a path labelled by the events of $s$
through this product graph. Any error can be detected after running through
such a trace and then either observing an event outside $[P/s]^0$ (the
violation described by Case~1) or identifying an illegal acceptance $A_Q$ (as
in Case~2). It is not guaranteed, however, that $s$ is short enough to be
executed by one of the test cases $U_F(j)$ with $0\le j < pq$. So, it has to
be shown that for any $s$ leading to an error situation, there exists a trace
$u$ of maximal length $pq-1$ leading to the same error.

\medskip
\noindent {\bf Case~1.} Consider a  trace $s.e\in\trc(Q)$ with $s\in\trc(P)$,
but $s.e\not\in\trc(P)$. Such a trace always exists because $\varepsilon$ is
a trace of every process. In this case, $s$ is also a trace of the product
graph $G = G(P)\times G(Q)$ defined in Section~\ref{sec:ntg}, and $G/s =
(G(P)/s,G(Q)/s)$ holds. The length of $s$ is not known, but from the
construction of $G$,  we know that $G$ has at most $pq$ reachable states,
because $G(P)$ has $p$ states, and $G(Q)$ has at most $q$ states. By
Lemma~\ref{lemma:reachproduc}, $(G(P)/s,G(Q)/s)$ can be reached by a trace
$u\in\trc(G)$ of length $\#u < pq$. Now the construction of the transition
function of $G$ implies that $u$ is also a trace of $P$ and $Q$, which means
that $(G(P)/s,G(Q)/s) = (G(P)/u,G(Q)/u)$. Since test $U_F(pq-1)$ accepts all
traces of $P$ up to length $pq-1$, $u$ is also a trace of this test, and, by
construction and by Lemma~\ref{lemma:ufproperties}, $U_F(pq-1)/u =
U_F(pq-1,\#u,G(P)/u)$. Since $s.e\not\in\trc(P)$, $e$ is an element of
$\Sigma-[P/u]^0 = \Sigma - [G(P)/s]^0$. Hence, in at least one execution,
$U_F(pq-1,\#u,G(P)/u)$ executes its first branch (\ref{eq:ufa}) with this
event $e$, so that the test fails. Again, the assumption of non-divergence of
$Q$ is needed for this conclusion.

\medskip
\noindent {\bf Case~2.} We note again that $s$ is a trace of the product
graph $G$, but we do not know its length. Again, by
Lemma~\ref{lemma:reachproduc}, the state $G/s$ can be reached by a trace
$u\in\trc(Q)\cap\trc(P)$ of maximal length $\#u < pq$. We consider the test
$U_F(\# u)$, for which $U_F(\# u)/u = U_F(\#u,\#u,G(P)/u)$, because of
Lemma~\ref{lemma:ufproperties}. $U_F(\#u)$ can always perform branch
(\ref{eq:ufc}) until the trace $u$ has been completely processed.
 $U_F(\#u,\#u,G(P)/u)$ may execute branches (\ref{eq:ufa}) or (\ref{eq:ufd})
only:~(\ref{eq:accsnotcontained}) implies that $P/s$ has at least one
non-empty minimal acceptance. By (\ref{eq:minhitminaccempty}) this is
equivalent to $\minhits(P/s) = \minhits(G(P)/s)\neq\varnothing$, and we
observe that $G(P)/s = G(P)/u$, so $\minhits(G(P)/u)\neq\varnothing$. As a
consequence, branch (\ref{eq:ufb}) cannot be taken because its guard
condition evaluates to $\isf$  for $U_F(\#u,\#u,G(P)/u)$. The guard condition
$(k < j)$ for branch (\ref{eq:ufc}) evaluates to $\isf$ for
$U_F(\#u,\#u,G(P)/u)$, too. If branch (\ref{eq:ufa}) is executed, the test
always fails. If branch (\ref{eq:ufd}) is executed, the test   deadlocks and
therefore fails for the execution where $Q/u$ selects the minimal acceptance
$A_Q$ as specified in (\ref{eq:accsnotcontained}) and $U_F(\#u,\#u,G(P)/u)$
selects a minimal hitting set $H\in\minhits(P/u)$ that has an empty
intersection with $A_Q$. The existence of such an $H$ is guaranteed because
of Lemma~\ref{lemma:hseta}. As a consequence, $(Q\parallel[\Sigma]U_F(\#
u))/u$ cannot produce the $\epass$ event in this execution; this means that
the test fails. The complete testing assumption guarantees that this
execution really occurs if $(Q\parallel[\Sigma]U_F(\# u))$ is executed
sufficiently often. This concludes the proof. \xbox
\end{proof}
%
Our notion of test can be specialised to deal with traces refinement~(see
Section~\ref{sec:finitecomplete}). We next present an example.

% ==========================================================================

% =======================================================================
\section{Testing for Failures Refinement -- an Example}
\label{sec:case}
% =======================================================================

Generating the test cases $U_F(p)$ specified in  (\ref{eq:UFP}) for the reference
process $P$ discussed in Example~\ref{example:CSP},
results in the   instantiations of initials, minimal hitting sets, and
transition function shown in Fig.~\ref{fig:initialsminhitstrans};
this can be directly derived from $P$'s normalised
transition graph with nodes $N =\{0,1,2,3\}$ displayed in Fig.~\ref{fig:tga}.

\begin{figure}[ht]
\footnotesize
\begin{center}
\begin{minipage}{0.2\textwidth}
	 \begin{eqnarray*}
{ }[0]^0 & = & \{ a \} \\
{ }[1]^0 & = & \{ a,b,c \} \\
{ }[2]^0 & = & \{ a,b,c \} \\
{ }[3]^0 & = & \{ b,c \}
\end{eqnarray*}
	\end{minipage}
	\hfill
	\begin{minipage}{0.33\textwidth}
	 \begin{eqnarray*}
\minhits(0) & = & \{ \{ a\} \} \\
\minhits(1) & = & \{ \{a,b\}, \{c\} \} \\
\minhits(2) & = & \{  \{a,b\},  \{a,c\}\} \\
\minhits(3) & = & \{ \{ b\}, \{c\} \}
\end{eqnarray*}

	\end{minipage}
	\hfill
	\begin{minipage}{0.2\textwidth}
	 \begin{eqnarray*}
t(0,a) & = & 1 \\
t(1,a) & = & 0 \\
t(1,b) & = & 0 \\
t(1,c) & = & 2
\end{eqnarray*}
	\end{minipage}
		\hfill
	\begin{minipage}{0.2\textwidth}
	 \begin{eqnarray*}
t(2,a) & = & 1 \\
t(2,b) & = & 0 \\
t(2,c) & = & 3 \\
t(3,b) & = & 0 \\
t(3,c) &  =& 3
\end{eqnarray*}
	\end{minipage}
\caption{Initials, minimal hitting sets, and transition function of the normalised transition graph displayed in Fig.~\ref{fig:tga}.}
\label{fig:initialsminhitstrans}
\end{center}
\normalsize
\end{figure}

% .....................................................................................
 \begin{figure}
 %%\hspace*{-40mm}
 \begin{center}
\includegraphics[trim=0cm 1.4cm 0cm 1.4cm,clip,scale=.6,%width=\textwidth
                 ]{z.pdf}
\end{center}
%%\vspace*{-10mm}
\caption{Normalised transition graph of faulty implementation $Z$  from Example~\ref{ex:uf1tests}.}
 \label{fig:tgZ}
 \end{figure}
% .......................................................................................

\begin{example}
\label{ex:uf1tests} Consider the following implementation $Z$ of process $P$
from Example~\ref{example:CSP} that is erroneous from the point of view of
failures refinement. In the specification of $Z$, it is assumed that $r_{max}\ge 0$.
\begin{eqnarray*}
  Z & = & a \then (Q_1 \intchoice R_1(r_{max},0))
  \\
  Q_1 & = & a\then Z \extchoice c\then Z
  \\
  R_1(r_{max},k) & = & (k < r_{max}) \& \big(b\then Z \extchoice  c\then R_1(r_{max},k+1)\big)
  \\ & & \extchoice
  \\ & & (k = r_{max}) \& \big(b\then Z \intchoice c\then R_1(r_{max},r_{max})\big)
\end{eqnarray*}
It can be checked with FDR that $Z$ is trace-equivalent to $P$. While $k <
r_{max}$, $Z$ also accepts the same sets of events as $P$. When
$R_1(r_{max},k)$ runs through several recursions and $k = r_{max}$, however,
$R_1(r_{max},k)$ makes an internal choice, instead of offering an external
choice, so $P\not\lessdet_F Z$. Fig.~\ref{fig:tgZ} shows the normalised
transition graph of $Z$ for $r_{max} = 3$.

Running the test $U_F(j)$ against $Z$ for $j=0,\dots,19$ ($G(P)$ has $p = 4$
states and $G(Z)$ has $q=5$, so $pq-1=19$ is the index of the last test to
be executed  according to Theorem~\ref{th:failurestest}), tests $U_F(0),\dots,
U_F(3)$ are passed by $Z$, but $Z$ fails $U_F(4)$, because after execution of
the trace
\[
s = a.c.c.c, \qquad\text{(note that $G(P)/s = \text{node}\ 3$ according to Fig.~\ref{fig:tga})},
\]
the test $U_F(4)$ offers hitting sets from $\minhits(3) = \{\{b\},\{c\}\}$
in branch (\ref{eq:ufd}). Therefore,  there exists one test execution
where $Z/s$ accepts only $\{b\}$ due to the internal choice (note
from Fig.~\ref{fig:tgZ} that
$G(Z)/s = \text{node}\ 4$), while $U_F(4)/s$
only offers $\{c\}$ in branch (\ref{eq:ufd}) or $\{ a\} = \Sigma - [3]^0$ for
branch (\ref{eq:ufa}).  As a consequence, this execution of
$(Z\parallel[\Sigma] U_F(4))/s$ deadlocks, and the $pass$ event cannot be
produced. Another failing execution arises if $Z/s$ chooses to accept only
$\{c \}$, while $U_F(4)/s$ choses to accept only $\{a,b\}$. Therefore,
$%\[
(\epass\then \Stop)\not\lessdet_F  (Z\parallel[\Sigma] U_F(4)) \hide\Sigma,
$ %\]
and the test fails. \xbox
\end{example}

% ======================================================================

% ==========================================================================
\section{Finite Complete Test Suites for CSP Traces Refinement}
\label{sec:finitecomplete}
% ==========================================================================

For establishing traces refinement, the following class of adaptive test
cases are used for a given reference process $P$ and  integers $j \ge 0$.
Just as for the tests developed  in Section~\ref{sec:finitecompletefails} to
verify failures refinement, the tests for traces refinement are derived from
the reference model's transition graph
$$
G(P) = ( N, \ii n, \Sigma, t : N\times\Sigma \pfun N, r : N \fun \mathbb{P}\mathbb{P}(\Sigma)).
$$
In contrast to the tests for failures refinement~(\ref{eq:UFP}), however, we
do not need to check the SUT with respect to its acceptance of hitting sets.
Therefore, these do not occur in the specification of the test cases below.
We use the condition on acceptances $\minaccs(n) = \{ \varnothing \}$ instead
of the condition on hitting sets $\minhits(n) = \varnothing$ in branch
(\ref{eq:utb}). From (\ref{eq:minhitminaccempty}) we know that these
conditions are equivalent, but, with the use of $\minaccs(n) = \{ \varnothing
\}$, \pagebreak we make it unnecessary to calculate hitting sets for
generating these tests from $G(P)$, which is expensive.

%\begin{eqnarray}
%U_T(p) & = & U_T(p,\varepsilon)
%\\
%U_T(p,s) & = & \big(\Extchoice e:(\Sigma - [P/s]^0) @ e \then \efail\then \Stop \big)
%\label{eq:uta}
%\\ & & \extchoice \nonumber
%\\ & & (\minaccs(P/s) = \{ \varnothing \})   \&   \big( \epass \then \Stop \big)
%\label{eq:utb}
%\\ & & \extchoice \nonumber
%\\ & & (\#s < p) \& \big(\Extchoice e:[P/s]^0 @ e \then U_T(p,s.e) \big)
%\label{eq:utc}
%\\ & & \extchoice \nonumber
%\\ & & (\#s = p) \& \big( \epass\then \Stop  \big)
%\label{eq:utd}
%\end{eqnarray}

\begin{eqnarray}
U_T(j) & = & U_T(j,0,\ii n)
\\
U_T(j,k,n) & = & \big(  e:(\Sigma - [n]^0)   \then \efail\then \Stop \big)
\label{eq:uta}
\\ & & \extchoice \nonumber
\\ & & (\minaccs(n) = \{ \varnothing \})   \&   \big( \epass \then \Stop \big)
\label{eq:utb}
\\ & & \extchoice \nonumber
\\ & & (k < j) \& \big( e:[P/s]^0   \then U_T(j,k+1,t(n,e)) \big)
\label{eq:utc}
\\ & & \extchoice \nonumber
\\ & & (k = j) \& \big( \epass\then \Stop  \big)
\label{eq:utd}
\end{eqnarray}
%
It is easy to see that the tests $U_T(j)$ satisfy the properties
\begin{eqnarray}
\label{eq:ifpaT}
  &  & U_T(j)/s = U_T(j,\#s,G(P)/s)
\\
\label{eq:ifpbT}
e\not\in [P/s]^0 & \implies & U_T(j)/s.e = (\efail\then\Stop)
\end{eqnarray}
proven in Lemma~\ref{lemma:ufproperties} for $U_F(j)$ for traces
$s\in\trc(P)$ with $\#s \le j$.

%The difference between adaptive tests $U_T(p)$ for trace refinement and
%$U_F(p)$ for failures refinement consists in the fact that the former do not
%``probe'' the SUT with respect to minimal sets of events to be accepted
%without blocking.

% ==========================================================================
Since the test $U_T(j)$ never blocks any event of an SUT process $Q$ before
terminating, the pass criterion, defined below, can be based on
trace instead of failures refinement as required in (\ref{eq:passF}).
%
\begin{equation}
\label{eq:passT}
Q\ \pass\ U_T(j) \defs (\epass\then\Stop) \lessdet_T (Q\parallel[\Sigma] U_T(j)) \hide \Sigma
\end{equation}
%
If the SUT process $Q$ deadlocks after a trace $s$, and in this case the
reference process $P$ is also in a state where deadlock is possible, this is
captured by the fact that $\minaccs(n) = \{ \varnothing\}$ for $n = G(P)/s$.
Therefore, branch (\ref{eq:utb}) of a test case execution state $U_T(j,k,n)$
with $\#s = k \le j$   can be entered and the  test execution terminates with
$\epass$. If, however, $Q$ blocks after a trace $s'$ and the reference
process satisfies $\minaccs(P/s') \neq\varnothing$, branch (\ref{eq:utb})
cannot be taken, and the test execution stops without producing $\epass$ or
$\efail$. In contrast to the test for failures refinement, this is
interpreted here as a successful test execution, because unexpected blocking
of the SUT does not violate the trace-refinement relation, as long as all
traces executed by the SUT are traces of the reference process. In
particular, if neither $\epass$ nor $\efail$ is ever produced, so that
$(Q\parallel[\Sigma] U_T(j)) \hide \Sigma = \Stop$, the test passes, because
$(\epass\then\Stop) \lessdet_T \Stop$ holds.

The existence of complete, finite test suites is expressed in analogy to
Theorem~\ref{th:failurestest}. A noteworthy difference is that the complete
suite for traces refinement just needs the single adaptive test case
$U_T(pq-1)$, while failures refinement requires the execution of $\{
U_F(0),\dots,U_F(pq-1)\}$. The reason is that $U_T(pq-1)$ identifies trace
errors for all traces up to length $pq$, while $U_F(pq-1)$ only probes for
erroneous acceptances at the end of each trace of length $(pq -1)$.
%
% -------------------------------------------------------------------------
\begin{theorem}\label{th:tracetest}
Let $P$ be a non-terminating, divergence-free CSP process over alphabet $\Sigma$ whose
normalised transition graph $G(P)$ has $p$ states. Define fault domain ${\cal
D}$ as the set of all non-terminating, divergence-free CSP processes over alphabet $\Sigma$,
whose transition graph has at most $q$ states with $q \ge p$. Then the test
suite
\[
\TS_T = \{ U_T(pq-1)   \}
\]
is complete with respect to ${\cal F} = (P,\lessdet_T,{\cal D})$.
\xbox
\end{theorem}
% -------------------------------------------------------------------------
%\begin{proof}
%The theorem follows directly
%from Step~1 in the proof of Lemma~\ref{lemma:mainfsound} and
%Case~1 in the proof of Lemma~\ref{lemma:mainfexhaustive}.
%\xbox
%\end{proof}
%%
%Examples are provided after our discussion of size of the test suites.
%
As for Theorem~\ref{th:failurestest}, the proof is structured in two lemmas,
the first ensuring soundness, and the second exhaustiveness.

% ===============================================================================
\begin{lemma}\label{lemma:mainfsoundtrace}
A test suite $\TS_T$ generated from a CSP process $P$, as specified in
Theorem~\ref{th:tracetest}, is passed by every CSP process $Q$ satisfying
$P\lessdet_T Q$.
\end{lemma}
\begin{proof}
Suppose that $P\lessdet_T Q$, so that $\trc(Q)\subseteq\trc(P)$, and assume
that $s\in\trc(Q)$ with $\#s < pq$. Since $s$ is also a trace of $P$, we can
conclude
$$U_T(pq-1)/s = U_T(pq-1,\#s,G(P)/s)$$
because of (\ref{eq:ifpaT}). \pagebreak Now $\trc(Q)\subseteq\trc(P)$ implies
$[Q/s]^0\subseteq [P/s]^0 = [G(P)/s]^0$, so $U_T(pq-1,\#s,G(P)/s)$ cannot
enter branch (\ref{eq:uta}) and produce a $\efail$ event when running in
parallel with $Q$ and synchronising over $\Sigma$. Therefore, only four
options are available for the test execution    $(Q\parallel[\Sigma]
U_T(j))/s$ to continue.

\medskip
\noindent {\bf Case~1.} $Q/s$ deadlocks and $\minaccs(G(P)/s) = \{
\varnothing \}$. In this case, the test $U_T(pq-1,\#s,G(P)/s)$ enters branch
(\ref{eq:utb}), and its execution stops after $\epass$.

\medskip
\noindent {\bf Case~2.} $Q/s$ deadlocks, but
$\minaccs(G(P)/s)\neq\{\varnothing\}$. In this case, the whole test execution
deadlocks, and this means that neither a $\epass$ nor a $\efail$ event is
produced, so the test execution is passed.

\medskip
\noindent {\bf Case~3.} $Q/s$ selects an event $e\in[Q/s]^0$ and $\#s <
pq-1$. In this case, the test $U_T(pq-1)$ in state $U_T(pq-1,\#s,G(P)/s)$ can
also engage in $e$ by entering branch (\ref{eq:utc}), and its execution
continues without producing a $\epass$ or a $\efail$ event.

\medskip
\noindent {\bf Case~4.} $\#s = pq-1$ holds. In this case,
$U_T(pq-1,\#s,G(P)/s)$ can enter branch (\ref{eq:utd}), and the test
execution stops after $\epass$.

\bigskip\noindent%
This case analysis shows that every   execution of $(Q\parallel[\Sigma]
U_T(j))$ either stops after $\epass$ or produces neither $\epass$ nor
$\efail$. This proves that $Q$ passes test $U_T(pq-1)$ according to the pass
criterion (\ref{eq:passT}). \xbox
\end{proof}
% ===============================================================================

\begin{lemma}\label{lemma:mainfexhaustivetrace}
A test suite $\TS_T$ specified as in Theorem~\ref{th:tracetest} is
exhaustive for the fault model specified there.
\end{lemma}
\begin{proof}
As in the proof for failures testing, we construct the product graph
$G=G(P)\times G(Q)$ and recall that every trace $s\in\trc(P)\cap\trc(Q)$ is
associated with a path through $G$ labelled with the same events as $s$, such
that $G/s = (G(P)/s,G(Q)/s)$. Furthermore, we recall from
Lemma~\ref{lemma:extendV} that the graph state $(G(P)/s,G(Q)/s)$ can always
be reached by a trace $u$ of length less or equal $pq-1$, where  the order of
$G(P)$ is $p$ and that of $G(Q)$ is $q$.

Suppose that $P\not\lessdet_T Q$. Since the empty trace is a trace of every
process, there exists a trace $s\in \trc(Q)\cap\trc(P)$ and an event $e\in
[Q/s]^0$ such that $e\not\in [P/s]^0$. Let $u\in \trc(Q)\cap\trc(P)$ be a
trace with $\#u < pq$ and $G/u = (G(P)/s,G(Q)/s)$. Then
$$
U_T(pq-1)/u = U_T(pq-1,\#u,G(P)/s).
$$
By assumption, $e\in (\Sigma -[P/s]^0) = (\Sigma - [G(P)/s]^0)$. Since
$G(Q)/u = G(Q)/s$, $Q/u$ can engage into $e$. Then $U_T(pq-1,\#u,G(P)/s)$ can
enter branch (\ref{eq:uta}), and the test execution stops after having
produced $\efail$. This proves that $Q$  fails test $U_T(pq-1)$. \xbox
\end{proof}
%
Having established completeness of our test suites, we consider
complexity of a testing technique that uses them.

\section{Complexity Considerations} \label{sec:complexity}
% ================================================================================

%In this section, we calculate estimates for the maximal number of test
%executions to be performed when testing for failures refinement
%and---as a corollary---trace refinement.
%Theorem~\ref{th:failurestest} specifies that all tests $U_F(j),\ 0\le j < pq$
%need to be executed, where $p$ denotes the number of nodes in the transition
%graph of the reference process $P$, and $q\ge p$ is an estimate for the
%maximal number of nodes in the SUT's transition graph. Therefore, we will first
%calculate a bound for the number of test executions to be performed for test
%$U_F(j)$ and then summarise  these bounds over all $j$ from $0$ to $pq-1$.
%
%For the worst-case estimate, we first introduce a CSP reference process $\pmax$
%which
%turns out to be--given a fixed alphabet $\Sigma$---the test
%model leading to the maximal number
%of test executions for every $U_F(j)$ when considering large alphabets. In the case
%of small alphabets and large values of $pq-1$, it will turn out below that a variant
%of $\pmax$ will lead to the maximal number of executions. The conditions for this
%can also be clearly specified.

%we assume that $P$ never allows for early
%deadlock (so $\minhits(P/s)$ is never empty) and that the SUT $Q$ is a
%correct failures refinement. Therefore, all test executions
%$(Q\parallel[\Sigma] U_F(j))$ stop after having run through a trace of $Q$ of
%length $j+1$, because their is no early termination due to entering branches
%(\ref{eq:ufa}), (\ref{eq:ufb}), or due to an illegal deadlock of $Q$. As can
%be seen from the specification of the test cases $U_F(j)$ (see
%Section~\ref{sec:finitecompletefails}), the number of executions ending in a
%$\epass$ event corresponds to the number $\ell$ of traces $s$ of $P$ with
%length equal to $j$, multiplied by the number $h$ of minimal hitting sets in
%$\minhits(P/s)$. For the tests $U_T(j)$ verifying trace refinement (see
%Section~\ref{sec:finitecomplete}), the number of executions equals $\ell$,
%since there is no equivalent in $U_T(j)$ to checking different hitting sets
%in the last step of a test execution. \fixme{But doesn't this add to the
%number of executions?} \fxnote{jp: should be clear now from the revised test}

Since we have finite complete CSP test suites, it is useful for the first
time to calculate how many test executions are needed when using them.
Previous work did not consider sufficient conditions for finiteness, so
complexity was not a concern. We answer the following questions. (1)~What is
the worst-case bound on the number of test executions to be performed to
verify an SUT with respect to failures refinement, when we use our test
suite? (2)~What is the worst-case bound for traces refinement? (3)~Is it
possible to reduce the maximal length of traces when testing for failures or
traces refinement? We consider the first question~(1) in
Section~\ref{section:complexity:failures}), where we also discuss whether it
it is possible to reduce the number of test executions with a different test
suite.  With the answer to question~(1), question~(2) is a fairly simple
consequence we discuss briefly also in
Section~\ref{section:complexity:failures}. Question~(3) is the subject of
Section~\ref{section:complexity:length}.

% -------------------------------------------------------------------------
\subsection{Estimates for the Maximal Number of Failures Test Executions}
\label{section:complexity:failures}

An arbitrary CSP process $P$ might have $\minhits(P/s) = \varnothing$ for
some traces $s$, so that a test case $U_F(j)$ for a $j$ greater than the size
of $s$ can enter branch~(\ref{eq:ufb}). In this case, further executions are
needed to consider traces that have $s$ as a prefix.  To provide a bound on
the number of test executions needed, we first define a process $\pmax$~(see
(\ref{eq:pmax})), which, when used as a reference process, requires the
maximal number of test executions among all reference processes $P$
fulfilling $\minhits(P/s) \neq \varnothing$ for all traces $s$. For $\pmax$,
we can establish the actual number of test executions required~(see
(\ref{eq:pmaxcomplexity})). We then show that the order of magnitude of the
worst-case bound for the number of test executions is the same also for
reference processes $P$ that may have $\minhits(P/s) = \varnothing$ for some
$s$.

% -----------------------------------------------------------------------------
\paragraph{A Reference Process} Given an alphabet $\Sigma$ of size $\card{\Sigma} =
n\ge 2$, define a collection of subsets of $\Sigma$ by
\begin{equation}\label{eq:defC}
  {\cal C} = \{ A\subseteq\Sigma~|~\card{A} = n - \lfloor\frac{n}{2}\rfloor + 1 \}.
\end{equation}
%
With this choice of ${\cal C}$, define
\begin{equation}\label{eq:pmax}
  \pmax = \Intchoice_{A\in{\cal C}} e:A\then \pmax
\end{equation}
%
The relevant properties of $\pmax$ are summarised in the following lemma.
%
\begin{lemma}\label{lemma:pmax}
  Given alphabet $\Sigma$ with cardinality $\card{\Sigma} = n\ge 2$,
  process $\pmax$ fulfils
  %
  \begin{eqnarray}
  {}[\pmax/s]^0 & = & \Sigma \quad\text{for all $s\in \Sigma^*$}
  \label{eq:pmaxa}
  \\
  \trc(\pmax) & = & \Sigma^*
  \label{eq:pmaxb}
  \\
  \minaccs(P/s) & = & {\cal C} \quad\text{for all $s\in \Sigma^*$}
  \label{eq:pmaxc}
  \\
  \minhits(P/s) & = & \minhits({\cal C}) \quad\text{for all $s\in \Sigma^*$}
  \label{eq:pmaxe}
  \\
  \card{\minhits(P/s)} & = & \binom{n}{\lfloor\frac{n}{2}\rfloor}
  \quad \text{for all $s\in \Sigma^*$}
  \label{eq:pmaxd}
  \\
  \minhits({\cal C})  & = & \{ H\subseteq \Sigma~|~\card{H} = \lfloor\frac{n}{2}\rfloor\}
  \label{eq:pmaxf}
  \end{eqnarray}
\end{lemma}
\begin{proof}
Since $\bigcup_{A\in{\cal C}} A = \Sigma $ by construction of ${\cal C}$,
$[\pmax/s]^0 = \Sigma$ as stated by (\ref{eq:pmaxa}). Since $\pmax/e = \pmax$
for all $e\in\Sigma$, this proves statement (\ref{eq:pmaxb}). The internal
choice construct used in the specification of $\pmax$ implies
$\minaccs(\pmax) = {\cal C}$. Again, $\pmax/e = \pmax$ for all $e\in\Sigma$
implies $\minaccs(\pmax/s) = {\cal C}$ for all traces of $\pmax$, so this
shows (\ref{eq:pmaxc}). Statement (\ref{eq:pmaxe}) is a direct consequence of
(\ref{eq:pmaxc}). Let $H$ be any minimal hitting set of ${\cal C}$. Then $H$
contains at least $\lfloor{\frac{n}{2}}\rfloor$ elements, because otherwise
$\card{\Sigma\setminus H} > n-\lfloor{\frac{n}{2}}\rfloor$, and any subset
$A\subseteq \Sigma\setminus H$ with cardinality
$n-\lfloor{\frac{n}{2}}\rfloor+1$  would be contained in ${\cal C}$, but
satisfy $A\cap H=\varnothing$. Since
$\lfloor{\frac{n}{2}}\rfloor+n-\lfloor{\frac{n}{2}}\rfloor+1=n+1$, we
conclude that any $\lfloor{\frac{n}{2}}\rfloor$-element subset of $\Sigma$
intersects  every element of ${\cal C}$.  Therefore, every minimal hitting
set of ${\cal C}$ has exactly $\lfloor{\frac{n}{2}}\rfloor$ elements; this
shows (\ref{eq:pmaxf}) and $\card{\minhits({\cal C})} =
\binom{n}{\lfloor{\frac{n}{2}}\rfloor}$. The latter shows  (\ref{eq:pmaxd})
and completes the proof. \xbox
\end{proof}

% -----------------------------------------------------------------------------
\paragraph{Test Cases of $\pmax$} The test cases $U_F(j)$ generated from $\pmax$ can
never enter branch (\ref{eq:ufa}), because $\pmax/s$ has initials $\Sigma$
for all traces $s\in\trc(\pmax)$ according to (\ref{eq:pmaxa}). Moreover,
they can never enter branch (\ref{eq:ufb}), because $\minhits(\pmax/s)$ is
never empty according to (\ref{eq:pmaxd}). Finally, the minimal hitting sets
used to probe the SUT at the end of a non-blocking test execution are always
the hitting sets of ${\cal C}$ according to (\ref{eq:pmaxe}). This results in
the following test case structure.
\begin{eqnarray*}
\label{eq:UFPpmax}
U_F(j) & = & U_F(j,0,\ii n)
\\
U_F(j,k,n) & = &   (k < j) \& \big(e:\Sigma   \then U_F(j,k+1,t(n,e) \big)
\\ & & \extchoice
\\ & & (k = j) \& \big( \Intchoice_{H\in\minhits({\cal C})} (e:H   \then \epass \then\Stop   )  \big)
\end{eqnarray*}
This means that the branches of $U_F(j)$ that can lead to an early
termination are not feasible. All tests deadlock, or run to the end of a
trace of size $j$ and then present a choice of events from a minimal hitting
set.

Moreover, Theorem~\ref{th:sperner} establishes that given an alphabet with
$n$ elements, there is no Sperner family consisting of more than
$\binom{n}{\lfloor\frac{n}{2}\rfloor}$. In addition, we recall the minimal
hitting sets calculated from a given set of minimal acceptances are a Sperner
family~(see discussion in Section~\ref{sec:sperner}). So, (\ref{eq:pmaxd})
establishes that there is no possibility of providing more choices fom
minimal hitting sets with a test derived from a process different from
$\pmax$.

In summary, the tests derived from $\pmax$ require the most test executions
when compared to tests derived from any other CSP process $P$ whose
collections of minimal hitting sets $\minhits(P/s)$ are never empty for any
trace $s$.

% -----------------------------------------------------------------------------
\paragraph{Maximal Number of Test Executions for $\pmax$}
When considering the number of test executions to be performed using $U_F(j)$
derived from $\pmax$ against some SUT $Q$ for all $j = 0,\dots,pq-1$, and
considering that we need to cover all the possible behaviours of $Q$, the
maximal number of test executions is only reached if (a)~$Q$ is a correct
refinement of $\pmax$ (or more generally, of the reference process) and
(b)~its traces are $\Sigma^*$.
%or if it fails in the very last execution of the very last $U_F(j)$
%executed against $Q$. If an erroneous behaviour of $Q$ is revealed before
%this very last execution, the test experiment is stopped (and $Q$ can be
%fixed before executing the suite again). Therefore, we consider only correct
%SUTs $Q$ when determining the maximal number of executions that can be
%executed.
%
%For the  $U_F(j)$ generated from $\pmax$, the number of test executions
% to be performed is maximal if $\pmax\lessdet_F Q$ and
%$\trc(Q) = \Sigma^*$.
In such a situation, no test execution blocks early, because $Q/s$ can always
engage into some $e\in\Sigma$  while $\#s<j$, and never blocks in the last
step when $\#s = j$ and a hitting set $H\in\minhits({\cal C})$ is offered by
the test case. The resulting number of executions in this case is
%
\begin{equation}
\label{eq:maxexec}
n^{j}\cdot \binom{n}{\lfloor\frac{n}{2}\rfloor},
\end{equation}
%
because all traces up to length $j$ can be executed with $U_F(j)$
entering branch (\ref{eq:ufc}), and each of these traces is followed by one
event from each of the hitting sets of ${\cal C}$ since $Q$ is correct.  %(b) leads to the maximal number of
%executions possible, because it is responsible for the factor *n^{(pq-1)} in
%the upper bound (47), (48) [n = |Sigma|]. Considering a reference process
%with fewer traces or an implementation Q with fewer traces would
%significantly reduce the value of this factor, while just adding one
%additional execution possibility for branch (20), after which the test suite
%is stopped due to failure.

The number of executions in (\ref{eq:maxexec}) is indeed maximal for all
reference processes $P$ fulfilling $\minhits(P/s)\neq\varnothing$ for all
traces $s$. All these processes can never enter branch (\ref{eq:ufb}), and,
if an execution with an SUT entered branch (\ref{eq:ufa}), this would only
lead to early termination of the whole test suite, because a failure has been
detected. As a consequence, $\card{\Sigma}^{j}$ is the maximal number of
traces to be executed up to length $j$, and from Theorem~\ref{th:sperner} we
know that the number $\binom{n}{\lfloor\frac{n}{2}\rfloor}$ of hitting sets
to be tested at the end of each trace of length $j$ is already maximal.

Summing up  formula (\ref{eq:maxexec}) over all test cases
$U_F(0),\dots,U_F(pq-1)$ to be executed and applying the formula for the sum
of a geometric progression,  this results int
%
\begin{equation}\label{eq:pmaxcomplexity}
\sum_{j=0}^{pq-1} n^{j}\cdot \binom{n}{\lfloor\frac{n}{2}\rfloor}  =
\binom{n}{\lfloor\frac{n}{2}\rfloor}\cdot\frac{1-n^{pq}}{1-n}\quad\text{with $n=\card{\Sigma}$}
\end{equation}
%
as the maximal number of test executions to be performed when testing an
error-free SUT $Q$ with $\trc(Q) = \Sigma^*$ against the reference process
$\pmax$. If we are interested only in the order of magnitude, we have
%
\begin{equation}
\label{eq:maxO}
O\big(\binom{n}{\lfloor\frac{n}{2}\rfloor}\cdot n^{pq-1}\big)\quad\text{with $n=\card{\Sigma}$}.
\end{equation}

% -----------------------------------------------------------------------------
\paragraph{Considering Empty Collections of Minimal Hitting Sets}
The argument so far has shown that the tests derived from $\pmax$ require the
most test executions when considering processes whose collections of minimal
hitting sets are never empty. %Lemma~\ref{lemma:failureshittingsets} implies
%that an empty collection $\minhits(P/s)$ is equivalent to $(s,A)\in\fails(P)$
%for all $A\subseteq\Sigma$. This is equivalent to $\maxrefs(P/s) = \{ \Sigma
%\}$ or $\minaccs(P/s) = \{ \varnothing\}$.
%
It remains to consider whether reference processes $Z$ possessing failures
$(s,\Sigma)$ may require more test executions for their associated tests
$U_F(j)$ than the bound given for $\pmax$ in (\ref{eq:pmaxcomplexity}),
because process states $Z/s$ with $\minhits(Z/s)=\varnothing$ allow for test
executions entering branch (\ref{eq:ufb}). To this end, consider a test case
$U_F(j)$ constructed from such a process $Z$. Every trace $s\in\trc(Z)$ with
$\#s<j$ ending in a process state $Z/s$ with $\minhits(Z/s)=\varnothing$
allows for
%
\begin{itemize}
\item one execution of branch (\ref{eq:ufb}), where the test execution
$(Z\parallel[\Sigma]U_F(j))/s$ stops after $\epass$, and
\item $\card{[Z/s]^0}$   continuations of the test execution with events $e\in [Z/s]^0$.
\end{itemize}
%
For every trace $s\in\trc(Z)$ with $\#s=j$,
%
\begin{itemize}
\item one execution of branch (\ref{eq:ufb}) follows if
$\minhits(Z/s)=\varnothing$, and otherwise
\item $\card{\minhits(Z/s)}$ executions checking acceptance of minimal hitting sets.
\end{itemize}
%
For a rough estimate of the worst-case upper bound suppose that
%
\begin{enumerate}
\item $[Z/s]^0 = \Sigma$ for all traces $s$ of $Z$,
\item all traces $s$ with $\#s<j$ end in a state with empty minimal hitting sets, and
\item all traces $s$ with $\#s=j$ end in a state with a maximal number $\binom{n}{\lfloor n/2\rfloor}$
 of hitting sets.
\end{enumerate}
%
We note that this scenario is not feasible for all $j\in\{0,\dots,pq-1\}$,
because the traces of $U_F(p-1)$ already cover all states of $Z$'s transition
graph according to Lemma~\ref{lemma:extendV}, and if all states of $Z$ have
empty hitting sets, there are no acceptance checks to be performed in the
last step of the test execution. Therefore, \pagebreak the upper bound
calculated next cannot be reached by a CSP process. With the three
assumptions above, nevertheless, we can calculate that
%
\begin{itemize}
\item  $U_F(0)$ has  $\binom{n}{\lfloor n/2\rfloor}$ executions,
\item $U_F(j)$, for $j > 0$ has $\sum_{i=0}^{j-1} n^i = \frac{n^j -
    1}{n-1}$ executions of branch (\ref{eq:ufb}), ($n = \card{\Sigma}$),
    and
\item $U_F(j)$, for $j > 0$ has $\binom{n}{\lfloor n/2\rfloor}\cdot n^j$
    executions where the acceptance of hitting sets is checked after having
    run through a trace of length $j$.
\end{itemize}
%
Summing up over all $U_F(j)$ for $j=0,\dots,pq-1$, an upper bound $B$ may be calculated as follows.
%
\begin{eqnarray*}
B & = & \binom{n}{\lfloor n/2\rfloor} + \sum_{j=1}^{pq-1} \frac{n^j - 1}{n-1} +
\sum_{j=1}^{pq-1}\binom{n}{\lfloor n/2\rfloor}\cdot n^j
   \\
& = &    \sum_{j=1}^{pq-1} \frac{n^j - 1}{n-1} +
\sum_{j=0}^{pq-1}\binom{n}{\lfloor n/2\rfloor}\cdot n^j
\\
& = & \frac{n^{pq}-n pq+pq-1}{(n-1)^2} +
\binom{n}{\lfloor n/2\rfloor} \cdot \frac{n^{pq} - 1}{n-1}
\\
& = &
\\
& = & \frac{\binom{n}{\left\lfloor \frac{n}{2}\right\rfloor}(n-1) \left(n^{pq}-1\right) +n^{pq}-npq+pq-1}{(n-1)^2}
\end{eqnarray*}
%
Since $B$ cannot be reached anyway, we just calculate its order of magnitude,
and this results again in $O\big(\binom{n}{\lfloor\frac{n}{2}\rfloor}  \cdot
n^{pq-1}\big)$, as calculated already for $\pmax$ in (\ref{eq:maxO}).
%
Summarising these complexity calculations, we have the following theorem.
%
\begin{theorem}
\label{th:maxexecs}
Given a process alphabet $\Sigma$, consider a fault model ${\cal F} = (P,\lessdet_F,{\cal D})$, such that
the normalised transition graph of $P$ has $p$ states, and the fault domain
${\cal D}$ contains all processes $Q$ over alphabet $\Sigma$, such that $G(Q)$ has at most
$q\ge p$ states. Then the maximal
number of test executions to be performed   using the complete test
suite $\TS_F = \{ U_F(j)~|~0 \le j < pq  \}$ created from $P$ as specified in Theorem~\ref{th:failurestest} is of order
%
\begin{equation*}
O\big(\binom{n}{\lfloor\frac{n}{2}\rfloor}\cdot n^{pq-1}\big)\quad\text{with $n=\card{\Sigma}$}.
\end{equation*}
For processes $P$ satisfying $(s,\Sigma)\not\in\fails(P)$ for all traces $s$, the reachable
precise  upper bound is given by
%
\begin{equation*}
\binom{n}{\lfloor\frac{n}{2}\rfloor}\cdot\frac{1-n^{pq}}{1-n}\quad\text{with $n=\card{\Sigma}$}.
\end{equation*}
\xbox
\end{theorem}
%
As established by Lemma~\ref{lemma:hseta}, the number of hitting sets used to
probe the SUT cannot be reduced:~if the reference process is $\pmax$, we need
to consider all of them, otherwise illegal blocking may remain undetected. In
addition, if $\pmax$ defines the behaviour of the SUT, using smaller sets
that are no longer hitting sets lead to a rejection of correct
implementations. Our explicit definition of $\pmax$ to establish this
worst-case is useful to illustrate this point.

In~\cite{Hennessy:1988:ATP:50497}, it is suggested to test {\it every}
non-empty subset of $\Sigma$ whose events cannot be completely refused in a
given process state of the reference model; this leads to a worst-case
estimate of $2^{\card{\Sigma}}-1$ for the number of different sets to be
offered to the SUT in the last step of the test execution. This number is
significantly larger than the worst-case estimate $\binom{n}{\lfloor
n/2\rfloor}$ above. In Fig.~\ref{fig:minhita}, the reduction is visualised by
plots of the two functions. In~\cite{DBLP:conf/icfem/CavalcantiG07}, the
authors also use minimal hitting sets\footnote{However, they are denoted by
{\it minimal acceptances} in~\cite{DBLP:conf/icfem/CavalcantiG07}.}, but do
not give an estimate for the number of test executions.

% .....................................................................................
 \begin{figure}
 %%\hspace*{-40mm}
 \begin{center}
\includegraphics[scale=.7%width=.8\textwidth
                        ]{curvecomparison.pdf}
\end{center}
%\vspace*{-10mm}
\caption{Function plot $2^{\card{\Sigma}}$ versus $\binom{n}{\lfloor \frac{n}{2}\rfloor}$.}
 \label{fig:minhita}
 \end{figure}
% .......................................................................................

% -------------------------------------------------------------------------
\paragraph{Estimate for the Maximal Number of Trace Test Executions}
%\label{section:complexity:traces}
According to Theorem~\ref{th:tracetest}, a complete test suite checking
traces refinement just contains the adaptive test case $U_T(pq-1)$. As
derived for $U_F(j)$ above, the maximal number of executions to be performed
by $(Q\parallel[\Sigma] U_T(pq-1))$ is of order
$O\big(\card{\Sigma}^{pq-1}\big)$.

% -------------------------------------------------------------------------
\subsection{Upper Bound $pq$ for the Maximal Length of Test Traces}
\label{section:complexity:length}

According to Theorem~\ref{th:failurestest}, the tests $U_F(j)$ need to be
executed for $j = 0,\dots,pq-1$ to guarantee completeness. So, the SUT is
verified with test traces up to, and including, length $pq$. With branch
(\ref{eq:ufa}), $U_F(j)$ accepts all traces $s.e$ with $s\in\trc(P), \#s
= j, e\not\in\trc(P/s)$, so erroneous traces up to length $j+1$ are detected.

It is interesting to investigate whether this maximal length is necessary, or
one could elaborate alternative complete test strategies where the
SUT is tested with shorter traces only. Indeed, an example
in~\cite[Exercise~5]{PeleskaHuangLectureNotesMBT} shows that, when testing
for equivalence of deterministic FSMs, it is sufficient to test with
traces of significantly shorter length.

The following example, however, shows that the maximal length $pq$ is really
required when testing for refinement.
%\begin{example}\label{ex:pq}
%Consider the CSP reference process $P$ and an erroneous implementation $Q$
%specified as follows.
%
%\begin{center}
%\begin{minipage}{.4\textwidth}
%\begin{eqnarray*}
%P & = & a \then P_1 \intchoice b \then P_1 \intchoice c \then P_1
%\\
%P_1 & = & a \then P \extchoice b\then P
%\end{eqnarray*}
%\end{minipage}
%\hfill
%\begin{minipage}{.4\textwidth}
%\begin{eqnarray*}
%Q & = & a\then Q_1 \extchoice b\then Q_1
%\\
%Q_1 & = & a\then Q_2 \extchoice b\then Q_2
%\\
%Q_2 & = & a\then Q \intchoice b\then Q
%\end{eqnarray*}
%\end{minipage}
%\end{center}
%
%
%\medskip
%Obviously, $P$'s normalised transition graph has 2 nodes, while $Q$'s graph
%has 3. It is easy to see (and can be checked with FDR4) that $P\lessdet_T Q$,
%but $\neg(P\lessdet_F Q)$. Furthermore, it can also be shown using FDR4 that
%the ``test passed condition''
%\[
%(\epass\then\Stop) \lessdet_F (Q\parallel[\Sigma] U_F(j))\hide \Sigma
%\]
%holds for $U_F(0),\dots,U_F(4)$, but fails for $U_F(5)$. So, the
%non-conformance of $Q$ cannot be detected by any test trace of length less or
%equal to 5, but is revealed (as expected from Theorem~\ref{th:failurestest})
%by a trace of length 6, because the last event offered by the test $U_F(5)$
%is refused by $Q$. \xbox
%\end{example}

\begin{example}\label{ex:pq}
Consider the CSP reference process $P$ and an erroneous implementation $Q$
specified as follows.
%
\begin{eqnarray*}
P & = &  P(0)
\\
P(k) & = & (k < p-1) \& \big( (a \then P(k)) \intchoice ( b \then P(k+1))\big)
\\ & & \extchoice
\\ & & (k = p-1) \& (a \then P(k))
\\
Q & = & Q(0)
\\
Q(k) & = & (k < q-1) \& \big( a \then Q(k+1)    \big)
\\ & & \extchoice
\\ & & ( k = q-1)\& \big( a\then Q(0) \extchoice b\then Q(0)  \big)
\end{eqnarray*}
%
The normalised transition graphs of $P$ and $Q$ are depicted in
Fig.~\ref{fig:examplepq} for the case $p=3,\ q=4$. Using FDR4, it can be
shown for concrete values of $p$ and $q$ that the  ``test passed conditions''
\[
(\epass\then\Stop) \lessdet_F (Q\parallel[\Sigma] U_F(j))\hide \Sigma
\quad
\mathrm{and}
\quad
(\epass\then\Stop) \lessdet_T (Q\parallel[\Sigma] U_T(j))\hide \Sigma
\]
hold for $j = 0,\dots,pq-2$. So, none of the test cases $U_F(j)$ and $U_T(j)$
are capable of detecting failures and traces-refinement violations, if they
only check traces up to length $pq-1$. (We recall that this corresponds to
$j\le pq-2$).

$Q$, however, neither conforms to $P$ in the failures refinement relation,
nor in the traces-refinement relation. This can only be seen when executing
the test $U_F(pq-1)$ and $U_T(pq-1)$, respectively. These tests fail, so this
shows that $P\not\lessdet_F Q$ and $P\not\lessdet_T Q$ according to
Theorem~\ref{th:failurestest} and Theorem~\ref{th:tracetest}. Moreover, this
shows that the maximal trace length $pq$ to be investigated in the tests
cannot be further reduced without losing the completeness property of the
test suites. \xbox
\end{example}
%
Generalising Example~\ref{ex:pq}, it can be shown that for any $p,q \geq 2$,
%\in\mathbb{N}$,
there exist reference processes $P$ with $p$ states and implementation
processes $Q$ with $q$ states, such that a violation of the traces-refinement
property can only be detected with a trace of length $pq$. This is proven in
the following
theorem. %In the proof, we use the processes $P$ and $Q$ introduced in Example~\ref{ex:pq}.

\begin{figure}[tbp]
\begin{center}
\begin{minipage}{.4\textwidth}
\includegraphics[trim=1.4cm 0cm 0cm 0cm,clip,width=1.1\textwidth]{theorem5p.pdf}
\end{minipage}
\hfill
\begin{minipage}{.55\textwidth}
\includegraphics[trim=1.4cm 0cm 0cm 0cm,clip,width=1.1\textwidth]{theorem5q.pdf}
\end{minipage}
\caption{Transition graphs of $P$ (left) and $Q$ (right) from Example~\ref{ex:pq}
 for $p=3$ and $q=4$.}
\label{fig:examplepq}
\end{center}
\end{figure}

% ----------------------------------------------------------------------------
\begin{theorem}\label{th:maxtracelen}
Let $2\le p,q \in\mathbb{N}$. Then there exists a reference process $P$ and an
implementation process $Q$ with the following properties.
\begin{enumerate}
\item $G(P)$ has $p$ states.
\item $G(Q)$ has $q$ states.
\item $P\not\lessdet_T Q$, and therefore, also $P\not\lessdet_F Q$.
\item $\forall s\in\trc(Q): \#s < pq\implies s\in\trc(P)$.
\item $Q\ conf\ P$.
\end{enumerate}
As a consequence, the upper bound $pq$ for the length of traces to be tested
when checking for failures refinement or traces refinement cannot be reduced
without losing the test suite's completeness property.
\end{theorem}
% ----------------------------------------------------------------------------
\begin{proof}
Given $2\le p,q \in\mathbb{N}$, define reference process $P$ and
implementation process $Q$ as in Example~\ref{ex:pq}. It is trivial to see
that $G(P)$ has $p$ nodes and $G(Q)$ has $q$ nodes, so statements 1 and 2 of
the theorem hold.

Using regular expression notation, the traces of $P$ can be specified as
$%\[
\trc(P) = \prefs\big(  (a^*b)^{p-1}a^* \big),
$ %\]
where $\prefs(M)$ denotes the set of all prefixes of traces in
$M\subseteq\Sigma^*$, including the traces of $M$ themselves. The traces of
$Q$ can be specified by
$%\[
\trc(Q) = \prefs\big( (a^{q-1}(a|b))^*  \big).
$ %\]
It is easy to see that $\trc(Q)\not\subseteq\trc(P)$; for example, the trace
$(a^{q-1}b)^p$ is in $\trc(Q)\setminus\trc(P)$, because traces of $P$ contain
at most $p-1$ $b$-events. This proves statement~3.

Let $s \in\trc(Q)$ be any trace of length $\#s = pq-1$, then $s =
(a^{q-1}(a|b))^{p-1}a^{q-1} \in \prefs\big( (a^{q-1}(a|b))^*  \big)$. So, $s$
is also an element of $\trc(P)$, because $(a^{q-1}(a|b))^{p-1}a^{q-1}$ is
also contained in $\prefs\big( (a^*b)^{p-1}a^* \big)$, since $\prefs\big(
(a^*b)^{p-1}a^* \big)$ contains all finite sequences of $a$, where at most
$p-1$ events $b$ have been inserted. This proves statement 4.

To prove statement~5, we observe that the specifications of $P$ and $Q$ lead
to the following sets of minimal acceptances.  In these definition, the
expression $(s\cnt b)$ denotes the number of $b$ events occurring in trace
$s$.
\[
\minaccs(P/s) = \left\{
\begin{array}{ll}
\{ \{a\}, \{b\} \} & \text{for all $s\in\trc(P)$ with $(s\cnt b) <p-1$.}
\\
\{ \{a\} \} & \text{for all $s\in\trc(P)$ with $(s\cnt b) = p-1$.}
\end{array}
\right.
\also
\minaccs(Q/s) = \left\{
\begin{array}{ll}
\{ \{a\}  \} & \text{for all $s\in\trc(Q)$ with $\#s \neq 0\mod(q-1)$.}
\\
\{ \{a,b\} \} & \text{for all $s\in\trc(P)$ with $\#s = 0\mod(q-1)$.}
\end{array}
\right.
\]
So, the minimal acceptance set $A_P = \{a\}$ contained in every
$\minaccs(P/s)$ fulfils $A_P \subseteq A_Q$ for any $A_Q\in\minaccs(Q/s)$,
when $s\in\trc(P)\cap \trc(Q)$. Now Lemma~\ref{lemma:tgtrcref}, in particular
(\ref{eq:failrefb}), can be applied to conclude that $Q\ conf\ P$. \xbox
\end{proof}
%
It is discussed next in Section~\ref{sec:conc} how the number of test traces
to be executed by complete test suites for failures or traces refinement can
still be reduced {\it without} reducing the maximal length.

% =================================================================================


% ==============================================================================
\section{Discussion and Conclusions}
\label{sec:conc}
% ==============================================================================

% ==============================================================================
\paragraph{Further Reductions of the Test Effort} As shown in
Theorem~\ref{th:maxtracelen}, the maximal length $pq$ of traces to be tested
for either  failures refinement or traces refinement cannot be further
reduced. It is noteworthy, however, that when testing FSMs for equivalence,
considerably shorter traces can be used. From the classical results published
in~\cite{chow:wmethod,vasilevskii1973}, for example, it follows that the
maximal trace length to be executed is less or equal to $3p -q$, which is
considerably smaller than $pq$ for $p,q\ge 3$. As a consequence, the
investigation of complete test suites establishing failures or trace
equivalence is of considerable interest and will be discussed in a future
paper.


It is also known from FSM testing that it is not necessary to check {\it all}
traces up to length $pq$ when testing for reduction of FSMs (which
corresponds to trace refinement). Notable complete strategies supporting this
fact have been presented, for example,
in~\cite{hierons_testing_2004,DBLP:conf/forte/DorofeevaEY05,petrenko_testing_2011,simao_reducing_2012}.
From~\cite{Huang2017} it is known that complete FSM testing theories can be
translated to other formalisms, such as Extended Finite State Machines,
Kripke Structures, or CSP, resulting in likewise complete test strategies for
the latter. We intend to study translations of several promising FSM
strategies to CSP. These will effectively reduce the upper bound for the
number of test executions to be performed, which  has been shown to be of the
order $O\big(\card{\Sigma}^{pq-1}\big)$ for our traces-refinement tests in
Section~\ref{sec:complexity}.  The bound $\binom{n}{\lfloor n/2\rfloor}$ for
the number of sets to be used in probing the SUT for illegal deadlocks,
however, cannot be further reduced, as established in
Lemma~\ref{lemma:hseta}.

% ==============================================================================
\paragraph{Adaptive Test Cases} The tests suggested
in~\cite{Hennessy:1988:ATP:50497,DBLP:conf/icfem/CavalcantiG07} were
\emph{preset} in the sense that the trace to be executed was pre-defined for
each test. As a consequence, the authors
of~\cite{DBLP:conf/icfem/CavalcantiG07} introduced \emph{inconclusive} as a
third test result, applicable to the situations where the intended trace of
the execution was blocked, due to legal, but nondeterministic behaviour of
the SUT. We consider this as a disadvantage, since, when aiming at executing
a specific trace $s$ before being able to check the test objective---for
example, the absence of deadlocks when offering a hitting set $H$ of the
minimal acceptances---it may take several tries until the full trace $s$ is
accepted by the SUT. Considering the complete testing assumption described in
Section~\ref{sec:finitecompletefails}, it may even take $c^{\#s}$ tries to
reach the end of the trace $s$, if the SUT can legally block every event of
$s$ due to nondeterminism, so that $c$ trials are required for each event is
accepted.

Those authors, later, in the context of a richer algebra based on CSP, have
considered a framework similar to adaptive testing~\cite{CG15}.  They have,
however, considered only traces refinement, and have not studied complexity.

%\fixme{From the proofs of the main theorems, and what you say below, I
%thought you did have verdicts coming from deadlock, and you simply chose not
%to mark it with an inc event.}
In  contrast to that, our test cases specified in
Section~\ref{sec:finitecompletefails} and \ref{sec:finitecomplete} are
adaptive. This has the advantage that test executions $(Q\parallel[\Sigma]
U_F(j))$ for failures refinement may only stop early with $\epass$ after
traces $s$ satisfying $\minhits(P/s) = \varnothing$, and may deadlock (a)
after a trace $s$ where the SUT illegally deadlocks (so $\minhits(P/s) \neq
\varnothing$ for the reference process, but $\minhits(Q/s) = \varnothing$ for
the implementation $Q$), or (b) in the final step when---just as in the
corresponding test cases specified in~\cite{DBLP:conf/icfem/CavalcantiG07}---
hitting sets $H$ are offered to the SUT and it refuses their acceptance. In
both situations (a) and (b) the test fails. As a consequence, far less test
repetitions are required according to the complete testing assumption than
for the successful execution of all test cases specified
in~\cite{DBLP:conf/icfem/CavalcantiG07}.

Another distinction of our failures test cases $U_F(j)$ to the tests
specified in~\cite{DBLP:conf/icfem/CavalcantiG07} consists in the fact that
the former test both traces refinement and the $conf$ conformance relation in
one go, whereas the latter use separate test suites to establish these two
correctness conditions. Again, we consider the structure of the  test cases
$U_F(j)$ as advantageous, since, when checking acceptance of a hitting set
$H$ after a trace $s$ for $c$ times according to the complete testing
assumption, any acceptance of an illegal event $e\in\Sigma - [P/s]^0$ should
also be revealed within these $c$ tries.


%
%never block before the final step specified
%by branch (\ref{eq:ufd}), and so we do not need inconclusive test results. It
%should be noted, however, that it is necessary for our test verdicts to
%recognise also deadlocks in the final test step and interpret them as FAIL,
%as described in Section~\ref{sec:finitecompletefails}. In practice, this is
%realised by adding a timeout event to the testing environment which indicates
%deadlock situations. For real-time systems, this is an accepted technique,
%because the SUT has to respond within a pre-defined latency interval,
%otherwise its behaviour is considered to be blocked and regarded as a
%failure. The tests executions $(Q\parallel[\Sigma] U_T(j))$ \fixme{Missing
%hiding.} \fxnote{jp: Here, the hiding operator is not necessary: the possible
%test executions are really the traces of $Q$ parallel $U_T(j)$. The hiding is
%only needed to specify the verdict.}for trace refinement never block at all
%before stopping after the verdict $\epass$ or $\efail$, unless the SUT
%process $Q$ has an unexpected deadlock state. Recall from the explanations
%given in Section~\ref{sec:finitecomplete} that the blocking situation also
%leads to passing the test, because $(\epass\then \Stop)\lessdet_T
%(Q\parallel[\Sigma] U_T(j))\hide\Sigma$ still holds if $(Q\parallel[\Sigma]
%U_T(j))\hide\Sigma$ only produces the empty trace.

%The adaptive behaviour of both our test case types $U_F(j)$ and $U_T(j)$,
%however, induces the obligation to check that {\it all} possible executions
%have been performed before the test can be considered as passed. Typically,
%it is therefore assumed that a \emph{complete testing
%assumption}~\cite{hierons_testing_2004} holds, which means that every
%possible behaviour of the SUT is performed after a finite number of test
%executions. In practice, this is realised by executing each test several
%times, recording the traces that have been performed, and using hardware or
%software coverage analysers to determine whether all possible test execution
%behaviours of the SUT have been observed. Therefore, adaptive test cases come
%at the price of having to apply some grey-box testing techniques enabling us
%to decide whether all SUT behaviours have been observed.

% ==============================================================================
\paragraph{Fault Domains} As already mentioned, the work
in~\cite{DBLP:conf/pts/CavalcantiS17} defines a fault domain as the set of
processes that refine a given CSP process.  In that context, only testing for
traces refinement is considered, and the complete test suites may not be
finite. So, the work presented here goes well beyond what is achieved there,
but is restricted to finite and nonterminating reference processes. In
addition, \cite{DBLP:conf/pts/CavalcantiS17} presents an algorithm for test
generation that can be adapted to consider additional selection and
termination criteria, like, for example, the length of the traces used to
construct tests. It would be possible, for instance, to use the bound
indicated here. Moreover, specifying a fault domain as a CSP process allows
us to model domain-specific knowledge using CSP. For example, if an
initialisation component defined by a process $I$ can be regarded as correct
without further testing, we can use $I; RUN$ as a fault domain, to indicate
that any SUT of interest implements $I$ correctly, but afterwards has an
arbitrary behaviour specified by $RUN$.

% ==============================================================================
\paragraph{Implications for CSP Model Checking} As explained in the previous
sections, passing a test is characterised by the failures-refinement check
$(\epass\then\Stop) \lessdet_F (Q\parallel[\Sigma] U_F(j)) \hide \Sigma$ for
failures testing. If the SUT $Q$ is not a programmed piece of software or an
integrated hardware or software system, but just another CSP process
specification, it is of course possible to verify the pass criterion using
the FDR4 model checker. For checking the refinement relation $P\lessdet_F Q$,
the pass criterion has to be verified for $j=0,\dots,pq-1$, where $p$ and $q$
indicate the number of nodes in $P$'s transition graph and the maximal number
of nodes in $Q$'s graph, respectively (Theorem~\ref{th:failurestest}). Since
the test cases $U_F(j)$ have such a simple structure, it is an interesting
question for further research whether checking $(\epass\then\Stop) \lessdet_F
(Q\parallel[\Sigma] U_F(j)) \hide \Sigma$ for $j=0,\dots,pq-1$ can be faster
than directly checking $P\lessdet_F Q$, as one would do in the usual approach
with FDR4. This is of particular interest, since the checks could be
parallelised on several CPUs. Alternatively it is interesting to investigate
whether the check of\footnote{We are grateful to Bill Roscoe for suggesting
this option.}
\[
(\epass\then\Stop) \lessdet_F (Q\parallel[\Sigma] \Intchoice_{j=0}^{pq-1} U_F(j)) \hide \Sigma
\]
may perform better than the check of $P\lessdet_F Q$, since the former allows for
other optimisations in the model checker.

For a large implementation process $Q$, it may be too time consuming to
generate its normalised transition graph, so that its number $q$ of nodes is
unknown. In such a case, our testing approach based on fault domains may
still be used as efficient bug finders:~use the number of nodes of the
normalised transition graph of the reference process as the initial value for
$q$ and increment $q$ from there, as long as each increment reveals new
errors.

% ==============================================================================
\paragraph{Conclusion}
In this paper, we have introduced finite complete testing strategies for
model-based testing against finite, non-terminating CSP reference models. The
strategies are applicable to the conformance relations failures refinement
and traces refinement. The underlying fault domains have been defined as the
sets of CSP processes whose normalised transition graphs do not have more
than a given number of additional nodes, when compared to the transition
graph of the reference process. For these domains, finite complete test
suites are available. We have shown for the strategy to check failures
refinement that the way of probing the SUT for illegal deadlocks in our test
cases is optimal, so that it is not possible to guarantee exhaustiveness with
fewer probes. Moreover, the maximal length of the test traces cannot be
reduced without losing the test suite's completeness property; this holds
both for traces and failures refinement.

% ==============================================================================

\paragraph{Acknowledgements}
The authors would like to thank Bill Roscoe and Thomas Gibson-Robinson for
their advice on using the FDR4 model checker and for very helpful discussions
concerning the potential implications of this paper in the field of model
checking. We are also grateful to Li-Da Tong from National Sun Yat-sen
University, Taiwan, for suggesting the applicability of Sperner's Theorem in
the context of the work presented here. Moreover, we thank Adenilso Simao for
several helpful suggestions. The work of Ana Cavalcanti is funded by the
Royal Academy of Engineering and UK EPSRC Grant EP/R025134/1.

% ===============================================================================

%%%%%%%%%%%%%%%%%%%%%%%%%%%%%%%%%%%%%%%%%%%%%%%%%%%%%%%%%%%%%%%%%%%
%%%%%%%%%%%%%%%%%%%%%%%%%%%%%%%%%%%%%%%%%%%%%%%%%%%%%%%%%%%%%%%%%%%

\begin{thebibliography}{10}
\expandafter\ifx\csname url\endcsname\relax
  \def\url#1{\texttt{#1}}\fi
\expandafter\ifx\csname urlprefix\endcsname\relax\def\urlprefix{URL }\fi
\expandafter\ifx\csname href\endcsname\relax
  \def\href#1#2{#2} \def\path#1{#1}\fi

\bibitem{jp2018ets}
J.~Peleska, \href{https://doi.org/10.1109/ETS.2018.8400703}{Model-based avionic
  systems testing for the airbus family}, in: 23rd {IEEE} European Test
  Symposium, {ETS} 2018, Bremen, Germany, May 28 - June 1, 2018, {IEEE}, 2018,
  pp. 1--10.
\newblock \href {http://dx.doi.org/10.1109/ETS.2018.8400703}
  {\path{doi:10.1109/ETS.2018.8400703}}.
\newline\urlprefix\url{https://doi.org/10.1109/ETS.2018.8400703}

\bibitem{DBLP:conf/isola/0001BH18}
J.~Peleska, J.~Brauer, W.~Huang,
  \href{https://doi.org/10.1007/978-3-030-03427-6\_11}{Model-based testing for
  avionic systems proven benefits and further challenges}, in: T.~Margaria,
  B.~Steffen (Eds.), Leveraging Applications of Formal Methods, Verification
  and Validation. Industrial Practice - 8th International Symposium, ISoLA
  2018, Limassol, Cyprus, November 5-9, 2018, Proceedings, Part {IV}, Vol.
  11247 of Lecture Notes in Computer Science, Springer, 2018, pp. 82--103.
\newblock \href {http://dx.doi.org/10.1007/978-3-030-03427-6\_11}
  {\path{doi:10.1007/978-3-030-03427-6\_11}}.
\newline\urlprefix\url{https://doi.org/10.1007/978-3-030-03427-6\_11}

\bibitem{hierons_testing_2004}
R.~M. Hierons,
  \href{http://doi.ieeecomputersociety.org/10.1109/TC.2004.85}{Testing from a
  nondeterministic finite state machine using adaptive state counting}, {IEEE}
  Trans. Computers 53~(10) (2004) 1330--1342.
\newblock \href {http://dx.doi.org/10.1109/TC.2004.85}
  {\path{doi:10.1109/TC.2004.85}}.
\newline\urlprefix\url{http://doi.ieeecomputersociety.org/10.1109/TC.2004.85}

\bibitem{simao_reducing_2012}
A.~Simão, A.~Petrenko, N.~Yevtushenko,
  \href{https://onlinelibrary.wiley.com/doi/abs/10.1002/stvr.452}{On reducing
  test length for {FSMs} with extra states}, Software Testing, Verification and
  Reliability 22~(6) (2012) 435--454.
\newblock \href {http://dx.doi.org/10.1002/stvr.452}
  {\path{doi:10.1002/stvr.452}}.
\newline\urlprefix\url{https://onlinelibrary.wiley.com/doi/abs/10.1002/stvr.452}

\bibitem{Springintveld2001}
J.~Springintveld, F.~Vaandrager, P.~D'Argenio, Testing timed automata,
  Theoretical Computer Science 254~(1-2) (2001) 225--257.

\bibitem{DBLP:journals/acta/CavalcantiG11}
A.~Cavalcanti, M.-C. Gaudel, Testing for refinement in {\it circus}, Acta Inf.
  48~(2) (2011) 97--147.

\bibitem{Schneider:1995:OST:203471.203475}
S.~Schneider, \href{http://dx.doi.org/10.1006/inco.1995.1014}{An operational
  semantics for timed csp}, Inf. Comput. 116~(2) (1995) 193--213.
\newblock \href {http://dx.doi.org/10.1006/inco.1995.1014}
  {\path{doi:10.1006/inco.1995.1014}}.
\newline\urlprefix\url{http://dx.doi.org/10.1006/inco.1995.1014}

\bibitem{DBLP:journals/cn/Tretmans96}
J.~Tretmans, Conformance testing with labelled transition systems:
  Implementation relations and test generation, Computer Networks and ISDN
  Systems 29~(1) (1996) 49--79.

\bibitem{DBLP:conf/icst/Petrenko16}
A.~Petrenko, \href{http://dx.doi.org/10.1109/ICSTW.2016.9}{Checking experiments
  for symbolic input/output finite state machines}, in: Ninth {IEEE}
  International Conference on Software Testing, Verification and Validation
  Workshops, {ICST} Workshops 2016, Chicago, IL, USA, April 11-15, 2016, {IEEE}
  Computer Society, 2016, pp. 229--237.
\newblock \href {http://dx.doi.org/10.1109/ICSTW.2016.9}
  {\path{doi:10.1109/ICSTW.2016.9}}.
\newline\urlprefix\url{http://dx.doi.org/10.1109/ICSTW.2016.9}

\bibitem{Huang2017}
W.-l. Huang, J.~Peleska,
  \href{http://dx.doi.org/10.1007/s00165-016-0402-2}{Complete model-based
  equivalence class testing for nondeterministic systems}, Formal Aspects of
  Computing 29~(2) (2017) 335--364.
\newblock \href {http://dx.doi.org/10.1007/s00165-016-0402-2}
  {\path{doi:10.1007/s00165-016-0402-2}}.
\newline\urlprefix\url{http://dx.doi.org/10.1007/s00165-016-0402-2}

\bibitem{Hoare:1985:CSP:3921}
C.~A.~R. Hoare, Communicating Sequential Processes, Prentice-Hall, Inc., Upper
  Saddle River, NJ, USA, 1985.

\bibitem{Roscoe2010}
A.~W. Roscoe, Understanding Concurrent Systems, Springer, London, Dordrecht
  Heidelberg, New York, 2010.

\bibitem{976937}
A.~Hall, R.~Chapman, Correctness by construction: developing a commercial
  secure system, IEEE Software 19~(1) (2002) 18--25.
\newblock \href {http://dx.doi.org/10.1109/52.976937}
  {\path{doi:10.1109/52.976937}}.

\bibitem{DBLP:conf/prdc/ShiPK99}
H.~Shi, J.~Peleska, M.~Kouvaras,
  \href{https://doi.org/10.1109/PRDC.1999.816222}{Combining methods for the
  analysis of a fault-tolerant system}, in: 1999 Pacific Rim International
  Symposium on Dependable Computing {(PRDC} 1999), 16-17 December 1999, Hong
  Kong, {IEEE} Computer Society, 1999, pp. 135--142.
\newblock \href {http://dx.doi.org/10.1109/PRDC.1999.816222}
  {\path{doi:10.1109/PRDC.1999.816222}}.
\newline\urlprefix\url{https://doi.org/10.1109/PRDC.1999.816222}

\bibitem{DBLP:conf/amast/ButhKPS97}
B.~Buth, M.~Kouvaras, J.~Peleska, H.~Shi,
  \href{https://doi.org/10.1007/BFb0000463}{Deadlock analysis for a
  fault-tolerant system}, in: M.~Johnson (Ed.), Algebraic Methodology and
  Software Technology, 6th International Conference, {AMAST} '97, Sydney,
  Australia, December 13-17, 1997, Proceedings, Vol. 1349 of Lecture Notes in
  Computer Science, Springer, 1997, pp. 60--74.
\newblock \href {http://dx.doi.org/10.1007/BFb0000463}
  {\path{doi:10.1007/BFb0000463}}.
\newline\urlprefix\url{https://doi.org/10.1007/BFb0000463}

\bibitem{Roscoe:1994:chapter}
A.~W. Roscoe, Model-checking csp, in: A.~W. Roscoe (Ed.), A Classical Mind:
  Essays in Honour of C. A. R. Hoare, Prentice Hall International (UK) Ltd.,
  Hertfordshire, UK, UK, 1994, Ch.~21, pp. 353--378.

\bibitem{Hennessy:1988:ATP:50497}
M.~Hennessy, Algebraic Theory of Processes, MIT Press, Cambridge, MA, USA,
  1988.

\bibitem{DBLP:conf/fm/PeleskaS96}
J.~Peleska, M.~Siegel, \href{https://doi.org/10.1007/3-540-60973-3\_106}{From
  testing theory to test driver implementation}, in: M.~Gaudel, J.~Woodcock
  (Eds.), {FME} '96: Industrial Benefit and Advances in Formal Methods, Third
  International Symposium of Formal Methods Europe, Co-Sponsored by {IFIP} {WG}
  14.3, Oxford, UK, March 18-22, 1996, Proceedings, Vol. 1051 of Lecture Notes
  in Computer Science, Springer, 1996, pp. 538--556.
\newblock \href {http://dx.doi.org/10.1007/3-540-60973-3\_106}
  {\path{doi:10.1007/3-540-60973-3\_106}}.
\newline\urlprefix\url{https://doi.org/10.1007/3-540-60973-3\_106}

\bibitem{peleska1997a}
J.~Peleska, M.~Siegel, Test automation of safety-critical reactive systems,
  South African Computer Jounal 19 (1997) 53--77.

\bibitem{DBLP:conf/icfem/CavalcantiG07}
A.~Cavalcanti, M.~Gaudel,
  \href{https://doi.org/10.1007/978-3-540-76650-6\_10}{Testing for refinement
  in {CSP}}, in: M.~J. Butler, M.~G. Hinchey, M.~M. Larrondo{-}Petrie (Eds.),
  Formal Methods and Software Engineering, 9th International Conference on
  Formal Engineering Methods, {ICFEM} 2007, Boca Raton, FL, USA, November
  14-15, 2007, Proceedings, Vol. 4789 of Lecture Notes in Computer Science,
  Springer, 2007, pp. 151--170.
\newblock \href {http://dx.doi.org/10.1007/978-3-540-76650-6\_10}
  {\path{doi:10.1007/978-3-540-76650-6\_10}}.
\newline\urlprefix\url{https://doi.org/10.1007/978-3-540-76650-6\_10}

\bibitem{DBLP:conf/pts/CavalcantiS17}
A.~Cavalcanti, A.~da~Silva~Sim{\~{a}}o,
  \href{https://doi.org/10.1007/978-3-319-67549-7\_2}{Fault-based testing for
  refinement in {CSP}}, in: N.~Yevtushenko, A.~R. Cavalli, H.~Yenig{\"{u}}n
  (Eds.), Testing Software and Systems - 29th {IFIP} {WG} 6.1 International
  Conference, {ICTSS} 2017, St. Petersburg, Russia, October 9-11, 2017,
  Proceedings, Vol. 10533 of Lecture Notes in Computer Science, Springer, 2017,
  pp. 21--37.
\newblock \href {http://dx.doi.org/10.1007/978-3-319-67549-7\_2}
  {\path{doi:10.1007/978-3-319-67549-7\_2}}.
\newline\urlprefix\url{https://doi.org/10.1007/978-3-319-67549-7\_2}

\bibitem{5533149}
L.~Shi, X.~Cai, An exact fast algorithm for minimum hitting set, in: 2010 Third
  International Joint Conference on Computational Science and Optimization,
  Vol.~1, 2010, pp. 64--67.
\newblock \href {http://dx.doi.org/10.1109/CSO.2010.240}
  {\path{doi:10.1109/CSO.2010.240}}.

\bibitem{chow:wmethod}
T.~S. Chow, Testing software design modeled by finite-state machines, IEEE
  Transactions on Software Engineering SE-4~(3) (1978) 178--186.

\bibitem{vasilevskii1973}
M.~P. Vasilevskii, Failure diagnosis of automata, Kibernetika (Transl.) 4
  (1973) 98--108.

\bibitem{luo_test_1994}
G.~Luo, G.~von Bochmann, A.~Petrenko,
  \href{http://doi.ieeecomputersociety.org/10.1109/32.265636}{Test selection
  based on communicating nondeterministic finite-state machines using a
  generalized wp-method}, {IEEE} Trans. Software Eng. 20~(2) (1994) 149--162.
\newblock \href {http://dx.doi.org/10.1109/32.265636}
  {\path{doi:10.1109/32.265636}}.
\newline\urlprefix\url{http://doi.ieeecomputersociety.org/10.1109/32.265636}

\bibitem{fdr}
T.~Gibson-Robinson, P.~Armstrong, A.~Boulgakov, A.~Roscoe, {FDR3 --- A Modern
  Refinement Checker for CSP}, in: E.~Ábrahám, K.~Havelund (Eds.), Tools and
  Algorithms for the Construction and Analysis of Systems, Vol. 8413 of Lecture
  Notes in Computer Science, 2014, pp. 187--201.

\bibitem{Roscoe:1997:TPC:550448}
A.~W. Roscoe, The Theory and Practice of Concurrency, Prentice Hall PTR, Upper
  Saddle River, NJ, USA, 1997.

\bibitem{sperner_satz_1928}
E.~Sperner, \href{https://doi.org/10.1007/BF01171114}{Ein {Satz} {\"u}ber
  {Untermengen} einer endlichen {Menge}}, Mathematische Zeitschrift 27~(1)
  (1928) 544--548.
\newblock \href {http://dx.doi.org/10.1007/BF01171114}
  {\path{doi:10.1007/BF01171114}}.
\newline\urlprefix\url{https://doi.org/10.1007/BF01171114}

\bibitem{PeleskaHuangLectureNotesMBT}
J.~Peleska, W.-l. Huang, Test Automation - Foundations and Applications of
  Model-based Testing, University of Bremen, 2017, lecture notes, available
  under
  http://www.informatik.uni-bremen.de/agbs/jp/papers/test-automation-huang-peleska.pdf.

\bibitem{DBLP:conf/forte/DorofeevaEY05}
R.~Dorofeeva, K.~El{-}Fakih, N.~Yevtushenko,
  \href{https://doi.org/10.1007/11562436_16}{An improved conformance testing
  method}, in: F.~Wang (Ed.), Formal Techniques for Networked and Distributed
  Systems - {FORTE} 2005, 25th {IFIP} {WG} 6.1 International Conference,
  Taipei, Taiwan, October 2-5, 2005, Proceedings, Vol. 3731 of Lecture Notes in
  Computer Science, Springer, 2005, pp. 204--218.
\newblock \href {http://dx.doi.org/10.1007/11562436_16}
  {\path{doi:10.1007/11562436_16}}.
\newline\urlprefix\url{https://doi.org/10.1007/11562436_16}

\bibitem{petrenko_testing_2011}
A.~Petrenko, N.~Yevtushenko, Adaptive testing of deterministic implementations
  specified by nondeterministic fsms, in: Testing Software and Systems, no.
  7019 in Lecture Notes in Computer Science, Springer, Berlin, Heidelberg,
  2011, pp. 162--178.

\bibitem{CG15}
A.~L.~C. Cavalcanti, M.-C. Gaudel,
  \href{https://www-users.cs.york.ac.uk/~alcc/publications/papers/CG15.pdf}{Test
  selection for traces refinement}, Theoretical Computer Science 563~(0) (2015)
  1 -- 42.
\newblock \href {http://dx.doi.org/10.1016/j.tcs.2014.08.012}
  {\path{doi:10.1016/j.tcs.2014.08.012}}.
\newline\urlprefix\url{https://www-users.cs.york.ac.uk/~alcc/publications/papers/CG15.pdf}

\end{thebibliography}

%%%%%%%%%%%%%%%%%%%%%%%%%%%%%%%%%%%%%%%%%%%%%%%%%%%%%%%%%%%%%%%%%%%
%%%%%%%%%%%%%%%%%%%%%%%%%%%%%%%%%%%%%%%%%%%%%%%%%%%%%%%%%%%%%%%%%%%

\end{document}

%%%%%%%%%%%%%%%%%%%%%%%%%%%%%%%%%%%%%%%%%%%%%%%%%%%%%%%%%%%%%%%%%%%
%%%%%%%%%%%%%%%%%%%%%%%%%%%%%%%%%%%%%%%%%%%%%%%%%%%%%%%%%%%%%%%%%%%
