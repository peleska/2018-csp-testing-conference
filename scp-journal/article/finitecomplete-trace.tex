% ==========================================================================
\section{Finite Complete Test Suites for CSP Trace Refinement}
\label{sec:finitecomplete}
% ==========================================================================

For establishing trace refinement, the following class of adaptive test cases
are used for a given reference process $P$ and  integers $j \ge 0$. Just as
for the tests developed  in Section~\ref{sec:finitecompletefails} to verify
failures refinement, the tests for trace refinement are derived from the
reference model's transition graph
$$
G(P) = ( N, \ii n, \Sigma, t : N\times\Sigma \pfun N, r : N \fun \mathbb{P}\mathbb{P}(\Sigma)).
$$
In contrast to the tests for failures refinement~(\ref{eq:UFP}), however, we
do not need to check the SUT with respect to its acceptance of hitting sets.
Therefore, these do not occur in the specification of the test cases below.
We use the condition on acceptances $\minaccs(n) = \{ \varnothing \}$ instead
of the condition on hitting sets $\minhits(n) = \varnothing$ in branch
(\ref{eq:utb}). From (\ref{eq:minhitminaccempty}) we know that these
conditions are equivalent, but, with the use of $\minaccs(n) = \{ \varnothing
\}$, we make it unnecessary to calculate hitting sets for generating these
tests from $G(P)$.

%\begin{eqnarray}
%U_T(p) & = & U_T(p,\varepsilon)
%\\
%U_T(p,s) & = & \big(\Extchoice e:(\Sigma - [P/s]^0) @ e \then \efail\then \Stop \big)
%\label{eq:uta}
%\\ & & \extchoice \nonumber
%\\ & & (\minaccs(P/s) = \{ \varnothing \})   \&   \big( \epass \then \Stop \big)
%\label{eq:utb}
%\\ & & \extchoice \nonumber
%\\ & & (\#s < p) \& \big(\Extchoice e:[P/s]^0 @ e \then U_T(p,s.e) \big)
%\label{eq:utc}
%\\ & & \extchoice \nonumber
%\\ & & (\#s = p) \& \big( \epass\then \Stop  \big)
%\label{eq:utd}
%\end{eqnarray}

\begin{eqnarray}
U_T(j) & = & U_T(j,0,\ii n)
\\
U_T(j,k,n) & = & \big(  e:(\Sigma - [n]^0)   \then \efail\then \Stop \big)
\label{eq:uta}
\\ & & \extchoice \nonumber
\\ & & (\minaccs(n) = \{ \varnothing \})   \&   \big( \epass \then \Stop \big)
\label{eq:utb}
\\ & & \extchoice \nonumber
\\ & & (k < j) \& \big( e:[P/s]^0   \then U_T(j,k+1,t(n,e)) \big)
\label{eq:utc}
\\ & & \extchoice \nonumber
\\ & & (k = j) \& \big( \epass\then \Stop  \big)
\label{eq:utd}
\end{eqnarray}
%
It is easy to see that the tests $U_T(j)$ satisfy the properties
\begin{eqnarray}
\label{eq:ifpaT}
  &  & U_T(j)/s = U_T(j,\#s,G(P)/s)
\\
\label{eq:ifpbT}
e\not\in [P/s]^0 & \implies & U_T(j)/s.e = (\efail\then\Stop)
\end{eqnarray}
proven in Lemma~\ref{lemma:ufproperties} for $U_F(j)$ for traces
$s\in\trc(P)$ with $\#s \le j$.

%The difference between adaptive tests $U_T(p)$ for trace refinement and
%$U_F(p)$ for failures refinement consists in the fact that the former do not
%``probe'' the SUT with respect to minimal sets of events to be accepted
%without blocking.

% ==========================================================================
Since the test $U_T(j)$ never blocks any event of an SUT process $Q$ before
terminating, the pass criterion, defined below, can be based on
trace instead of failures refinement as required in (\ref{eq:passF}).
%
\begin{equation}
\label{eq:passT}
Q\ \pass\ U_T(j) \defs (\epass\then\Stop) \lessdet_T (Q\parallel[\Sigma] U_T(j)) \hide \Sigma
\end{equation}
%
If the SUT process $Q$ deadlocks after a trace $s$, and in this case the
reference process $P$ is also in a state where deadlock is possible, this is
captured by the fact that $\minaccs(n) = \{ \varnothing\}$ for $n = G(P)/s$.
Therefore, branch (\ref{eq:utb}) of a test case execution state $U_T(j,k,n)$
with $\#s = k \le j$   can be entered and the  test execution terminates with
$\epass$. If, however, $Q$ blocks after a trace $s'$ and the reference
process satisfies $\minaccs(P/s') \neq\varnothing$, branch (\ref{eq:utb})
cannot be taken, and the test execution stops without producing $\epass$ or
$\efail$. In contrast to the test for failures refinement, this is
interpreted here as a successful test execution, because unexpected blocking
of the SUT does not violate the trace-refinement relation, as long as all
traces executed by the SUT are traces of the reference process. In
particular, if neither $\epass$ nor $\efail$ is ever produced, so that
$(Q\parallel[\Sigma] U_T(j)) \hide \Sigma = \Stop$, the test passes, because
$(\epass\then\Stop) \lessdet_T \Stop$ holds.



The existence of complete, finite test suites is expressed in analogy to
Theorem~\ref{th:failurestest}. A noteworthy difference is that the complete
suite for trace refinement just needs the single adaptive test case
$U_T(pq-1)$, while failures refinement requires the execution of $\{
U_F(0),\dots,U_F(pq-1)\}$. The reason is that $U_T(pq-1)$ identifies trace
errors for all traces up to length $pq$, while $U_F(pq-1)$ only probes for
erroneous acceptances at the end of each trace of length $(pq -1)$.
%
% -------------------------------------------------------------------------
\begin{theorem}\label{th:tracetest}
Let $P$ be a non-terminating, divergence-free CSP process over alphabet $\Sigma$ whose
normalised transition graph $G(P)$ has $p$ states. Define fault domain ${\cal
D}$ as the set of all non-terminating, divergence-free CSP processes over alphabet $\Sigma$,
whose transition graph has at most $q$ states with $q \ge p$. Then the test
suite
\[
\TS_T = \{ U_T(pq-1)   \}
\]
is complete with respect to ${\cal F} = (P,\lessdet_T,{\cal D})$.
\xbox
\end{theorem}
% -------------------------------------------------------------------------
%\begin{proof}
%The theorem follows directly
%from Step~1 in the proof of Lemma~\ref{lemma:mainfsound} and
%Case~1 in the proof of Lemma~\ref{lemma:mainfexhaustive}.
%\xbox
%\end{proof}
%%
%Examples are provided after our discussion of size of the test suites.
%
As for Theorem~\ref{th:failurestest}, the proof is structured in two lemmas,
the first ensuring soundness, and the second exhaustiveness.

% ===============================================================================
\begin{lemma}\label{lemma:mainfsoundtrace}
A test suite $\TS_T$ generated from a CSP process $P$, as specified in
Theorem~\ref{th:tracetest}, is passed by every CSP process $Q$ satisfying
$P\lessdet_T Q$.
\end{lemma}
\begin{proof}
Suppose that $P\lessdet_T Q$, so that $\trc(Q)\subseteq\trc(P)$, and assume
that $s\in\trc(Q)$ with $\#s < pq$. Since $s$ is also a trace of $P$, we can
conclude
$$U_T(pq-1)/s = U_T(pq-1,\#s,G(P)/s)$$ because of (\ref{eq:ifpaT}). Now $\trc(Q)\subseteq\trc(P)$
implies $[Q/s]^0\subseteq [P/s]^0 = [G(P)/s]^0$, so $U_T(pq-1,\#s,G(P)/s)$
cannot enter branch (\ref{eq:uta}) and produce a $\efail$ event when running
in parallel with $Q$ and synchronising over $\Sigma$. Therefore, only four
options are available for the test execution    $(Q\parallel[\Sigma]
U_T(j))/s$ to continue.

\medskip
\noindent {\bf Case~1.} $Q/s$ deadlocks and $\minaccs(G(P)/s) = \{
\varnothing \}$. In this case, the test $U_T(pq-1,\#s,G(P)/s)$ enters branch
(\ref{eq:utb}), and its execution stops after $\epass$.

\medskip
\noindent {\bf Case~2.} $Q/s$ deadlocks, but
$\minaccs(G(P)/s)\neq\{\varnothing\}$. In this case, the whole test execution
deadlocks, and this means that neither a $\epass$ nor a $\efail$ event is
produced, so the test execution is passed.

\medskip
\noindent {\bf Case~3.} $Q/s$ selects an event $e\in[Q/s]^0$ and $\#s <
pq-1$. In this case, the test $U_T(pq-1)$ in state $U_T(pq-1,\#s,G(P)/s)$ can
also engage in $e$ by entering branch (\ref{eq:utc}), and its execution
continues without producing a $\epass$ or a $\efail$ event.

\medskip
\noindent {\bf Case~4.} $\#s = pq-1$ holds. In this case,
$U_T(pq-1,\#s,G(P)/s)$ can only enter branch (\ref{eq:utd}), and the test
execution stops after $\epass$.

This case analysis shows that every   execution of $(Q\parallel[\Sigma] U_T(j))$
either stops after $\epass$ or produces neither $\epass$ nor $\efail$. This
proves that $Q$ passes test $U_T(pq-1)$ according to the pass  criterion (\ref{eq:passT}).
\xbox
\end{proof}
% ===============================================================================

\begin{lemma}\label{lemma:mainfexhaustivetrace}
A test suite $\TS_T$ specified as in Theorem~\ref{th:tracetest} is
exhaustive for the fault model specified there.
\end{lemma}
\begin{proof}
As before in the proofs for failures testing, we construct the product graph
$G=G(P)\times G(Q)$ and recall that every trace $s\in\trc(P)\cap\trc(Q)$ is
associated with a path through $G$ labelled with the same events as $s$, such
that $G/s = (G(P)/s,G(Q)/s)$. Furthermore, we recall from
Lemma~\ref{lemma:extendV} that graph state $(G(P)/s,G(Q)/s)$ can always be
reached by a trace $u$ of length less or equal $pq-1$, where  the order of
$G(P)$ is $p$ and that of $G(Q)$ is $q$.

Suppose that $P\not\lessdet_T Q$. Since the empty trace is a trace of every
process, there exists a trace $s\in \trc(Q)\cap\trc(P)$ and an event $e\in
[Q/s]^0$ such that $e\not\in [P/s]^0$. Let $u\in \trc(Q)\cap\trc(P)$ be a
trace with $\#u < pq$ and $G/u = (G(P)/s,G(Q)/s)$. Then
$$
U_T(pq-1)/u = U_T(pq-1,\#u,G(P)/s).
$$
By assumption, $e\in (\Sigma -[P/s]^0) = (\Sigma - [G(P)/s]^0)$. Since
$G(Q)/u = G(Q)/s$, $Q/u$ can engage into $e$. Then $U_T(pq-1,\#u,G(P)/s)$
will enter branch (\ref{eq:uta}), and the test execution stops after having
produced $\efail$. This proves that $Q$  fails test $U_T(pq-1)$. \xbox
\end{proof}
%
Having established completeness of our test suites, we now consider the
complexity of a testing technique that uses them.
