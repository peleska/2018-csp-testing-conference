% ==============================================================================
\section{Discussion and Conclusions}
\label{sec:conc}
% ==============================================================================

% ==============================================================================
\paragraph{Further Reductions of the Test Effort} As shown in
Theorem~\ref{th:maxtracelen}, the maximal length $pq$ of traces to be tested
for either  failures refinement or traces refinement cannot be further
reduced. It is noteworthy, however, that when testing FSMs for equivalence,
considerably shorter traces can be used. From the classical results published
in~\cite{chow:wmethod,vasilevskii1973}, for example, it follows that the
maximal trace length to be executed is less or equal to $3p -q$, which is
considerably smaller than $pq$ for $p,q\ge 3$. As a consequence, the
investigation of complete test suites establishing failures or trace
equivalence is of considerable interest and will be discussed in a future
paper.


It is also known from FSM testing that it is not necessary to check {\it all}
traces up to length $pq$ when testing for reduction of FSMs (which
corresponds to trace refinement). Notable complete strategies supporting this
fact have been presented, for example,
in~\cite{hierons_testing_2004,DBLP:conf/forte/DorofeevaEY05,petrenko_testing_2011,simao_reducing_2012}.
From~\cite{Huang2017} it is known that complete FSM testing theories can be
translated to other formalisms, such as Extended Finite State Machines,
Kripke Structures, or CSP, resulting in likewise complete test strategies for
the latter. We intend to study translations of several promising FSM
strategies to CSP. These will effectively reduce the upper bound for the
number of test executions to be performed, which  has been shown to be of the
order $O\big(\card{\Sigma}^{pq-1}\big)$ for our traces-refinement tests in
Section~\ref{sec:complexity}.  The bound $\binom{n}{\lfloor n/2\rfloor}$ for
the number of sets to be used in probing the SUT for illegal deadlocks,
however, cannot be further reduced, as established in
Lemma~\ref{lemma:hseta}.

% ==============================================================================
\paragraph{Adaptive Test Cases} The tests suggested
in~\cite{Hennessy:1988:ATP:50497,DBLP:conf/icfem/CavalcantiG07} were
\emph{preset} in the sense that the trace to be executed was pre-defined for
each test. As a consequence, the authors
of~\cite{DBLP:conf/icfem/CavalcantiG07} introduced \emph{inconclusive} as a
third test result, applicable to the situations where the intended trace of
the execution was blocked, due to legal, but nondeterministic behaviour of
the SUT. We consider this as a disadvantage, since, when aiming at executing
a specific trace $s$ before being able to check the test objective---for
example, the absence of deadlocks when offering a hitting set $H$ of the
minimal acceptances---it may take several tries until the full trace $s$ is
accepted by the SUT. Considering the complete testing assumption described in
Section~\ref{sec:finitecompletefails}, it may even take $c^{\#s}$ tries to
reach the end of the trace $s$, if the SUT can legally block every event of
$s$ due to nondeterminism, so that $c$ trials are required for each event is
accepted.

Those authors, later, in the context of a richer algebra based on CSP, have
considered a framework similar to adaptive testing~\cite{CG15}.  They have,
however, considered only traces refinement, and have not studied complexity.

%\fixme{From the proofs of the main theorems, and what you say below, I
%thought you did have verdicts coming from deadlock, and you simply chose not
%to mark it with an inc event.}
In  contrast to that, our test cases specified in
Section~\ref{sec:finitecompletefails} and \ref{sec:finitecomplete} are
adaptive. This has the advantage that test executions $(Q\parallel[\Sigma]
U_F(j))$ for failures refinement may only stop early with $\epass$ after
traces $s$ satisfying $\minhits(P/s) = \varnothing$, and may deadlock (a)
after a trace $s$ where the SUT illegally deadlocks (so $\minhits(P/s) \neq
\varnothing$ for the reference process, but $\minhits(Q/s) = \varnothing$ for
the implementation $Q$), or (b) in the final step when---just as in the
corresponding test cases specified in~\cite{DBLP:conf/icfem/CavalcantiG07}---
hitting sets $H$ are offered to the SUT and it refuses their acceptance. In
both situations (a) and (b) the test fails. As a consequence, far less test
repetitions are required according to the complete testing assumption than
for the successful execution of all test cases specified
in~\cite{DBLP:conf/icfem/CavalcantiG07}.

Another distinction of our failures test cases $U_F(j)$ to the tests
specified in~\cite{DBLP:conf/icfem/CavalcantiG07} consists in the fact that
the former test both traces refinement and the $conf$ conformance relation in
one go, whereas the latter use separate test suites to establish these two
correctness conditions. Again, we consider the structure of the  test cases
$U_F(j)$ as advantageous, since, when checking acceptance of a hitting set
$H$ after a trace $s$ for $c$ times according to the complete testing
assumption, any acceptance of an illegal event $e\in\Sigma - [P/s]^0$ should
also be revealed within these $c$ tries.


%
%never block before the final step specified
%by branch (\ref{eq:ufd}), and so we do not need inconclusive test results. It
%should be noted, however, that it is necessary for our test verdicts to
%recognise also deadlocks in the final test step and interpret them as FAIL,
%as described in Section~\ref{sec:finitecompletefails}. In practice, this is
%realised by adding a timeout event to the testing environment which indicates
%deadlock situations. For real-time systems, this is an accepted technique,
%because the SUT has to respond within a pre-defined latency interval,
%otherwise its behaviour is considered to be blocked and regarded as a
%failure. The tests executions $(Q\parallel[\Sigma] U_T(j))$ \fixme{Missing
%hiding.} \fxnote{jp: Here, the hiding operator is not necessary: the possible
%test executions are really the traces of $Q$ parallel $U_T(j)$. The hiding is
%only needed to specify the verdict.}for trace refinement never block at all
%before stopping after the verdict $\epass$ or $\efail$, unless the SUT
%process $Q$ has an unexpected deadlock state. Recall from the explanations
%given in Section~\ref{sec:finitecomplete} that the blocking situation also
%leads to passing the test, because $(\epass\then \Stop)\lessdet_T
%(Q\parallel[\Sigma] U_T(j))\hide\Sigma$ still holds if $(Q\parallel[\Sigma]
%U_T(j))\hide\Sigma$ only produces the empty trace.

%The adaptive behaviour of both our test case types $U_F(j)$ and $U_T(j)$,
%however, induces the obligation to check that {\it all} possible executions
%have been performed before the test can be considered as passed. Typically,
%it is therefore assumed that a \emph{complete testing
%assumption}~\cite{hierons_testing_2004} holds, which means that every
%possible behaviour of the SUT is performed after a finite number of test
%executions. In practice, this is realised by executing each test several
%times, recording the traces that have been performed, and using hardware or
%software coverage analysers to determine whether all possible test execution
%behaviours of the SUT have been observed. Therefore, adaptive test cases come
%at the price of having to apply some grey-box testing techniques enabling us
%to decide whether all SUT behaviours have been observed.

% ==============================================================================
\paragraph{Fault Domains} As already mentioned, the work
in~\cite{DBLP:conf/pts/CavalcantiS17} defines a fault domain as the set of
processes that refine a given CSP process.  In that context, only testing for
traces refinement is considered, and the complete test suites may not be
finite. So, the work presented here goes well beyond what is achieved there,
but is restricted to finite and nonterminating reference processes. In
addition, \cite{DBLP:conf/pts/CavalcantiS17} presents an algorithm for test
generation that can be adapted to consider additional selection and
termination criteria, like, for example, the length of the traces used to
construct tests. It would be possible, for instance, to use the bound
indicated here. Moreover, specifying a fault domain as a CSP process allows
us to model domain-specific knowledge using CSP. For example, if an
initialisation component defined by a process $I$ can be regarded as correct
without further testing, we can use $I; RUN$ as a fault domain, to indicate
that any SUT of interest implements $I$ correctly, but afterwards has an
arbitrary behaviour specified by $RUN$.

% ==============================================================================
\paragraph{Implications for CSP Model Checking} As explained in the previous
sections, passing a test is characterised by the failures-refinement check
$(\epass\then\Stop) \lessdet_F (Q\parallel[\Sigma] U_F(j)) \hide \Sigma$ for
failures testing. If the SUT $Q$ is not a programmed piece of software or an
integrated hardware or software system, but just another CSP process
specification, it is of course possible to verify the pass criterion using
the FDR4 model checker. For checking the refinement relation $P\lessdet_F Q$,
the pass criterion has to be verified for $j=0,\dots,pq-1$, where $p$ and $q$
indicate the number of nodes in $P$'s transition graph and the maximal number
of nodes in $Q$'s graph, respectively (Theorem~\ref{th:failurestest}). Since
the test cases $U_F(j)$ have such a simple structure, it is an interesting
question for further research whether checking $(\epass\then\Stop) \lessdet_F
(Q\parallel[\Sigma] U_F(j)) \hide \Sigma$ for $j=0,\dots,pq-1$ can be faster
than directly checking $P\lessdet_F Q$, as one would do in the usual approach
with FDR4. This is of particular interest, since the checks could be
parallelised on several CPUs. Alternatively it is interesting to investigate
whether the check of\footnote{We are grateful to Bill Roscoe for suggesting
this option.}
\[
(\epass\then\Stop) \lessdet_F (Q\parallel[\Sigma] \Intchoice_{j=0}^{pq-1} U_F(j)) \hide \Sigma
\]
may perform better than the check of $P\lessdet_F Q$, since the former allows for
other optimisations in the model checker.

For a large implementation process $Q$, it may be too time consuming to
generate its normalised transition graph, so that its number $q$ of nodes is
unknown. In such a case, our testing approach based on fault domains may
still be used as efficient bug finders:~use the number of nodes of the
normalised transition graph of the reference process as the initial value for
$q$ and increment $q$ from there, as long as each increment reveals new
errors.

% ==============================================================================
\paragraph{Conclusion}
In this paper, we have introduced finite complete testing strategies for
model-based testing against finite, non-terminating CSP reference models. The
strategies are applicable to the conformance relations failures refinement
and traces refinement. The underlying fault domains have been defined as the
sets of CSP processes whose normalised transition graphs do not have more
than a given number of additional nodes, when compared to the transition
graph of the reference process. For these domains, finite complete test
suites are available. We have shown for the strategy to check failures
refinement that the way of probing the SUT for illegal deadlocks in our test
cases is optimal, so that it is not possible to guarantee exhaustiveness with
fewer probes. Moreover, the maximal length of the test traces cannot be
reduced without losing the test suite's completeness property; this holds
both for traces and failures refinement.

% ==============================================================================

\paragraph{Acknowledgements}
The authors would like to thank Bill Roscoe and Thomas Gibson-Robinson for
their advice on using the FDR4 model checker and for very helpful discussions
concerning the potential implications of this paper in the field of model
checking. We are also grateful to Li-Da Tong from National Sun Yat-sen
University, Taiwan, for suggesting the applicability of Sperner's Theorem in
the context of the work presented here. Moreover, we thank Adenilso Simao for
several helpful suggestions. The work of Ana Cavalcanti is funded by the
Royal Academy of Engineering and UK EPSRC Grant EP/R025134/1.
