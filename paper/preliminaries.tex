% =========================================================================
\section{Preliminaries}
\label{section:preliminaries}
% =========================================================================



% =========================================================================
\subsection{CSP and Refinement}

% =========================================================================

\subsubsection*{Communicating Sequential Processes} @todo
\fxwarning{alcc: I can make this small contribution.}

Throughout this paper, the alphabet of CSP processes $P$ is denoted by $\Sigma$.
Since we are only considering ``classical'' CSP processes, the alphabet is always finite.

\begin{itemize}
  \item traces and failures model
  \item refinement and equivalence
\end{itemize}

The FDR tool~\cite{fdr} supports model checking and semantic analyses of CSP
processes.

% =========================================================================
\subsubsection*{Normalised Transition Graphs for CSP Processes}
\label{sec:ntg}

As shown in~\cite{Roscoe:1994:CME:197600}, any finite-state CSP process $P$
can be represented by a \emph{normalised transition graph}
$$
G(P) = ( N, \ii n, \Sigma, t : N\times\Sigma \pfun N, r : N \fun \mathbb{P}\mathbb{P}(\Sigma)),
$$
with nodes $N$, initial node $\ii n\in N$, and process alphabet $\Sigma$. The
partial \emph{transition function} $t$ maps a node $n$ and an event
$e\in\Sigma$ to its successor node $t(n,e)$. If $(n,e)$ is in the domain of
$t$, then there is a transition from $n$ with label $e$. Normalisation of
$G(P)$ is reflected by the fact that $t$ is a function.

A finite sequence of events $s\in\Sigma^*$ is a \emph{trace} of $P$, if there
is a path through $G(P)$ starting  at $\ii n$ whose edge labels coincide with
$s$. The set of traces of $P$ is denoted by $\trc(P)$. If $s\in\trc(P)$, then
the process corresponding to $P$ after having executed $s$ is denoted by
$P/s$. Since $G(P)$ is normalised, there is a unique node reached by applying
the events from $s$ one by one, starting in $\ii n$. Therefore, $G(P)/s$  is
also well defined. By $[n]^0$ we denote the \emph{initials} of $n$:~the set
of events occurring as labels in any outgoing transitions.
$$
[n]^0 = \{ e\in\Sigma~|~(n,e)\in\dom~t \}
$$
We also use this notation for CSP processes:~$[P]^0$ is the set of events $P$
may engage into, in other words, the initials of $P$ after the empty trace of
events, that is, $initials(P/\langle\rangle)$ as defined
in~\cite{Roscoe2010}, for example.

The total function $r$ maps each node $n$ to its \emph{refulsals} $r(n) =
\refs(n)$. Each element of $r(n)$ is a set of events that the CSP process $P$
might refuse to engage into, when in a process state corresponding to $n$.
 If $P/s$ is deterministic, its refusals coincide with the set of
subsets of $\Sigma - [P/s]^0$, including the empty set.

For finite CSP processes, since the refusals of each process state are
subset-closed~\cite{Hoare:1985:CSP:3921,Roscoe2010}, $\refs(P/s)$ can be
constructed from the set of \emph{maximal refusals}
$\maxrefs(P/s)\subseteq\refs(P/s)$. More formally, $\maxrefs(P/s)$ is defined
as follows.
%
\begin{equation}
\maxrefs(P/s) = \{ R \in\refs(P/s)~|~\forall R'\in \refs(P/s) - \{ R\}: R \not\subseteq R'\}
\end{equation}
%
Conversely, with the maximal refusals $\maxrefs(P/s)$ at hand, we can
reconstruct the refusals $\refs(P/s)$ by subset-closure as follows.
%
\begin{equation}
\refs(P/s) = \{ R'\in\power(\Sigma)~|~\exists R\in \maxrefs(P/s): R'\subseteq R \}.
\end{equation}
%
The cardinality of $\maxrefs(P/s)$ reflects the degree of
nondeterminism that is present in process state $P/s$: the more maximal refusal sets
contained in  $\maxrefs(P/s)$, the more nondeterministic is the behaviour in
state $P/s$. Deterministic process states $P/s$ have exactly the one maximal refusal
$\Sigma-[P/s]^0$.

To see that this approach works only for finite CSP processes, we consider
the example where $\Sigma$ is infinite. In this case,
$\maxrefs(\Stop/\langle\rangle)$ is empty, and so we cannot use this set to
calculate the refusals of $\Stop$, that is, $\refs(\Stop/\langle\rangle)$ as
defined above. As with refusals, we also use the transition graph-oriented
notation $\maxrefs(n) \subseteq r(n)$ to denote the maximal refusals
associated with graph state $n$: if $n$ is the state reached in the
transition graph by following the edge labels in trace $s$, then $\maxrefs(n)
= \maxrefs(P/s)$.

Well-formed normalised transition graphs must not refuse an event of the
initials of a state in {\it every} refusal applicable in this state; more
formally,
%
\begin{equation}
\label{eq:wellformedg}
\forall n\in N, e\in\Sigma: (n,e)\in\dom~t \Rightarrow
\exists R\in \maxrefs(n): e\not\in R
\end{equation}
%
By construction, normalised transition graphs reflect the \emph{failures
semantics} of finite-state CSP processes:~the traces $s$ of a process are
defined by the sequences of transitions associated with paths through its
graph, starting at $\ii n$. The pairs $(s,R)$ with $s\in\trc(P)$ and $R\in
r(G(P)/s)$ represent the failures of $P$.

When investigating  tests for failures refinement, the notion of
\emph{acceptances}~\cite{Hennessy:1988:ATP:50497}, which is dual to refusals,
is also useful. An acceptance set of $P/s$ is a subset of the initials
$[P/s]^0$, i.e., a subset of events labelling  outgoing transitions of
$G(P)/s$. If the behaviour of  $P/s$ is deterministic, its only acceptance
equals $[P/s]^0$, because $P/s$ never refuses any of the events contained in
this set. If $P/s$ is nondeterministic, it internally chooses one of its
\emph{minimal acceptance sets} $A$ and never refuses any event in $A$, while
refusing the events in $\Sigma-A$. The acceptances of $P/s$ are denoted by
$\accs(P/s)$, and the minimal acceptances by $\minaccs(P/s)$. They satisfy
the following properties.
%
\begin{eqnarray}
A\in \minaccs(P/s) & \Leftrightarrow & \exists R\in\maxrefs(P/s) \wedge A = \Sigma-R
\label{eq:accref}
\\
\bigcup \{ A~|~A\in \accs(P/s)\} & = & [P/s]^0
\label{eq:accinitials}
\\
 X\in\accs(P/s) &\Leftrightarrow & A\in \minaccs(P/s) \wedge A\subseteq X \subseteq [P/s]^0
 \label{eq:accsubset}
\end{eqnarray}
%
Note that (\ref{eq:accref}) can be regarded as a definition of minimal
acceptances by means of maximal refusals. In this case,
(\ref{eq:accinitials}) and (\ref{eq:accsubset}) are consequences of this
definition and the fact that refusals are subset-closed. \fixme{alcc: are you
sure? (4) does not talk about Acc. Is this important?}

Every node of a normalised transition graph can alternatively be labelled
with their minimal acceptances, and this information is equivalent to that
contained in the maximal refusals. This is the representation used by FDR.

% .....................................................................................
 \begin{figure}[t]
   %%\hspace*{-40mm}
   \begin{center}
     \includegraphics[width=\textwidth]{q0.pdf}
   \end{center}
   %%\vspace*{-10mm}
   \caption{Normalised transition graph of CSP process $P$ from Example~\ref{ex:a}.}
   \label{fig:tga}
 \end{figure}
% .......................................................................................

\begin{example}\label{ex:a}
Consider the CSP process $P$ defined below, and $\Sigma = \{a,b,c\}$.
\begin{eqnarray*}
P & = & a \then (Q\intchoice R)
\\
Q & = & a \then P \extchoice c \then P
\\
R & = & b \then P \extchoice c \then R
\end{eqnarray*}
Its transition graph $G(P)$ is shown in Fig.~\ref{fig:tga}. Process state
$P/\langle\rangle$ is represented there as Node\_0, with $\{ a\}$ as the
only minimal acceptance, since $a$ can never be refused, and no other events are
accepted. Having engaged into $a$, the transition emanating from Node\_0
leads to Node\_2 representing  the process state $P/a = Q\intchoice R$. The
internal choice operator induces several minimal acceptances derived from $Q$ and
$R$. Since these processes accept their initial events in external choice,
$Q\intchoice R$ induces minimal acceptance sets $\{a,c\}$ and
$\{b,c\}$. Note that event $c$ can never be refused, since it is contained in each minimal acceptance set.

Having engaged into $c$, the next process state is represented by Node\_1.
Due to normalisation, there is only a single transition satisfying
$t(\text{Node\_2},c) = \text{Node\_1}$. This transition, however, can have
been caused by either $Q$ or $R$ engaging into $c$, so Node\_1 corresponds to
process state $Q/c \intchoice R/c = P \intchoice R$. This is reflected by the
two minimal acceptance sets labelling Node\_1.
Similar considerations explain the other nodes and transitions in
Fig.~\ref{fig:tga}.

Note that the node names including their number suffixes are generated by the
FDR tool. The numbering is generated during the normalisation procedure. So,
the node numbers do not reflect the distance from the initial node Node\_0.
\xbox
\end{example}

Refinement relations between finite-state CSP processes $P, Q$ can be be
expressed by means of their normalised transition graphs in the following
way. \fixme{alcc: trc was defined for P, not G(P) and this lemma is used
later for trc(P). So, we need to be clearer as to what we mean.}
%
\begin{lemma}
  \label{lemma:tgtrcref}
  \begin{eqnarray}
  P \lessdet_T Q & \Leftrightarrow & \trc(G(Q)) \subseteq\trc(G(P))
  \label{eq:trcrefa}
  \\
  \label{eq:failrefa}
  P \lessdet_F Q & \Leftrightarrow & \trc(G(Q)) \subseteq\trc(G(P)) \wedge {} \nonumber
  \\ & & \forall s\in\trc(G(Q)), R_Q\in\maxrefs(G(Q)/s):  \nonumber
  \\ & & \tabd
  \exists R_P\in\maxrefs(G(P)/s): R_Q\subseteq R_P
  \\
  \label{eq:failrefb}
  P \lessdet_F Q & \Leftrightarrow & \trc(G(Q)) \subseteq\trc(G(P)) \wedge {} \nonumber
  \\ & & \forall s\in\trc(G(Q)), A_Q\in\minaccs(G(Q)/s): \nonumber
  \\ & & \tabd
  \exists A_P\in\minaccs(G(P)/s): A_P\subseteq A_Q
  \end{eqnarray}
  \xbox
\end{lemma}
%
For proving our main theorems, it is necessary to consider the \emph{product}
of normalised transition graphs. We need this only for the investigation of
corresponding traces in reference processes and processes for SUTs.
Therefore, the labelling of nodes with maximal refusals or minimal
acceptances are disregarded in the product construction. Consider two
normalised transition graphs
\[
G_i = ( N_i, \ii n_i, \Sigma, t_i : N_i\times\Sigma \pfun N_i, r_i : N_i \fun \mathbb{P}\mathbb{P}(\Sigma)),\qquad i = 1,2,
\]
over the same alphabet $\Sigma$. Their product is defined by
\begin{eqnarray}
G_1\times G_2 & = & (N_1\times N_2,(\ii n_1,\ii n_2), t:(N_1\times N_2)\times\Sigma\pfun (N_1\times N_2))
\\
\dom~t & = & \{ ((n_1,n_2),e)\in (N_1\times N_2)\times\Sigma~|   \nonumber
\\ & & \tabc
(n_1,e)\in\dom~t_1\wedge
(n_2,e) \in\dom~t_2    \}
\\
t((n_1,n_2),e) & = & (t_1(n_1,e),t_2(n_2,e))\ \text{for $((n_1,n_2),e)\in\dom~t$}
\end{eqnarray}


% =========================================================================

\subsubsection*{Tool Considerations}
FDR provides an API~\cite{fdrmanual} that can be used to construct normalised
transition graphs for CSP processes. The FDR graph nodes are labelled by
\emph{minimal acceptances} instead of maximal refusals as described above.
Since such a minimal acceptance set is the complement of a maximal refusal,
the function $r$ mapping states to their refusals can be implemented by
taking the complements of all minimal acceptances and then building all their
subsets. For practical applications, the subset closure is never constructed
in an explicit way; instead, sets are checked with respect to containment in
a maximal refusal.

@todo

%The maximal refusals in each process state $P/s$
% are the complements of
%the minimal acceptances of the node $n$
%corresponding to $P/s$. As a consequences, all failures
%of $P$ are represented by some $(s,R)$, where $s$ is an initialised path through the transition graph and $R\subseteq (\Sigma-A)$ for some minimal acceptance $A\in ac(n)$,
%such that $n$ is the node corresponding to $P/s$.

% =========================================================================
\subsection{Test Cases and Complete Test Suites}
\label{sec:cspcompletedef}
% =========================================================================

@todo

% =========================================================================
\subsection{Minimal Hitting Sets}
\label{sec:hit}
% =========================================================================

The main idea of the underlying test strategy for failures refinement can be
based on solving a \emph{hitting set problem}. Given a finite collection of
finite sets $C = \{ A_1,\dots,A_n\}$, such that each $A_i$ is a subset of a
universe $\Sigma$, a \emph{hitting set} $H\subseteq\Sigma$ is a set
satisfying the following property.
%
\begin{equation}
  \label{eq:hit}
  \forall A\in C: H\cap A \neq\varnothing.
\end{equation}
%
A \emph{minimal hitting set} is a hitting set that cannot be further reduced
without losing the characteristic property (\ref{eq:hit}). The problem of
determining minimal hitting sets is known to be
NP-hard~\cite{Book1975-BOOKRM}, but we will see below that it reduces the
effort of testing for failures refinement from a factor of $2^\Sigma$ to a
factor that equals the number of minimal hitting sets.

For testing, the following lemma about hitting sets are required.

\begin{lemma}
\label{lemma:hseta}
Let $P, Q$ be two finite-state CSP processes satisfying $P\lessdet_T Q$.
For each $s\in\trc(P)$,
let $\text{minHit}(P/s)$ denote the
collection of all minimal hitting sets of $\minaccs(P/s)$.
Then the following statements are equivalent.
\begin{enumerate}
\item $P\lessdet_F Q$ \fixme{Note that if traces refinement is already
    given, this is conf.}
\item For all $s\in\trc(P)\cap \trc(Q)$ and $H \in  \text{minHit}(P/s)$, $H$ is
a (not necessariliy minimal) hitting set of $\minaccs(Q/s)$.
\end{enumerate}
\end{lemma}
\begin{proof}
For showing ``$1 \Rightarrow 2$'', assume that $P\lessdet_F Q$ and suppose
that $s\in\trc(P)\cap \trc(Q)$. Lemma~\ref{lemma:tgtrcref},
(\ref{eq:failrefb}), states that
\[
\forall A_Q\in\minaccs(G(Q)/s):
\exists A_P\in\minaccs(G(P)/s): A_P\subseteq A_Q
\]
Therefore, $H \in  \text{minHit}(P/s)$ not only implies $H\cap
A_P\neq\varnothing$ for all minimal acceptances $A_P$, but also $H\cap
A_Q\neq\varnothing$ for every minimal acceptance $A_Q$, because $A_P\subseteq
A_Q$ for at least one $A_P$. As a consequence, each $H \in
\text{minHit}(P/s)$ is also a hitting set for $\minaccs(G(Q)/s)$.

To prove ``$2 \Rightarrow 1$'', assume $P\lessdet_T Q$, but $P\not\lessdet_F
Q$. According to Lemma~\ref{lemma:tgtrcref}, (\ref{eq:failrefb}), there
exists $s\in\trc(P)\cap \trc(Q)$ such that
\[
\exists A_Q\in\minaccs(G(Q)/s): \forall A_P\in\minaccs(G(P)/s): A_P\not\subseteq A_Q
\qquad (*)
\]
Let $A$ be such a set $A_Q$ fulfilling (*).
Define
\[
\overline H = \bigcup\{ A_P \setminus A~|~A_P\in\minaccs(G(P)/s) \}.
\]
Since $A_P \setminus A \neq\varnothing$ for all $A_P$ because of (*),
$\overline H$ is a hitting set of $\minaccs(G(P)/s)$ which has an  empty
intersection with $A_Q$. \fixme{alcc: minaccs is used for P above.}
Minimising $\overline H$ induces the existence of a minimal hitting set $H\in
\text{minHit}(P/s)$ which is {\it not} a hitting set of $\minaccs(G(Q)/s)$, a
contradiction to Assumption~2. This completes the proof of the lemma. \xbox
\end{proof}
%
This result is used in the sequel in Section~\ref{}.

%\begin{lemma}
%\label{lemma:hsetb}
%
%\end{lemma}
%\begin{proof}
%asd
%\end{proof}
%
