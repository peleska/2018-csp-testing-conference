% =========================================================================
\section{Preliminaries}
\label{section:preliminaries}
% =========================================================================

In this section, we present the background material relevant to our work.

% =========================================================================
\subsection{Complete Test Suites}
\label{sec:fsmfm}
% =========================================================================

We use the term \emph{signature} to denote a collection of comparable models
represented in an arbitrary formalism. In this article, signatures represent
sets of finite state machines over fixed input and output alphabets, or CSP
processes with finite state, represented by their normalised transition
graphs (see Section~\ref{sec:ntg}).

Given a signature $Sig$  of models, a  \emph{fault model} ${\cal F} =
(M,\le,Dom)$ specifies a \emph{reference model} $M\in Sig$, a
\emph{conformance relation} $\le\ \subseteq Sig\times Sig$ between models,
and a \emph{fault domain} $Dom\subseteq Sig$. This terminology
follows~\cite{gotzhein_fault_1996}, where fault models have been originally
introduced in the context of testing of finite state machines. Note that
fault domains may contain both models conforming to the reference model and
models violating the conformance relation. Note further that the reference
model $M$ is not necessarily a member of the fault domain, although a model
of the SUT behaviour is. For example, $M$ could be nondeterministic, while
only deterministic implementation behaviours might be considered in the fault
domain.
%By $F(Sig,\le)$ we denote the set of all fault models ${\cal F}$
%defined for
%signature $Sig$ and conformance relation $\le$.

Let $\tc(Sig)$ denote the set of all \emph{test cases} applicable to elements
of $Sig$. This abstract notion of test case requires only the existence of a
relation $\pass\subseteq Sig\times \tc(Sig)$. For $(M,U)\in\pass$, the infix
notation $M\ \pass\ U$ is used, and interpreted as {\it `model $M$ passes the
test case $U$'}. If $(M,U)\not\in\pass$ holds, this is abbreviated by $M\
\tfail\ U$. Our specific notion of test cases for CSP models is elaborated in
Section~\ref{sec:csptc}.

A \emph{test suite} $\TS \subseteq \tc(Sig)$ denotes  a set of test cases. A
model $M$ \emph{passes the test suite} $\TS$, also written as $M\ \pass\
\TS$, if, and only if, $M\ \pass\ U$ for all $U\in \TS$. A test suite $\TS$
is called \emph{complete} for fault model ${\cal F} = (M,\le,Dom)$, if, and
only if, the following properties hold.
\begin{enumerate}
\item If a member $M'$ of the fault domain  conforms to the reference model $M$,
it passes the test suite, that is,
$$
\forall M'\in Dom: M'\le M \Rightarrow M'\ \pass\ \TS
$$
This property is usually called \emph{soundness} of the test suite.

\item If a member of the fault domain passes the test suite, it conforms to the reference model, that is,
$$
\forall M'\in Dom: M'\ \pass\ \TS \Rightarrow M'\le M
$$
This property is usually called \emph{exhaustiveness}.
\end{enumerate}
A test suite $\TS$ is \emph{finite} if it contains finitely many test cases and every test
case $U\in\TS$ is finite in the sense that it terminates after a finite number of steps.
It is trivial to see that, if $\TS$ is complete  for   ${\cal F} = (M,\le,Dom)$
and $Dom'\subseteq Dom$, then $\TS$ is also complete for ${\cal F}' = (M,\le,Dom')$.

 
% =========================================================================
\subsection{CSP and Refinement}

% =========================================================================

\subsubsection*{Communicating Sequential Processes} @todo
\fxwarning{alcc: I can make this small contribution.}

% =========================================================================
\subsubsection*{Normalised Transition Graphs for CSP Processes}
\label{sec:ntg}

As shown in~\cite{Roscoe:1994:CME:197600}, any finite-state CSP process $P$
can be represented by a \emph{normalised transition graph}
$$
G(P) = ( N, \ii n, \Sigma, t : N\times\Sigma \pfun N, r : N \fun \mathbb{P}\mathbb{P}(\Sigma)),
$$
with nodes $N$, initial node $\ii n\in N$, and process alphabet $\Sigma$. The
partial \emph{transition function} $t$ maps a node $n$ and an event
$e\in\Sigma$ to its successor node $t(n,e)$, if, and only if, $(n,e)$ is in
the domain of $t$, that is, there is a transition from $n$ with label $e$.
Normalisation of $G(P)$ is reflected by the fact that $t$ is a function.

A finite sequence of events $s\in\Sigma^*$ is a \emph{trace} of $P$, if there
is a path through $G(P)$ starting  at $\ii n$ whose edge labels coincide with
$s$. The set of traces of $P$ is denoted by $\trc(P)$. If $s\in\trc(P)$, then
the process corresponding to $P$ after having executed $s$ is denoted by
$P/s$. Since $G(P)$ is normalised, there is a unique node reached by applying
the events from $s$ one by one, starting in $\ii n$. Therefore, $G(P)/s$  is
also well defined.

By $[n]^0$ we denote the \emph{fan-out} of $n$:~the set of events occurring
as labels in any outgoing transitions.
$$
[n]^0 = \{ e\in\Sigma~|~(n,e)\in\dom~t \}
$$
We also use this notation for CSP processes:~$[P]^0$ is the set
of events $P$ may engage into, in other words, the initials of $P$ after the
empty trace of events, that is, $initials(P/\langle\rangle)$ as defined
in~\cite{Roscoe2010}.

The total function $r$ maps each node $n$ to its \emph{refulsals} $r(n) =
\refs(n)$. Each element of $r(n)$ is a set of events that the CSP process $P$
might refuse to engage into, when in a process state corresponding to $n$.
The number of refusal sets in $\refs(P/s)$ specifies the degree of
nondeterminism that is present in process state $P/s$: the more refusal sets
contained in  $\refs(P/s)$, the more nondeterministic is the behaviour in
state $P/s$. If $P/s$ is deterministic, its refusals coincide with the set of
subsets of $\Sigma - [P/s]^0$, including the empty set.

For finite CSP processes, since the refusals of each process state are
subset-closed~\cite{Hoare:1985:CSP:3921,Roscoe2010}, $\refs(P/s)$ can be
re-constructed by knowing the set of \emph{maximal refusals}
$\maxrefs(P/s)\subseteq\refs(P/s)$. More formally, the maximal refusals
$\maxrefs(P/s)$ are defined as
$$
\maxrefs(P/s) = \{ R \in\refs(P/s)~|~\forall R'\in \refs(P/s) - \{ R\}: R \not\subseteq R'\}
$$
Conversely, with the maximal refusals at hand, we can reconstruct the refusals by subset-closure:
$$
\refs(P/s) = \{ R'\in\power(\Sigma)~|~\exists R\in \maxrefs(P/s): R'\subseteq R \}.
$$
To see that this approach works only for finite CSP processes, consider the
example where $\Sigma$ is infinite. In this case,
$\maxrefs(STOP/\langle\rangle)$ is empty, and so we cannot use this set to
calculate the refusals of $STOP$, that is, $\refs(STOP/\langle\rangle)$ as
defined above. As with refusals, we also use the transition graph-oriented
notation $\maxrefs(n) \subseteq r(n)$ to denote the maximal refusals
associated with graph state $n$: if $n$ is the state reached in the
transition graph by following the edge labels in trace $s$, then $\maxrefs(n)
= \maxrefs(P/s)$.

Well-formed normalised transition graphs must not refuse an event of the
fan-out of a state in {\it every} refusal applicable in this state; more
formally,
\begin{equation}
\label{eq:wellformedg}
\forall n\in N, e\in\Sigma: (n,e)\in\dom~t \Rightarrow
\exists R\in \maxrefs(n): e\not\in R
\end{equation}
%The total function $a$ maps each node to its set of \emph{minimal acceptances}:
%if $n\in N$ corresponds to a deterministic process state of $P$, $ac(n)$ contains a single acceptance $A\subseteq \Sigma$, and every $e\in A$ is in one-one-correspondence with a transition $t(n,e)$. If $n$ corresponds to a nondeterministic process state, $ac(n)$ contains at least two acceptances $A_1, A_2, \dots, A_k$. This reflects the fact that in a nondeterministic state, $P$ must accept all events of one acceptance
%$A_i, i \in \{ 1,\dots,k\}$, but may refuse all events $e$ from
%$A_j \setminus A_i, j\neq i$.
%
%Each well-defined transition graph $G(P)$ fulfils the following condition. The union of all minimal acceptances in each node corresponds to the set of events labelling its outgoing transitions.
%\begin{equation}
%\label{eq:wellformedg}
%\forall n\in N: (n,e)\in\dom~t \Leftrightarrow e\in\bigcup ac(n)
%\end{equation}
%In this condition, $\dom~t$ denotes the domain of function $t$.
By construction, normalised transition graphs reflect the \emph{failures
semantics} of finite-state CSP processes:~the traces $s$ of a process are the
sequences of transition associated with paths through the graph,
starting at $\ii n$. The pairs $(s,R)$ with $s\in\trc(P)$ and $R\in
r(G(P)/s)$ represent the failures $\failure(P)$ of $P$.

% .....................................................................................
 \begin{figure}
 %%\hspace*{-40mm}
 \begin{center}
\includegraphics[width=\textwidth]{q0.pdf}
\end{center}
%%\vspace*{-10mm}
\caption{Normalised transition graph of CSP process $P$ from Example~\ref{ex:a}.}
 \label{fig:tga}
 \end{figure}
% .......................................................................................

\begin{example}{ex:a}
Consider the CSP process $P$ defined below, and that the set of events
$\Sigma$ is $\{a,b,c\}$.
\begin{eqnarray*}
P & = & a \then (Q\intchoice R)
\\
Q & = & a \then P \extchoice c \then P
\\
R & = & b \then P \extchoice c \then R
\end{eqnarray*}
Its transition graph $G(P)$ is shown in Fig.~\ref{fig:tga}. Process state
$P/\langle\rangle$ is represented there as Node\_0, with $\{ b,c\}$ as the
only maximal refusal, since $a$ can never be refused, and no other events are
accepted. Having engaged into $a$, the transition emanating from Node\_0
leads to Node\_2 representing  the process state $P/a = Q\intchoice R$. The
internal choice operator induces several refusal sets derived from $Q$ and
$R$. Since these processes accept their initial events in external choice,
$Q\intchoice R$ induces two maximal refusal sets $\{b\}$ and
$\{a\}$. Note that event $c$ can never be refused, since it is not a member
of any maximal refusal.

Having engaged into $c$, the next process state is represented by Node\_1.
Due to normalisation, there is only a single transition satisfying
$t(\text{Node\_2},c) = \text{Node\_1}$. This transition, however, can have
been caused by either $Q$ or $R$ engaging into $c$, so Node\_1 corresponds to
process state $Q/c \intchoice R/c = P \intchoice R$. This is reflected by the
two maximal refusals labelling Node\_1.

Similar considerations explain the other nodes and transitions in
Fig.~\ref{fig:tga}.

Note that the node names are generated by the FDR tool (see next paragraph).
The node numbering is generated by FDR during the normalisation procedure.
Therefore, the node numbers do not reflect the distance from the initial node
Node\_0.
%Node\_2 is a direct successor to the initial node Node\_0, while Node\_1 is a direct successor of Node\_2.
\end{example}

% =========================================================================

\subsubsection*{Tool Considerations}
The FDR tool~\cite{fdr} supports model checking and semantic analyses of CSP
processes. It provides an API~\cite{fdrmanual} that can be used to construct
normalised transition graphs for CSP processes.

The FDR graph nodes are labelled by \emph{minimal acceptances} instead of
maximal refusals as described above. Since such a minimal acceptance set is
the complement of a maximal refusal, the function $r$ mapping states to their
refusals can be implemented by creating the complements of all minimal
acceptances and then building all subsets of these complements. For practical
applications, the subset closure is never constructed in an explicit way;
instead, sets are checked with respect to containment in a maximal refusal.

%
%The maximal refusals in each process state $P/s$
% are the complements of
%the minimal acceptances of the node $n$
%corresponding to $P/s$. As a consequences, all failures
%of $P$ are represented by some $(s,R)$, where $s$ is an initialised path through the transition graph and $R\subseteq (\Sigma-A)$ for some minimal acceptance $A\in ac(n)$,
%such that $n$ is the node corresponding to $P/s$.

 