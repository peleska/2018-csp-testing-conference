% ==========================================================================
\section{Finite Complete Test Suites for CSP Trace Refinement}
\label{sec:finitecomplete}
% ==========================================================================

For establishing trace refinement, the following class of adaptive test cases
will be used for a given reference process $P$ and  integers $p \ge 0$.
Just as for the tests developed  in 
Section~\ref{sec:finitecompletefails} to verify failures refinement, 
the tests for trace refinement are
derived from the reference model's transition graph
$$
G(P) = ( N, \ii n, \Sigma, t : N\times\Sigma \pfun N, r : N \fun \mathbb{P}\mathbb{P}(\Sigma)).
$$
In contrast to the failures tests (\ref{eq:UFP}), however, we do not need to
check the SUT with respect to its acceptance of hitting sets. Therefore, these
do not occur in the specification of the test cases below, and we use the
condition $\minaccs(n) = \{ \varnothing \}$ instead of $\minhits(n) = \varnothing$
to indicate that minimal hitting sets need not to be calculated for generating
these tests from $G(P)$. From (\ref{eq:minhitminaccempty}) we know that both conditions
are equivalent.



%
%\begin{eqnarray}
%U_T(p) & = & U_T(p,\varepsilon)
%\\
%U_T(p,s) & = & \big(\Extchoice e:(\Sigma - [P/s]^0) @ e \then \efail\then \Stop \big)
%\label{eq:uta}
%\\ & & \extchoice \nonumber
%\\ & & (\minaccs(P/s) = \{ \varnothing \})   \&   \big( \epass \then \Stop \big)
%\label{eq:utb}
%\\ & & \extchoice \nonumber
%\\ & & (\#s < p) \& \big(\Extchoice e:[P/s]^0 @ e \then U_T(p,s.e) \big)
%\label{eq:utc}
%\\ & & \extchoice \nonumber
%\\ & & (\#s = p) \& \big( \epass\then \Stop  \big)
%\label{eq:utd}
%\end{eqnarray}




\begin{eqnarray}
U_T(p) & = & U_T(p,0,\ii n)
\\
U_T(p,k,n) & = & \big(  e:(\Sigma - [n]^0)   \then \efail\then \Stop \big)
\label{eq:uta}
\\ & & \extchoice \nonumber
\\ & & (\minaccs(n) = \{ \varnothing \})   \&   \big( \epass \then \Stop \big)
\label{eq:utb}
\\ & & \extchoice \nonumber
\\ & & (k < p) \& \big( e:[P/s]^0   \then U_T(p,k+1,t(n,e)) \big)
\label{eq:utc}
\\ & & \extchoice \nonumber
\\ & & (k = p) \& \big( \epass\then \Stop  \big)
\label{eq:utd}
\end{eqnarray}

It is easy to see that the tests $U_T(p)$ satisfy the properties
\begin{eqnarray}
\label{eq:ifpaT}
  &  & U_T(p)/s = U_T(p,\#s,G(P)/s)
\\
\label{eq:ifpbT}
e\not\in [P/s]^0 & \implies & U_T(p)/s.e = (\efail\then\Stop)
\end{eqnarray}
proven in Lemma~\ref{lemma:ufproperties} for $U_F(p)$ for traces $s\in\trc(P)$.


%
%The difference between adaptive tests $U_T(p)$ for trace refinement and
%$U_F(p)$ for failures refinement consists in the fact that the former do not
%``probe'' the SUT with respect to minimal sets of events to be accepted
%without blocking.

% ==========================================================================
The existence of complete, finite test suites is expressed in analogy to
Theorem~\ref{th:failurestest}. A noteworthy difference is that the complete
suite for trace refinement just needs the single adaptive test case
$U_T(pq-1)$, while failures refinement required the execution of $\{
U_F(0),\dots,U_F(pq-1)\}$. The reason for this is that $U_T(pq-1)$ identifies
trace errors for all traces up to length $pq$, while $U_F(pq-1)$ only probes
for erroneous acceptances at the end of each trace of length $(pq -1)$.
Since $U_T(pq-1)$ never blocks any $Q$-event before terminating, the pass 
criterion can be based on trace refinement instead of failures refinement
as required in (\ref{eq:passF}).
%
\begin{equation}
\label{eq:passT}
Q\ \pass\ U_T(p) \defs (\epass\then\Stop) \lessdet_T (Q\parallel[\Sigma] U_T(p)) \hide \Sigma
\end{equation}
%


\begin{theorem}\label{th:tracetest}
Let $P$ be a non-terminating, divergence-free CSP process over alphabet $\Sigma$ whose
normalised transition graph $G(P)$ has $p$ states. Define fault domain ${\cal
D}$ as the set of all non-terminating, divergence-free CSP processes over alphabet $\Sigma$,
whose transition graph has at most $q$ states with $q \ge p$. Then the test
suite
\[
\TS_T = \{ U_T(pq-1)   \}
\]
is complete with respect to ${\cal F} = (P,\lessdet_T,{\cal D})$.
\xbox
\end{theorem}
\begin{proof}
The theorem follows directly  
from Step~1 in the proof of Lemma~\ref{lemma:mainfsound} and
Case~1 in the proof of Lemma~\ref{lemma:mainfexhaustive}.
\xbox
\end{proof}
 