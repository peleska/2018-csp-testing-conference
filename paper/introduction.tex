% =========================================================================

\section{Introduction}
\label{sec:intro}

% =================================================================================

\subsubsection*{Motivation}

Model-based testing (MBT) is an active research field; results are currently
being evaluated and integrated into industrial verification processes by many
companies worldwide. This holds particularly for the embedded and
cyber-physical systems domains, where critical systems require rigorous
testing.

In the safety-critical domain, test suites with guaranteed fault coverage are
of particular interest. For black-box testing, guarantees can be given only
if certain hypotheses are satisfied. These hypotheses are usually specified
by a \emph{fault domain}:~a set of models that may or may not conform to a
given reference model. \emph{Complete} test strategies guarantee to accept
every  system under test (SUT) conforming to the reference model, and uncover
every conformance violation, provided that the SUT behaviour is captured by a
member of the fault domain.

Generation techniques for complete test suites have been developed for
various formalisms. Here, we tackle \emph{Communicating Sequential
Processes~(CSP)}~\cite{Hoare:1985:CSP:3921,Roscoe2010}. This is a mature
process algebra that has been shown to be well-suited for the description of
reactive control systems in many publications over almost five decades. Many
of these applications are described in~\cite{Roscoe2010} and in the
references there. Industrial success has also been reported; see, for
example, \cite{976937,DBLP:conf/prdc/ShiPK99,DBLP:conf/amast/ButhKPS97}.

% ==================================================================================

\subsubsection*{Contributions}

This paper presents complete black-box test suites for software and systems
modelled using CSP. They can be generated for non-terminating,
divergence-free, finite-state CSP processes with finite alphabets,
interpreted both in the trace and the failures semantics. Divergence freedom
is usually assumed in black-box testing, since it cannot distinguish between
divergence and deadlock.

Our results complement work in~\cite{DBLP:conf/pts/CavalcantiS17}. There,
fault domains are specified as collections of processes refining a ``most
general'' fault domain member. With that concept, complete test suites may be
finite or infinite. This result gives important insight into the theory of
fault-domain testing for CSP, but we are particularly interested in {\it
finite} suites when it comes to practical application.
While~\cite{DBLP:conf/pts/CavalcantiS17} may require additional criteria to
select tests from still infinite test suites, here, we further restrict fault
domains using a graph representation of processes (originally elaborated for
model checking) to obtain finite test suites.

Our approach to the definition of CSP fault domains is presented in this
paper. We take advantage of results on model checking of CSP
processes, where the failures semantics of a finite-state CSP process is
represented as a finite normalised transition graph, whose edges are labelled
by the events the process engages in, and whose nodes are labelled by minimal
acceptances or, alternatively, maximal
refusals~\cite{Roscoe:1994:chapter}. The maximal refusals express the
degree of nondeterminism present in a process state that is in
one-one-correspondence to a node of the normalised transition graph. Inspired
by the way fault-domains are specified for finite state machines (FSMs),
we define them here as the set of CSP processes whose normalised transition
graphs do not exceed the order (that is, the number of nodes) of the
reference model's graph by more than a given constant.

The main contribution of this paper is the construction of two test suites to
verify the conformance relations \emph{trace refinement} and \emph{failures
refinement} for any given non-terminating, non-divergent, finite-state,
finite alphabet CSP process. We prove their completeness with respect to
fault domains of the described type.  The existence of -- possibly infinite
-- complete test suites has been established for process algebras, and for
CSP in particular, by several
authors~\cite{Hennessy:1988:ATP:50497,Schneider:1995:OST:203471.203475,DBLP:conf/fm/PeleskaS96,peleska1997a,DBLP:conf/icfem/CavalcantiG07,DBLP:conf/pts/CavalcantiS17}.
To the best of our knowledge, this article is the first to present {\it
finite}, complete test suites associated with this class of fault domains and
conformance relations.

Our result is not a simple transcription of existing knowledge about finite,
complete test suites for FSMs. The capabilities of CSP to express
nondeterminism in a more fine-grained way than it is possible for FSMs
requires a more complex approach to testing for conformity in the failures
model than required for model-based testing against nondeterministic FSMs, as
published, for example,
in~\cite{hierons_testing_2004,DBLP:conf/hase/PetrenkoY14}. CSP distinguishes
between external choice, where the environment can control the behaviour of a
process, and internal choice where the behaviour is decided internally and
cannot be influenced by the environment. In contrast, FSMs specify
nondeterministic behaviour by offering more than one possible output for a
given input, and this output can be controlled by an FSM representing the
environment. Loosely speaking, this corresponds to external choice in CSP,
while there is no equivalent to internal choice in nondeterministic FSMs.
\fixme{alcc: I'm slightly uncomfortable with this claim. Can the environment
really control outputs? Do you not have tau (internal) events in FSM? I'm
unsure, but you guys know about FSMs much more than I do.}

Finally, we prove two optimality results here. (1)~We show that our approach
to testing the admissibility of an SUT's nondeterministic behaviour is
optimal in the sense that it cannot be established with fewer ``probes''
investigating the SUT's degree of nondeterminism. These probes are minimal
sets of events offered by the tests to the SUT, such that a refusal of the
SUT to engage into at least one of these events reveals a violation of
failures refinement. (2)~Furthermore, it is shown that the maximal length of
traces to be investigated in a complete test suite cannot be further reduced
without loosing the suite's capability to uncover any violation of trace or
failures refinement.

% ==================================================================================

\subsubsection*{Overview}
In Section~\ref{section:preliminaries}, we present the background
relevant to our work. In Section~\ref{sec:finitecompletefails}, finite
complete test suites for verifying failures refinement are presented. Test
suites checking trace refinement are a simplified version of the former
class; they are presented in Section~\ref{sec:finitecomplete}.
The optimality results are presented in Section~\ref{sec:complexity}, together with further
complexity considerations.
A sample test
suite is presented in Section~\ref{sec:case}. Our results are discussed in
Section~\ref{sec:conc}, where we also conclude. References to related work
are given throughout the paper where appropriate.

% ==================================================================================
