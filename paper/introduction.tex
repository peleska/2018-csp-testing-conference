% =========================================================================

\section{Introduction}
\label{sec:intro}

% =================================================================================

\subsubsection*{Motivation}

Model-based testing (MBT) is an  active research field that is currently
evaluated and integrated into industrial verification processes by many
companies worldwide. This holds particularly for the embedded and
cyber-physical systems domains, where critical systems require rigorous
testing.

While MBT is applied in different flavours, we consider the most effective
variant to be the one where test cases and concrete test data, as well as
checkers for the expected results~(\emph{test oracles}), are automatically
generated from a reference model. This guarantees maximal return of the
investment of time and effort to create the test model. The test suites
generated in this way, however, usually have different test strength,
depending on the generation algorithms applied.

For the safety-critical domain, test suites with guaranteed fault coverage
are of particular interest. For black-box testing, guarantees can be given
only if certain hypotheses are satisfied. These hypotheses are usually
specified by a \emph{fault domain}:~a set of models that may or may not
conform to the SUT. The so-called \emph{complete} test strategies guarantee
to uncover every conformance violation of the SUT with respect to a reference
model, provided that the true SUT behaviour is captured by a member of the
fault domain.

Generation methods for complete test suites have been developed for various
modelling formalisms. In this paper, we use \emph{Communicating Sequential
Processes (CSP)}~\cite{Hoare:1985:CSP:3921,Roscoe2010}. This is a mature
process-algebraic approach that has been shown to be well-suited for the
description of reactive control systems in many publications over almost five
decades. Industrial success has also been reported.

% ==================================================================================

\subsubsection*{Contributions}

This paper presents complete black-box test suites for CSP processes that are
divergence-free\footnote{The assumption of divergence freedom is usually
applied in black-box testing, since it cannot distinguish between divergence
and deadlock.} and interpreted both in the trace and the failures semantics.
Our results complement work by two of the authors
in~\cite{DBLP:conf/pts/CavalcantiS17}. There, fault domains are specified as
collections of processes refining a  ``most general'' fault domain member.
With that concept, complete test suites may be finite or infinite. Yhis gives
important insight into the theory of fault-domain testing for CSP, but we are
particularly interested in finite suites when it comes to practical
application. While~\cite{DBLP:conf/pts/CavalcantiS17} requires additional
criteria to select tests from infinite test suites, here, we further restrict
fault domains using a graph representation of processes used in model
checking to obtain test suites that are finite.

Our complementary approach to the definition of CSP fault domains is
presented in this paper. We observe that every finite-state CSP process can
be semantically represented as a finite normalised transition graph, whose
edges are labelled by the events the process engages in, and whose nodes are
labelled by minimal acceptances or, alternatively, maximal
refusals~\cite{Roscoe:1994:CME:197600}. The maximal refusals express the
degree of nondeterminism present in a given process state that is in
one-one-correspondence to a node of the normalised transition graph. Inspired
by the way that fault-domains are specified for finite state machines, we
define them as the set of CSP processes whose normalised transition graphs do
not exceed the size of the reference model's graph by more than a given
constant.

The main contribution of this paper is the proof that for fault domains of
the described type, finite, complete test suite generation methods can be
given for testing against trace and failures refinement and equivalence. The
existence of -- possibly infinite -- complete test suites has been
established for process algebras, and for CSP in particular, by several
authors~\cite{Hennessy:1988:ATP:50497,Schneider:1995:OST:203471.203475,DBLP:conf/fm/PeleskaS96,peleska1997a,DBLP:conf/pts/CavalcantiS17}.
To the best of our knowledge, however, this article is the first to present
{\it finite}, complete test suites associated with this class of fault
domains and conformance relations.

It should be noted that the presented results are of a theoretical
nature:~despite being finite, the resulting test suite size will still be too
large to be applied in practice to real-world problems. We discuss a number
of promising options how the test suite size can be reduced to become
practically applicable.

% ==================================================================================

\subsubsection*{Overview}
In Section~\ref{section:preliminaries},
we present the background material relevant to our work.

@todo


% ==================================================================================
