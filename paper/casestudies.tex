\section{Applications}
\label{sec:case}




% =======================================================================
\subsection{Testing for Trace Refinement}
 



% =======================================================================
\subsection{Testing for Failures Refinement}

For implementing the test case $U_F(p)$ with sub-processes $U_F(p,s)$, 
it is advisable to avoid an enumeration
of traces $s$ the reference process has run through. Instead, we calculate
the following auxiliary functions from $P$'s transition graph.
\begin{eqnarray*}
\text{initials} & : & N \fun \power (\Sigma) 
\\
\text{minHit} & : & N \fun \power\power(\Sigma)
\end{eqnarray*}
In a state $n = G(P)/s$, the set $\text{initials(n)}$ equals the events labelling
outgoing edges of $n$, so  $\text{initials(n)} = [P/s]^0$. Function $\text{minHit}$
maps $n$ to the set of all minimal hitting sets associated with the minimal acceptances
of $n$, so $\text{minHit}(n) = \text{minHit}(P/s)$. Then, using the transition function
of $t$ in addition to the two auxiliary functions, $U_F(p)$ can be re-written
as the failures-equivalent CSP process
\begin{eqnarray}
U_F^1(p) & = & U_F^1(p,0,\ii n)
\\
U_F^1(p,k,n) & = & \big( e:(\Sigma - \text{initials}(n)) \then \efail\then \Stop \big)
\label{eq:xufa}
\\ & & \extchoice \nonumber
\\ & & (\text{initials}(n) = \varnothing)    \&   \big( \epass \then \Stop \big)
\label{eq:xufb}
\\ & & \extchoice \nonumber
\\ & & (k < p) \& \big( e:\text{initials}(n) \then U_F^1(p,(k+1),t(n,e)) \big)
\label{eq:xufc}
\\ & & \extchoice \nonumber
\\ & & (k = p) \& \big( \sqcap_{H\in\text{minHit}(n)} ( e:H \then \epass \then\Stop   )  \big)
\label{eq:xufd}
\end{eqnarray}


\begin{example}
\label{ex:uf1tests}

\end{example}
 
 


% ====================================================================== 